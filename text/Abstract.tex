\pdfbookmark[1]{Abstract}{Abstract} % Bookmark name visible in a PDF viewer

\begingroup
\let\clearpage\relax
\let\cleardoublepage\relax
\let\cleardoublepage\relax

\chapter*{Abstract} % Abstract name
The development of programming languages needs to reflect important changes in the way
programs execute. In recent years, this has included the development of parallel programming
models (in reaction to the multi-core revolution) or improvements in data access technologies.
This thesis is a response to another such revolution -- the diversification of devices and
systems where programs run.

The key point made by this thesis is the realization that an execution environment or
a \emph{context} is fundamental for writing modern applications and that programming
languages should provide abstractions for programming with context and verifying how
it is accessed.

We identify a number of program properties that were not connected before, but model some notion
of context. Our examples include tracking different execution platforms (and their versions)
in cross-platform development, resources available in different execution environments (\eg~GPS
sensor on a phone and database on the server), but also more traditional notions such as
variable usage (\eg~in liveness analysis and linear logics) or past values in
stream-based dataflow programming. Our first contribution is the discovery of the connection
between the above examples and their novel presentation in the form of calculi (\emph{coeffect systems}).
The presented type systems and formal semantics highlight the relationship between different
notions of context.

Our second contribution is the definition of two unified coeffect calculi that capture the common
structure of the examples. In particular, our \emph{flat coeffect calculus} models languages
with contextual properties of the execution environment and our \emph{structural coeffect
calculus} models languages where the contextual properties are attached to the variable usage.
We define the semantics of the calculi in terms of category theoretical structure of an
\emph{indexed comonad} (based on dualisation of the well-known monad structure), use it
to define operational semantics and prove type safety of the calculi.

Our third contribution is a novel presentation of our work in the form of web-based
\emph{interactive essay}. This provides a simple implementation of three context-aware programming
languages and lets the reader write and run simple context-aware programs, but also explore the
theory behind the implementation including the typing derivation and semantics.


\endgroup

\vfill
