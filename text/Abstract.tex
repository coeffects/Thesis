% Abstract

\pdfbookmark[1]{Abstract}{Abstract} % Bookmark name visible in a PDF viewer

\begingroup
\let\clearpage\relax
\let\cleardoublepage\relax
\let\cleardoublepage\relax

\chapter*{Abstract} % Abstract name
The development of programming languages needs to reflect important changes in the industry.
In recent years, this included \eg~the development of parallel programming models (in reaction 
to the multi-core revolution). This thesis is a response to another such revolution -- the 
diversification of devices and systems where programs run. 

We develop the foundations for statically typed functional languages that understand 
the \emph{context} or environment in which programs execute. Such context includes different 
platforms (and their versions) in cross-platform applications, resources available in different 
execution environments (\eg \; GPS sensor on a phone and database on the server), but also more 
traditional notions such as variable context (tracking variable usage in static analyses) or 
past values in stream-based data-flow programming.

The thesis presents three \emph{coeffect} calculi that capture different notions of 
context-awareness: \emph{flat} calculus capturing contextual properties of the execution 
environment, \emph{structural} calculus capturing contextual properties related to 
variable usage and \emph{meta-language} calculus which allows reasoning about multiple notions
of context-dependence in a single language and provides pathway to embedding the other two
calculi in existing languages.

Although the focus of this thesis is on the syntactical properties of the presented 
systems, we also discuss their category-theoretical motivation. We introduce the notion of
an \emph{indexed} comonad (the dual of monads) and show how they provide semantics of the
presented three calculi. 

\endgroup			

\vfill