% ==================================================================================================

\setcounter{chapter}{3}
\chapter{Types for flat coeffects} 
\label{ch:flat} 

Successful programming language abstractions need to generalize a wide range of recurring
problems while capturing the key commonalities. These two aims are typically in opposition -- 
more general abstractions are less powerful, while less general abstractions cannot be
used as often.

In the previous chapter, we outlined a number of systems that capture how computations
access the environment in which they are executed. We identified two kinds of systems --
\emph{flat systems} capturing whole-context properties and \emph{structural systems} capturing 
per-variable properties. As we show in Section~X, the systems can be further unified using a 
single abstraction, but such abstraction is  \emph{less powerful} -- \ie~its generality hides 
useful properties that we can see when we consider the systems separately. For this reason, we 
discuss \emph{flat coeffects} and \emph{structural coeffects} separately.

In this and the next chapter, we discuss the type system and the semantics of flat coeffect 
systems, respectively. In this chapter, we develop a parameterized type system for flat coeffect
systems and we study its general syntactic properties. We also consider variations of the type system
that resolve the ambiguity of coeffectful lambda abstraction for concrete instances of the system.
In the next chapter, we give operational meaning of concrete coeffect languages using the unified
system and we discuss their safety. 

\paragraph{Chapter structure and contributions}
\begin{itemize}
\item We present a \emph{flat coeffect calculus} as a type system that is parameterized by a 
  \emph{flat coeffect algebra} (Section~\ref{sec:flat-calculus}). We show that the system can be 
  instantiated to obtain three of the systems discussed in Section~\ref{sec:applications-flat},
  namely implicit parameters, liveness and dataflow.

\item The type system permits multiple typing derivation for certain programs due to the ambiguity
  inherent in conextual lambda abstraction rule. In Section~\ref{sec:flat-unique}, we discuss
  variations of the type system that resolve the ambiguity and give a unique typing derivation
  for the three coeffect systems covered in this chapter.
  
\item We discuss syntactic properties of the calculus, covering type-pre\-ser\-vation for call-by-name
  and call-by-value reduction (Section~\ref{sec:flat-syntax}). We also extend the calculus
  with subtyping and pairs (Section~\ref{sec:flat-exts}). These two sections motivate the laws
  of the flat coeffect algebra.
\end{itemize}

% ==================================================================================================
%
%    ###
%     #  #    # ##### #####   ####  #####  #    #  ####  ##### #  ####  #    #
%     #  ##   #   #   #    # #    # #    # #    # #    #   #   # #    # ##   #
%     #  # #  #   #   #    # #    # #    # #    # #        #   # #    # # #  #
%     #  #  # #   #   #####  #    # #    # #    # #        #   # #    # #  # #
%     #  #   ##   #   #   #  #    # #    # #    # #    #   #   # #    # #   ##
%    ### #    #   #   #    #  ####  #####   ####   ####    #   #  ####  #    #
%
% ==================================================================================================

\section{Introduction}
\label{sec:flat-intro}

In the previous chapter, we looked at three important examples of systems that track whole-context 
properties. The type systems for whole-context liveness (Section~\ref{sec:applications-flat-live}) 
and whole-context data-flow (Section~\ref{sec:applications-flat-dataflow}) have a very similar 
structure. First, their lambda abstraction duplicates the requirements. Given a body with context 
requirements $\cclrd{r}$, the declaration site context \emph{as well as} the function arrow are
annotated with $\cclrd{r}$. Second, their application arises from the combination of \emph{sequential} 
and \emph{pointwise} composition.

The system for tracking of implicit parameters and similar (Section~\ref{sec:applications-flat-impl})
differ in two ways. In lambda abstraction, they split the context requirements between the 
declaration site and the call site and they use only a single operator on the indices, typically $\cup$.

Despite the differences, the systems fit the same unified framework. This becomes apparent when
we consider the categorical structure (Section~\ref{sec:flat-semantics}). However, rather than 
starting from the semantics, we first explain how the systems can be unified syntactically
(Section~\ref{sec:flat-calculus-lambda}) and then provide the semantics as a justification.

The development in this chapter can be seen as a counterpart to the well-known development of 
\emph{effect systems} \cite{effects-gifford}. The Chapter~\ref{ch:transl} then links \emph{coeffects} 
with \emph{comonads} in the same way in which effect systems can be linked with monads
\cite{monad-notions}. The syntax and type system of the flat coeffect calculus follows 
a similar style as effect systems \cite{effects-polymorphic,effects-talpin-et-al}, but differs
in the structure, as explained in the previous chapter, most importantly in lambda abstraction
(the relationship with monads is further discussed in Section~\ref{sec:flat-related}).


% ==================================================================================================
%                                                            
%     #####
%    #     #   ##   #       ####  #    # #      #    #  ####
%    #        #  #  #      #    # #    # #      #    # #
%    #       #    # #      #      #    # #      #    #  ####
%    #       ###### #      #      #    # #      #    #      #
%    #     # #    # #      #    # #    # #      #    # #    #
%     #####  #    # ######  ####   ####  ######  ####   ####
%                                                           
% ==================================================================================================

\section{Flat coeffect calculus}
\label{sec:flat-calculus}

The flat coeffect calculus is defined in terms of \emph{flat coeffect algebra}, which defines
the structure of context annotations, such as $\cclrd{r}, \cclrd{s}, \cclrd{t}$. These can be
sets of implicit parameters, versions represented as integers or other values. The expressions of 
the calculus are those of the $\lambda$-calculus with \emph{let} binding. We also include a type 
\ident{num} as an example of a concrete base type with numerical constants written as $n$:
%
\begin{equation*}
\begin{array}{rcl}
e &::=& x \sep n \sep \lambda x:\tau.e \sep e_1~e_2 \sep \kvd{let}~x = e_1~\kvd{in}~e_2\\
\tau &::=& \ident{num} \sep \tau_1 \xrightarrow{\cclrd{r}} \tau_2
\end{array}
\end{equation*}
%
Note that the lambda abstrction in the syntax is written in the Church-style and requires a type
annotation. This will be used in Section~\ref{sec:flat-unique} where we discuss how to find a 
unique typing derivation for context-aware computations. Using Church-style lambda abstraction,
we can directy focus on the more interesting problem of finding unique \emph{coeffect annotations} 
rather than solving the problem of type reconstruction. 

We discuss subtyping and pairs in Section~\ref{sec:flat-exts}. The type $\tau_1 \xrightarrow{\cclrd{r}} \tau_2$
represents a function from $\tau_1$ to $\tau_2$ that requires additional context $\cclrd{r}$.
It can be viewed as a pure function that takes $\tau_1$ \emph{with} or \emph{wrapped in} a 
context $\cclrd{r}$. 

In the categorically-inspired translation in the next chapter, the function 
$\tau_1 \xrightarrow{\cclrd{r}} \tau_2$ is translated into a function 
$C^{\cclrd{r}} \tau_1 \rightarrow \tau_2$. However, the type constructor $C^{\cclrd{r}}$
does not itself exist as a syntactical value in the coeffect calculus. This is because we use 
comonads to define the \emph{semantics} rather than \emph{embedding} them into the language as in 
the meta-language approaches (the distinction has been discussed in Section~\ref{sec:path-sem-langs}).
The annotations $\cclrd{r}$ are formed by an algebraic structure discussed next.

%---------------------------------------------------------------------------------------------------

\subsection{Reconciling lambda abstraction}
\label{sec:flat-calculus-lambda}

Recall the lambda abstraction rules for the implicit parameters system (annotating the context
with sets of required parameters) and the data-flow system (annotating the context with the
number of past required values):
%
\begin{equation*}
\tyrule{param}
  {\coctx{\Gamma, x\!:\!\tau_1}{\cclrd{r} \cup \cclrd{s}} \vdash e : \tau_2}
  {\coctx{\Gamma}{\cclrd{r}} \vdash \lambda x.e : \tau_1 \xrightarrow{\cclrd{s}} \tau_2 }
\;
\tyrule{df1}
  {\coctx{\Gamma, x\!:\!\tau_1}{\cclrd{n}} \vdash e : \tau_2}
  {\coctx{\Gamma}{\cclrd{n}} \vdash \lambda x.e : \tau_1 \xrightarrow{\cclrd{n}} \tau_2 }
\end{equation*}
%
In order to capture both systems using a single calculus, we need a way of unifying the two
systems. For the data-flow system, this can be achieved by over-approximating the number of 
required past elements:
%
\begin{equation*}
\tyrule{df2}
  {\coctx{\Gamma, x\!:\!\tau_1}{\cclrd{\textnormal{min}(n, m)}} \vdash e : \tau_2}
  {\coctx{\Gamma}{\cclrd{n}} \vdash \lambda x.e : \tau_1 \xrightarrow{\cclrd{m}} \tau_2 }
\end{equation*}
%
The rule (\emph{df1}) is admissible in a system that includes the (\emph{df2}) rule. 
Furthermore, if we include sub-typing rule (on annotations of functions) and sub-coeffecting rule (on 
annotations of contexts), then the reverse is also true -- because 
$\cclrd{\textit{min}(n, m)} \leq \cclrd{m}$ and $\cclrd{\textit{min}(n, m)} \leq \cclrd{n}$.
In other words (\emph{df1}) is more precise, but (\emph{df2}) gives a sound over-approximation
with a structure that can be unified with (\emph{param}).

Using a rule such as (\emph{df2}) allows us to give a unified formulation of the flat coeffect 
calculus in Section~\ref{sec:flat-calculus-types}, however the coeffect-specific handling of
lambda abstraction remains important in practical implementation in order to obtain a unique
typing derivation for each coeffect program as discussed in Section~\ref{sec:flat-unique}
(and in the implementation in Chatper~\ref{ch:impl}).

%---------------------------------------------------------------------------------------------------

\subsection{Understanding flat coeffects}
\label{sec:flat-calculus-undestanding}

Before looking at the type system in Figure~\ref{fig:flat-types}, let us clarify how the rules
should be understood. The coeffect calculus provides both analysis of context dependence (type 
system) and semantics for context (how it is propagated). These two aspects provide different
ways of reading the judgements $\coctx{\Gamma}{\cclrd{r}} \vdash e : \tau$ and the typing rules
used to define it.

\begin{itemize}
\item \textsc{Analysis of context dependence.}
Syntactically, coeffect annotations $\cclrd{r}$ model \emph{context requirements}. This means
we can over-approximate them and require more than is actually needed at runtime. 

Syntactically, the typing rules should be read top-down. In function application, the context 
requirements of multiple assumptions (arising from two sub-expressions) are \emph{merged}; in 
lambda abstraction, the requirements of a single expression (the body) are split between
the declaration site and the call site.

\item \textsc{Semantics of context passing.}
Semantically, coeffect annotations $\cclrd{r}$ mo\-del \emph{contextual capabilities}. This means
that we can throw away capabilities, if a sub-expression requires fewer than we 
currently have.

Semantically, the typing rules should be read bottom-up. In application, the capabilities 
provided to the term $e_1~e_2$ are \emph{split} between the two sub-expressions; in abstraction,
the capabilities provided by the call site and declaration site are \emph{merged} and passed
to the body.
\end{itemize}

The reason for this asymmetry follows from the fact that the context appears in a \emph{negative
position} in the semantic model (Section~\ref{sec:flat-semantics}). It means that we need to be
careful about using the words \emph{split} and \emph{merge}, because they can be read as meaning
exactly the opposite things. To disambiguate, we always use the term \emph{context requirements} 
when using the syntactic view, especially in the rest of Chapter~\ref{chf:flat}, and 
\emph{context capabilities} or just \emph{available context} when using the semantic view, 
especially in Chapter~\ref{ch:transl}.

%---------------------------------------------------------------------------------------------------

\subsection{Flat coeffect algebra}
\label{sec:flat-calculus-algebra}

To make the flat coeffect system general enough, the algebra consists of three operations.
Two of them, $\cseq$ and $\cpar$, represent the \emph{sequential} and \emph{pointwise} composition, 
which are mainly used in function application. The third operator, $\czip$ is used in lambda 
abstraction and represents \emph{splitting} of context requirements (or, semantically, \emph{merging} 
of available context capabilities). 

In addition to the three operations, we also require two special values used to annotate
variable access and constant access and a relation that defines the ordering.

\begin{definition}
A \emph{\cclrd{flat coeffect algebra}} $(\C, \cseq, \cpar, \czip, \cunit, \czero, \cleq)$ is a set 
$\C$ together with elements $\cunit, \czero \in \C$, relation $\cleq$ and binary operations 
$\cseq, \cpar, \czip$ such that $(\C, \cseq, \cunit)$ is a monoid, $(\C, \cpar, \czero)$ is an
idempotent monoid, $(\C, \cleq)$ is a pre-order and $(\C, \czip)$ is a band (idempotent semigroup). 
That is, for all $r,s,t\in \C$: 
%
\begin{equation*}
\begin{array}{ccc}
r \;\cseq\; (s \;\cseq\; t) = (r \;\cseq\; s) \;\cseq\; t &&
\cunit \;\cseq\; r = r = r \;\cseq\; \cunit 
\\
r \;\cpar\; (s \;\cpar\; t) = (r \;\cpar\; s) \;\cpar\; t &
r\; \cpar\; r = r~~~~&
\czero \;\cpar\; r = r = r \;\cpar\; \czero 
\\
r \;\czip\; (s \;\czip\; t) = (r \;\czip\; s) \;\czip\; t &&
r\; \czip\; r = r \qquad
\\
\textnormal{if}~~r\; \cleq\; s ~~\textnormal{and}~~s\; \cleq\; t~~\textnormal{then}~~r\; \cleq\; t &&
t\; \cleq\; t \\[0.5em]
\end{array}
\end{equation*}
%
In addition, the following distributivity axioms hold:
\begin{align*}
\quad (\cclrd{r}\, \cpar\, \cclrd{s}) \;\cseq\; \cclrd{t} & = (\cclrd{r} \,\cseq\, \cclrd{t}) \;\cpar\; (\cclrd{s}\, \cseq\, \cclrd{t}) \\
\quad \cclrd{t} \;\cseq\; (\cclrd{r}\, \cpar\, \cclrd{s}) & = (\cclrd{t} \,\cseq\, \cclrd{r}) \;\cpar\; (\cclrd{t}\, \cseq\, \cclrd{s})
\end{align*}
\end{definition}

\noindent
In two of the three systems, some of the operators of the flat coeffect algebra coincide, but
in the data-flow system all three are distinct. Similarly, the two special elements  
coincide in some, but not all systems. The required laws are motivated by the aim to capture
common properties of the three examples, without unnecessarily restricting the system:

\begin{itemize}
\item The monoid $(\C, \cseq, \cunit)$ represents \emph{sequential} composition of (semantic)
functions. The laws of a monoid are required in order to form a category structure in the 
semantics (Section~\ref{sec:flat-semantics}).

\item The idempotent monoid $(\C, \cpar, \czero)$ represents \emph{pointwise} composition, 
\ie~the case when the same context is passed to multiple (independent) computations. The monoid 
laws guarantee that usual syntactic transformations on tuples and the unit value 
(Section~\ref{sec:flat-exts}) preserve the coeffect. Idempotence holds for all our examples
and allows us to unify the flat and structural systems in Chapter~\ref{ch:unified}.

\item For the $\czip$ operation, we require associativity and idempotence. The idempotence
requirement makes it possible to duplicate the coeffects and place the same requirement on both
call site and declaration site. Using the example from Section~\ref{sec:flat-calculus-lambda},
this guarantees that the rule (\emph{df1}) is not a special case, but can always be derived 
from (\emph{df2}). In some cases, the operator forms a monoid with the unit being the greatest 
element of the set $\C$. 
\end{itemize}

\noindent
It is worth noting that, in some of the systems, the operators $\cpar$ and $\czip$ are the least 
upper bound and the greatest lower bounds of a lattice. For example, in data-flow computations, they 
are \emph{max} and \emph{min} respectively. However, this duality does not hold for implicit parameters
(we discuss lattice-based formulation of coeffects in Section~\ref{sec:unified-impl-semilattice}. 

Using the syntactic reading, the two operators represent \emph{merging} and \emph{splitting} of context 
requirements -- in the (\emph{abs}) rule, $\czip$ appears in the assumption and the combined context 
requirements of the body are split between two positions in the conclusions; in the (\emph{app}) rule, 
$\cpar$ appears in the conclusion and combines two context requirements from the assumptions.

\paragraph{Ordering.}

The flat coeffect algebra requires a pre-order relation $\cleq$. This will be used to introduce
sub-coeffecting and subtyping in Section~\ref{sec:flat-exts-sub}, but we make it a part of the flat
coeffect algebra, as it will be useful for characterization of different kinds of coeffect calculi.
When the idempotent monoid $(\C, \cpar, \czero)$ also has the commutative property (\ie~forms a 
semi-lattice), the $\cleq$ relation can be defined as the ordering of the semi-lattice:
%
\begin{equation*}
\cclrd{r} \;\cleq\; \cclrd{s} \;\Longleftrightarrow\; \cclrd{r} \;\cpar\; \cclrd{s} \;=\; \cclrd{s}
\end{equation*}
%
This definition is consistent with all three examples that motivate flat coeffect calculus, but
it cannot be used with the structural coeffects (where it fails for the bounded reuse 
calculus) and so we choose not to use it.

Furthermore, the $\cunit$ coeffect is often the top or the bottom element of the semi-lattice. 
As discussed in Section~\ref{sec:flat-syntax}, when this is the case, we are able to prove certain 
syntactic properties of the calculus.

%---------------------------------------------------------------------------------------------------

\begin{figure}[t]
\begin{equation*}
\tyrule{var}
  {x : \tau \in \Gamma}
  {\coctx{\Gamma}{\cunit} \vdash x : \tau }
\end{equation*}
\begin{equation*}
\tyrule{const}
  {~}
  {\coctx{\Gamma}{\czero} \vdash n : \ident{num} }
\end{equation*}
\begin{equation*}
\tyrule{app}
  {\coctx{\Gamma}{\cclrd{r}} \vdash e_1 : \tau_1 \xrightarrow{\cclrd{t}} \tau_2 &
   \coctx{\Gamma}{\cclrd{s}} \vdash e_2 : \tau_1 }
  {\coctx{\Gamma}{\cclrd{r} \;\cpar\; (\cclrd{s} \,\cseq\, \cclrd{t})} \vdash e_1~e_2 : \tau_2}
\end{equation*}
\begin{equation*}
\tyrule{abs}
  {\coctx{\Gamma, x\!:\!\tau_1}{\cclrd{r}\;\czip\;\cclrd{s}} \vdash e : \tau_2}
  {\coctx{\Gamma}{\cclrd{r}} \vdash \lambda x\!:\!\tau_1.e : \tau_1 \xrightarrow{\cclrd{s}} \tau_2 }
\end{equation*}
\begin{equation*}
\tyrule{let}
  { \coctx{\Gamma}{\cclrd{r}} \vdash e_1 : \tau_1 &
    \coctx{\Gamma, x\!:\!\tau_1}{\cclrd{s}} \vdash e_2 : \tau_2}
  {\coctx{\Gamma}{\cclrd{s} \;\cpar\; (\cclrd{s} \,\cseq\, \cclrd{r})} \vdash \kvd{let}~x=e_1~\kvd{in}~e_2 : \tau_2 }
\end{equation*}

\figcaption{Type system for the flat coeffect calculus}
\label{fig:flat-types}
\end{figure}

%---------------------------------------------------------------------------------------------------

\subsection{Flat coeffect types}
\label{sec:flat-calculus-types}

The type system for flat coeffect calculus is shown in Figure~\ref{fig:flat-types}. Variables 
(\emph{var}) and constants (\emph{const}) are annotated with special values provided by the 
coeffect algebra. 

The (\emph{abs}) rule is defined as discussed in Section~\ref{sec:flat-calculus-lambda}. The
body is annotated with context requirements $\cclrd{r} \,\czip\, \cclrd{s}$, which are then split
between the context-requirements on the declaration site $\cclrd{r}$ and context-requirements on
the call site $\cclrd{s}$. Examples of the $\czip$ operator are discussed in the next section.

In function application (\emph{app}), context requirements of both expressions and the 
function are combined. As discussed in Chapter~\ref{ch:applications}, the pointwise composition
$\cpar$ is used to combine the context requirements of the expression representing a function 
$\cclrd{r}$ and the context requirements of the argument, sequentially composed with the 
context-requirements of the function $\cclrd{s}\, \cseq \,\cclrd{t}$.

The type system also includes a rule for let-binding. The rule is \emph{not} equivalent to the
typing derivation for $(\lambda x.e_2)~e_1$, but it corresponds to \emph{one} possible typing 
derivation. As we show in \ref{sec:flat-exts-let}, the typing used in (\emph{let}) is more
precise than the general rule that can be derived from $(\lambda x.e_2)~e_1$. Additional 
constructs such as pairs, sub-coeffecting and sub-typing are covered in Section~\ref{sec:flat-exts}.

%---------------------------------------------------------------------------------------------------

\subsection{Examples of flat coeffects}
\label{sec:flat-calculus-examples}

The flat coeffect calculus generalizes the flat systems discussed in 
Section~\ref{sec:applications-flat} of the previous chapter. We can instantiate it to a specific
use just by providing a flat coeffect algebra. The following summary defines the systems for implicit 
parameters, liveness and data-flow. For the latter two, the general calculus has a lambda abstraction
that is compatible with those discussed in Chapter~\ref{ch:flat}, but includes implicit sub-coeffecting.

\begin{example}[Implicit parameters]
Assuming \ident{Id} is a set of implicit parameter names, the flat coeffect algebra 
is formed by $(\mathcal{P}(\ident{Id}), \cup, \cup, \cup, \emptyset, \emptyset, \subseteq)$.
\end{example}

\noindent
For simplicity, we assume that all parameters have the same type $\ident{num}$ and so the annotations only
track sets of names. The definition uses a set union for all three operations. Both variables and
constants are are annotated with $\emptyset$ and the ordering is defined by $\subseteq$. The 
definition satisfies the flat coeffect algebra laws because $(S, \cup, \emptyset)$ is an idempotent, 
commutative monoid. The language has additional syntax for defining an implicit parameter and for
accessing it, together with associated typing rules:
%
\begin{equation*}
\quad~ e ~::=~ \ldots \sep \ident{?p} \sep \kvd{let}~\ident{?p}~=~e_1~\kvd{in}~e_2
\end{equation*}
\begin{equation*}  
\tyrule{param}
  { ~ }
  { \coctx{\Gamma}{\cclrd{ \{ \ident{?p} \} }} \vdash \ident{?p} : \ident{num} }
\end{equation*}
\begin{equation*}
\tyrule{letpar}
  { \coctx{\Gamma}{\cclrd{r}} \vdash e_1 : \tau_1 &
    \coctx{\Gamma}{\cclrd{s}} \vdash e_2 : \tau_2}
  {\coctx{\Gamma}{\cclrd{r} \;\cclrd{\cup\; (\cclrd{s} \setminus \{ \ident{?p}) \}} } \vdash \kvd{let}~\ident{?p}=e_1~\kvd{in}~e_2 : \tau_2 }
\end{equation*}
%
The (\emph{param}) rule specifies that the accessed parameter $\ident{?p}$ needs to be in the set of required
parameters $\cclrd{r}$. As discussed earlier, we use the same type $\ident{num}$ for all parameters, but
it is possible to define a coeffect calculus that uses mappings from names to types (care is needed to 
avoid assigning multiple types to a parameter of the same type).

The (\emph{letpar}) rule is the same as the one discussed in Section~\ref{sec:applications-flat-impl}.
As both of the rules are specific to implicit parameters, we write the operations on coeffects directly
using set operations -- coeffect-specific operations such as set subtraction are not a part of the
unified coeffect algebra.

\begin{example}[Liveness]
Let $\mathcal{L}=\{ \ident{L}, \ident{D} \}$ be a two-point lattice such that $\ident{D} \sqsubseteq \ident{L}$
with a join $\sqcup$ and meet $\sqcap$. The flat coeffect algebra for liveness is then formed by
$(\mathcal{L}, \sqcap, \sqcup, \sqcap, \ident{L}, \ident{D}, \sqsubseteq)$.
\end{example}

\noindent
The liveness example is interesting because it does not require any additional syntactic extensions
to the langauge. It annotates constants and variables with $\ident{D}$ and $\ident{L}$, respectively
and it captures how those annotation propagate through the remaining language constructs.

As in Section~\ref{sec:applications-flat-live}, sequential composition $\cseq$ is modelled by 
the meet operation $\sqcap$ and pointwise composition $\cpar$ is modelled by join $\sqcup$. 
The two-point lattice is a commutative, idempotent monoid. Distributivity 
$(r \sqcup s) \sqcap t = (r \sqcap t) \sqcup (s \sqcap t)$ does not hold for \emph{every} 
lattice, but it trivially holds for the two-point lattice used here.

The definition uses join $\sqcup$ for the $\czip$ operator that is used by lambda abstraction.
This means that, when the body is live $\ident{L}$, both declaration site and call site are 
marked as live $\ident{L}$. When the body is dead $\ident{D}$, the declaration site and call site
can be marked as dead $\ident{D}$, or as live $\ident{L}$. The latter is less precise, but 
permissible over-approximation, which could otherwise be obtained via sub-typing.

\begin{example}[Data-flow]
In data-flow, context is annotated with natural numbers and the flat coeffect algebra is formed 
by $(\mathbb{N}, +, \mathit{max}, \mathit{min}, 0, 0, \leq)$.
\end{example}

\noindent
As discussed earlier, sequential composition $\cseq$ is represented by $+$ and pointwise 
composition $\cpar$ uses $\emph{max}$. For data-flow, we need a third separate operator for
lambda abstraction. Annotating the body with $\emph{min}(\cclrd{r}, \cclrd{s})$ ensures that
both call site and declaration site annotations are equal or greater than the annotation 
of the body. As with liveness, this allows over-approximation. 

As required by the laws, $(\mathbb{N}, +, 0)$ and $(\mathbb{N}, \mathit{max}, 0)$ form monoids
and $(\mathbb{N}, \mathit{min})$ forms a band. Note that data-flow is our first example where 
$\cseq$ is not idempotent. The distributivity laws require the following to be the case:
$\mathit{max}(r,s) + t = \mathit{max}(r+t, s+t)$, which is easy to see. 

A simple dataflow langauge includes an additional construct $\kvd{prev}$ for accessing the 
previous value in a stream with an additional typing rule that look as follows:
%
\begin{equation*}
\quad~ e ~::=~ \ldots \sep \kvd{next}~e
\end{equation*}
\begin{equation*}
\tyrule{prev}
  { \coctx{\Gamma}{\cclrd{n}} \vdash e : \tau }
  { \coctx{\Gamma}{\cclrd{n+1}} \vdash \kvd{prev}~e : \tau }
\end{equation*}
%
As a further example that was not covered earlier, it is also possible to combine liveness analysis
and data-flow. In the above calculus, $0$ denotes that we require the current value, but no previous
values. However, for constants, we do not even need the current value.

\begin{example}[Optimized data-flow]
In optimized data-flow, context is annotated with natural numbers extended with the $\bot$ element,
that is $\mathbb{N}_{\bot} = \{\bot, 0, 1, 2, 3, \ldots \}$ such that $\forall n \in \mathbb{N}. \bot \leq n$.
The flat coeffect algebra is $(\mathbb{N}_{\bot}, +, \mathit{max}, \mathit{min}, 0, \bot, \leq)$
where $m + n$ is $\bot$ whenever $m=\bot$ or $n=\bot$ and \emph{min}, \emph{max} treat $\bot$ as the
least element.
\end{example}

\noindent
Note that $(\mathbb{N}_{\bot}, +, 0)$ is a monoid for the extended definition of $+$; for the
bottom element $0 + \bot = \bot$ and for natural numbers $0 + n = n$. The structure 
$(\mathbb{N}, \emph{max}, \bot)$ is also a monoid, because $\bot$ is the least element and so
$\mathit{max}(n, \bot) = n$. Finally,  $(\mathbb{N}, \emph{min})$ is a band (the extended
\emph{min} is still idempotent and associative) and the distributivity law also holds
for $\mathbb{N}_{\bot}$.


% ==================================================================================================
%
%       #
%      # #   #    # #####   ####  # #    # # ##### #   #
%     #   #  ##  ## #    # #    # # #    # #   #    # #
%    #     # # ## # #####  #      # #    # #   #     #
%    ####### #    # #    # #  ### # #    # #   #     #
%    #     # #    # #    # #    # # #    # #   #     #
%    #     # #    # #####   ####  #  ####  #   #     #
%
% ==================================================================================================

\section{Choosing unique typing}
\label{sec:flat-unique}

As discussed in Chapter~\ref{sec:applications}, the lambda abstraction rule for coeffect systems
differs from the rule for effect systems in that it does not delay all context requirements. 
In case of implicit parameters (Section~\ref{sec:applications-flat-impl}), the requirements
can be satisfied either by the call-site or by the declaration-site. In case of dataflow and
liveness, the rule discussed in Section~\ref{sec:flat-calculus} reintroduces similar ambiguity
because it allows multiple valid typing derivations. 

Furthermore, the semantics of context-aware languages in Chapter~\ref{sec:applications} and
also in Chapter~\ref{sec:transl} is defined over \emph{typing derivation} and so the same program
could have a different meaning, depending on the typing derivation chosen. In this section, we
specify how to choose \emph{unique} typing derivation in each of the coeffect systems we consider.

The most interesting case is that of implicit parameters. For example, consider the following 
program written using the coeffect calculus with implicit parameter extensions:
%
\begin{equation*}
\begin{array}{l}
\kvd{let}~f~= (\kvd{let}~\ident{?x}=1~\kvd{in}~(\lambda y.~\ident{?x}))~\kvd{in} \\[-0.2em]
\kvd{let}~\ident{?x}=2~\kvd{in}~f~0 \\[-0.2em]
\end{array}  
\end{equation*}

\noindent
There are two possible typings allowed by the typing rules discussed in Section~\ref{sec:flat-calculus-types}
that lead to two possible meanings of the program -- evaluating to $1$ and $2$, respectively:
%
\begin{itemize}
  \item $f : \ident{num} \xrightarrow{\cclrd{\emptyset}} \ident{num}$ -- in this case, the value
    of \ident{?x} is captured from the declaration-site and the program produces $1$.
  \item $f : \ident{num} \xrightarrow{\cclrd{\{ \ident{?x} \}}} \ident{num}$ -- in this case, the
    parameter \ident{?x} is required from the call-site and the program produces $2$.
\end{itemize}
%
The coeffect calculus intentionally allows both of the options, acknowledging the fact that the
choice needs to be made for each individual concrete context-aware programming language. In this
section, we discuss the choices for implicit parameters, dataflow and liveness.

In this section, we use the fact that the coeffect calculus uses Church-style syntax for lambda
abstraction and has a type annotation for the type of the variable. This does not affect the 
handling of coeffects (those are not defined by the type annotation), but it lets us prove 
uniqueness typing property of the specialized coeffect type systems. This shows that we define a 
\emph{unique} way of assigning coeffects to otherwise well-typed programs.

%---------------------------------------------------------------------------------------------------

\begin{figure}[t]
\begin{equation*}
\tyrule{var}
  {x : \tau \in \Gamma}
  {\coctx{\Gamma;\cclrd{\Delta}}{\cunit} \vdash x : \tau }
\end{equation*}
\begin{equation*}
\tyrule{const}
  {~}
  {\coctx{\Gamma;\cclrd{\Delta}}{\czero} \vdash n : \ident{num} }
\end{equation*}
\begin{equation*}
\tyrule{app}
  {\coctx{\Gamma;\cclrd{\Delta}}{\cclrd{r}} \vdash e_1 : \tau_1 \xrightarrow{\cclrd{t}} \tau_2 &
   \coctx{\Gamma;\cclrd{\Delta}}{\cclrd{s}} \vdash e_2 : \tau_1 }
  {\coctx{\Gamma;\cclrd{\Delta}}{\cclrd{r} \;\cpar\; (\cclrd{s} \,\cseq\, \cclrd{t})} \vdash e_1~e_2 : \tau_2}
\end{equation*}
\begin{equation*}
\tyrule{let}
  { \coctx{\Gamma;\cclrd{\Delta}}{\cclrd{r}} \vdash e_1 : \tau_1 &
    \coctx{\Gamma, x\!:\!\tau_1;\cclrd{\Delta}}{\cclrd{s}} \vdash e_2 : \tau_2}
  {\coctx{\Gamma;\cclrd{\Delta}}{\cclrd{s} \;\cpar\; (\cclrd{s} \,\cseq\, \cclrd{r})} \vdash \kvd{let}~x=e_1~\kvd{in}~e_2 : \tau_2 }
\end{equation*}
\begin{equation*}  
\tyrule{param}
  { ~ }
  { \coctx{\Gamma;\cclrd{\Delta}}{\cclrd{ \{ \ident{?p} \} }} \vdash \ident{?p} : \ident{num} }
\end{equation*}
\begin{equation*}
\tyrule{abs}
  {\coctx{\Gamma, x\!:\!\tau_1;\cclrd{\Delta}}{\cclrd{r}} \vdash e : \tau_2}
  {\coctx{\Gamma;\cclrd{\Delta}}{\cclrd{\Delta}} \vdash \lambda x\!:\!\tau_1.e : \tau_1 \xrightarrow{\cclrd{r}\setminus\cclrd{\Delta}} \tau_2 }
\end{equation*}
\begin{equation*}
\tyrule{letpar}
  { \coctx{\Gamma;\cclrd{\Delta}}{\cclrd{r}} \vdash e_1 : \ident{num} &
    \coctx{\Gamma;\cclrd{\Delta \cup \{ \ident{?p} \}}}{\cclrd{s}} \vdash e_2 : \tau}
  {\coctx{\Gamma;\cclrd{\Delta}}{\cclrd{r} \;\cclrd{\cup\; (\cclrd{s} \setminus \{ \ident{?p}) \}} } \vdash \kvd{let}~\ident{?p}=e_1~\kvd{in}~e_2 : \tau }
\end{equation*}

\figcaption{Choosing unique typing for implicit parameters}
\label{fig:flat-resolve-impl}
\end{figure}

%---------------------------------------------------------------------------------------------------

\subsection{Implicit parameters}

For implicit parameters we follow the behaviour implemented by Haskell \cite{app-implicit-parameters}
where function abstraction captures all parameters that are available at the declaration-site and 
places all other requirements on the call-site. For the example in the introduction, this means
that the body of $f$ captures the value of \ident{?p} available from the declaration-site
and $f$ will be typed as a function requiring no parameters (coeffect $\cclrd{\emptyset}$). The 
program thus evaluates to $1$.

To express this behaviour formally, we extend the coeffect type system to additionally track 
implicit parameters that are currently in scope. The typing judgement becomes:
%
\begin{equation*}
\coctx{\Gamma;\cclrd{\Delta}}{\cclrd{r}} \vdash e : \tau
\end{equation*}
%
Here, $\cclrd{\Delta}$ is a set of implicit parameters that are in scope at the declaration-site.
The modified typing rules are shown in Figure~\ref{fig:flat-resolve-impl}. The rules (\emph{var}),
(\emph{const}), (\emph{app}) and (\emph{let}) are modified to use the new typing judgement, but they
simply propagate the information tracked by $\cclrd{\Delta}$ to all assumptions. The (\emph{param})
rule also remains unchanged -- the implicit parameter access is still tracked by the coeffect 
$\cclrd{r}$ meaning that we still allow a form of dynamic binding (the parameter does not have
to be in static scope).

The most interesting rule is (\emph{abs}). The body of a function requires implicit parameters
tracked by $\cclrd{r}$ and the parameters currently in (static) scope are $\cclrd{\Delta}$.
The coeffect on the declaration site becomes $\cclrd{\Delta}$ (capture all available parameters)
and the latent coeffect attached to the function becomes $\cclrd{r\setminus\Delta}$ (require all
remaining parameters from the call-site). Finally, in the (\emph{letpar}) rule, we add the newly
bound implicit parameter \ident{?p} to the static scope in the sub-expression $e_2$.

\paragraph{Properties.}
If a program written in a coeffect language with implicit parameters is well-typed according to the 
type system presented in Figure~\ref{fig:flat-resolve-impl}, then the type system gives a unique
derivation. We use this unique typing derivation to give the semantics of coeffect language
with implicit parameters in Chapter~\ref{ch:transl} and we also implement this algorithm as
discussed in Chatper~\ref{ch:impl}. 

The type system is more restrictive than the fully general one and it reject certain programs that
could be typed using the more general system. This is expected -- we are restricting the fully
general coeffect calculus to match the typing and semantics of implicit parameters as known from 
Haskell.

In order to prove the uniqueness of typing theorem (Theorem~\ref{thm:flat-impl-unique}), we first
need the inversion lemma (Lemma~\ref{thm:flat-impl-invert}).

\begin{lemma}[Inversion lemma for implicit parameters]
\label{thm:flat-impl-invert} 
For the type system defined in Figure~\ref{fig:flat-resolve-impl}:
%
\begin{enumerate}
\raggedright
\item If $\coctx{\Gamma;\cclrd{\Delta}}{\cclrd{c}} \vdash x : \tau$ then $x : \tau \in \Gamma$ and $\cclrd{c} = \cclrd{\emptyset}$.
\item If $\coctx{\Gamma;\cclrd{\Delta}}{\cclrd{c}} \vdash n : \tau$ then $\tau = \ident{num}$ and $\cclrd{c} = \cclrd{\emptyset}$.
\item If $\coctx{\Gamma;\cclrd{\Delta}}{\cclrd{c}} \vdash e_1~e_2 : \tau_2$
 then there is some $\tau_1,\cclrd{r},\cclrd{s}$ and $\cclrd{t}$ such that
 $\coctx{\Gamma;\cclrd{\Delta}}{\cclrd{r}} \vdash e_1 : \tau_1 \xrightarrow{\cclrd{t}} \tau_2$ and 
 $\coctx{\Gamma;\cclrd{\Delta}}{\cclrd{s}} \vdash e_2 : \tau_1$ and also
 $\cclrd{c} = \cclrd{r \cup s \cup t}$.
\item If $\coctx{\Gamma;\cclrd{\Delta}}{\cclrd{c}} \vdash \kvd{let}~x=e_1~\kvd{in}~e_2 : \tau_2$
 then there is some $\tau_1, \cclrd{s}$ and $\cclrd{r}$ such that
 $\coctx{\Gamma;\cclrd{\Delta}}{\cclrd{r}} \vdash e_1 : \tau_1$ and
 $\coctx{\Gamma, x\!:\!\tau_1;\cclrd{\Delta}}{\cclrd{s}} \vdash e_2 : \tau_2}$ and also
 $\cclrd{c} = \cclrd{s \cup r}$.
\item If $\coctx{\Gamma;\cclrd{\Delta}}{\cclrd{c}} \vdash \ident{?p} : \ident{num}$ then 
 $\ident{?p} \in \cclrd{c}$ and $\cclrd{c} = \cclrd{ \{\ident{?p}\} }$.
\item If $\coctx{\Gamma;\cclrd{\Delta}}{\cclrd{c}} \vdash \lambda x\!:\!\tau_1.e : \tau$ then 
  there is some $\tau_2$ such that $\tau = \tau_1 \xrightarrow{\cclrd{s}} \tau_2$ 
  ${\coctx{\Gamma, x\!:\!\tau_1;\cclrd{\Delta}}{\cclrd{r}} \vdash e : \tau_2}$ and
  $\cclrd{c} = \cclrd{\Delta}$ and also
  $\cclrd{s} = \cclrd{r\setminus\Delta}$.
\item If $\coctx{\Gamma;\cclrd{\Delta}}{\cclrd{c}} \vdash \kvd{let}~\ident{?p}=e_1~\kvd{in}~e_2 : \tau }$ then 
  there is some $\cclrd{r}, \cclrd{s}$ such that
  $\coctx{\Gamma;\cclrd{\Delta}}{\cclrd{r}} \vdash e_1 : \ident{num}$ and
  $\coctx{\Gamma;\cclrd{\Delta \cup \{ \ident{?p} \}}}{\cclrd{s}} \vdash e_2 : \tau$ and also
  $\cclrd{c}=\cclrd{r} \;\cclrd{\cup\; (\cclrd{s} \setminus \{ \ident{?p}) \}}$.

\end{enumerate}  
\end{lemma}
\begin{proof}
Follows from the individual rules given in Figure~\ref{fig:flat-resolve-impl}.
\end{proof}

\begin{theorem}[Uniqueness of coeffect typing for implicit parameters]
\label{thm:flat-impl-unique}
In the type system for implicit parameters defined in Figure~\ref{fig:flat-resolve-impl}, when
$\coctx{\Gamma;\cclrd{\Delta}}{\cclrd{r}} \vdash e : \tau$ and 
$\coctx{\Gamma;\cclrd{\Delta}}{\cclrd{r'}} \vdash e : \tau'$ then $\tau = \tau'$ and $\cclrd{r} = \cclrd{r'}$.
\end{theorem}
\begin{proof}
Suppose that (A) $\coctx{\Gamma;\cclrd{\Delta}}{\cclrd{c}} \vdash e : \tau$ and 
(B) $\coctx{\Gamma;\cclrd{\Delta}}{\cclrd{c'}} \vdash e : \tau'$. We show by induction over the typing 
derivation of $\coctx{\Gamma;\cclrd{\Delta}}{\cclrd{c}} \vdash e : \tau$ that $\tau = \tau'$ and $\cclrd{c}=\cclrd{c'}$. 

\vspace{0.5em}\noindent\hangindent=0.6cm 
Case (\emph{abs}): $e = \lambda x\!:\!\tau_1.e_1$ and $\cclrd{c}=\cclrd{\Delta}$.
  $\tau = \tau_1 \xrightarrow{\cclrd{r}\setminus\cclrd{\Delta}} \tau_2$ for some $\cclrd{r}, \tau_2$ and also
  $\coctx{\Gamma, x\!:\!\tau_1;\cclrd{\Delta}}{\cclrd{r}} \vdash e : \tau_2}$. 
  By case (6) of Lemma~\ref{thm:flat-impl-invert}, the final rule of the derivation (B) must
  have also been (\emph{abs}) and this derivation has a sub-derivation with a conclusion
  ${\coctx{\Gamma, x\!:\!\tau_1;\cclrd{\Delta}}{\cclrd{r}} \vdash e : \tau_2'}$.
  By the induction hypothesis $\tau_2 = \tau_2'$ and $\cclrd{c}=\cclrd{c'}$ and 
  therefore $\tau = \tau'$.

\vspace{0.5em}\noindent\hangindent=0.7cm 
Case (\emph{param}): $e = \ident{?p}$, from Lemma~\ref{thm:flat-impl-invert}, $\tau = \tau' = \ident{int}$
  and $\cclrd{c} = \cclrd{c'} = \cclrd{\{ \ident{?p} \}}$.

\vspace{0.5em}\noindent\hangindent=0.7cm 
Cases (\emph{var}), (\emph{const}) are direct consequence of Lemma~\ref{thm:flat-impl-invert}.
  
\vspace{0.5em}\noindent\hangindent=0.7cm 
Cases (\emph{var}), (\emph{const}), (\emph{app}), (\emph{let}), (\emph{param}) and (\emph{letpar})
  similarly to (\emph{abs}).
\end{proof}

\paragraph{Implementation.}
From the presentation in this section, it might appear that resolving the ambiguity related to
lambda abstraction for implicit parameters requires a type system that is quite different from
the core flat coeffect type system shown earlier in Figure~\ref{fig:flat-types}. This is not the
case. As discussed in Chapter~\ref{ch:impl}, the required changes in the implementation are 
simpler.

Briefly, the implementation collects constraints on the coeffects and then finds the smallest sets 
of implicit parameters to satisfy the constraints. We still need to track implicit parameters in 
scope $\cclrd{\Delta}$, but the rest of the (\emph{abs}) rule from the implementation is close
to the one from Figure~\ref{fig:flat-types}:
%
\begin{equation*}
\tyrule{abs}
  {\coctx{\Gamma, x\!:\!\tau_1;\cclrd{\Delta}}{\cclrd{t}} \vdash e : \tau_2 ~|~ C}
  {\coctx{\Gamma;\cclrd{\Delta}}{\cclrd{r}} \vdash \lambda x\!:\!\tau_1.e : \tau_1 \xrightarrow{\cclrd{s}} \tau_2 ~|~
    C\cup\{\cclrd{t}=\cclrd{r} \,\czip\, \cclrd{s}, \cclrd{r}=\cclrd{\Delta} \}} 
\end{equation*}
%
Given a typing derivation for the body that produced constraints $C$, we generate an additional
constraint that restricts $\cclrd{r}$ (declaration-site requirements) to those available in the
current static scope $\cclrd{\Delta}$. The constraint satisfaction algorithm then finds 
the minimal set $\cclrd{s}$ which is $\cclrd{t}\cclrd{\setminus \Delta}$.

%---------------------------------------------------------------------------------------------------

\subsection{Dataflow and liveness}
\label{sec:flat-unique-ldf}

Resolving the ambiguity for liveness and dataflow computations is easier than for implicit 
parameters. It suffices to use a lambda abstraction rule that duplicates the coeffects of the 
body:
%
\begin{equation*}
\tyrule{idabs}
  {\coctx{\Gamma, x\!:\!\tau_1}{\cclrd{r}} \vdash e : \tau_2}
  {\coctx{\Gamma}{\cclrd{r}} \vdash \lambda x.e : \tau_1 \xrightarrow{\cclrd{r}} \tau_2 }
\end{equation*}
%
This is the rule that we originally used for liveness and dataflow computations in 
Chapter~\ref{ch:applications}. This rule cannot be used with implicit parameters and so 
the additional flexibility provided by the $\czip$ operator is needed in the general flat
coeffect calculus.

For livenes and dataflow, the (\emph{idabs}) rule provides the most precise coeffect.
Assume we have lambda abstraction with body that has coeffects $\cclrd{r}$. The 
ordinary (\emph{abs}) rule requires us to find $\cclrd{s}, \cclrd{t}$ such that
$\cclrd{r}=\cclrd{s}\,\czip\,\cclrd{t}$.

\begin{itemize}
  \item[--] For dataflow, this is $\cclrd{r}=\mathit{min}(\cclrd{s}, \cclrd{t})$. The smallest
    $\cclrd{s}, \cclrd{t}$ such that the equality holds are $\cclrd{s}=\cclrd{t}=\cclrd{r}$.
 \item[--] For livenes, this is $\cclrd{r}=\cclrd{s} \sqcup \cclrd{t}$. When $\cclrd{r}=\ident{L}$,
   the only solution is $\cclrd{s}=\cclrd{t}=\ident{L}$; when $\cclrd{r}=\ident{D}$, the most
   precise solution is $\cclrd{s}=\cclrd{t}=\ident{D}$ because $\ident{D}\sqsubseteq\ident{L}$.    
\end{itemize}

The notion of ``more precise'' solution can be defined in terms of sub-coeffecting and subtyping.
We return to this topic in Section~\ref{sec:flat-exts-lambda} and we also precisely characterise
for which coeffect system is the (\emph{idabs}) rule preferable over the (\emph{abs}) rule.

\paragraph{Properties.}
If a program written in a coeffect language for liveness or dataflow is well-typed according to
the type system presented in Figure~\ref{fig:flat-types} with the (\emph{abs}) rule replaced by
(\emph{idabs}), then the type system gives a unique derivation. As for implicit parameters, this
defines the semantics of coeffect program (Chapter~\ref{ch:transl}) and it is used in the
implementation (Chatper~\ref{ch:impl}). 

In order to prove the uniqueness of typing theorem (Theorem~\ref{thm:flat-ldf-unique}), we first
need the inversion lemma (Lemma~\ref{thm:flat-ldf-invert}).

\begin{lemma}[Inversion lemma for liveness and dataflow]
\label{thm:flat-ldf-invert} 
For the type system defined in Figure~\ref{fig:flat-types} with the (abs) rule replaced
by (idabs):
%
\begin{enumerate}
\raggedright
\item If $\coctx{\Gamma}{\cclrd{c}} \vdash x : \tau$ then $x : \tau \in \Gamma$ and $\cclrd{c} = \cunit$.
\item If $\coctx{\Gamma}{\cclrd{c}} \vdash n : \tau$ then $\tau = \ident{num}$ and $\cclrd{c} = \czero$.
\item If $\coctx{\Gamma}{\cclrd{c}} \vdash e_1~e_2 : \tau_2$
 then there is some $\tau_1,\cclrd{r},\cclrd{s}$ and $\cclrd{t}$ such that
 $\coctx{\Gamma}{\cclrd{r}} \vdash e_1 : \tau_1 \xrightarrow{\cclrd{t}} \tau_2$ and 
 $\coctx{\Gamma}{\cclrd{s}} \vdash e_2 : \tau_1$ and also
 $\cclrd{c} = \cclrd{r} \;\cpar\; (\cclrd{s} \,\cseq\, \cclrd{t})$.
\item If $\coctx{\Gamma}{\cclrd{c}} \vdash \kvd{let}~x=e_1~\kvd{in}~e_2 : \tau_2$
 then there is some $\tau_1, \cclrd{s}$ and $\cclrd{r}$ such that
 $\coctx{\Gamma}{\cclrd{r}} \vdash e_1 : \tau_1$ and
 $\coctx{\Gamma, x\!:\!\tau_1}{\cclrd{s}} \vdash e_2 : \tau_2}$ and also
 $\cclrd{c} = \cclrd{s} \;\cpar\; (\cclrd{s} \,\cseq\, \cclrd{r})$.
\item If $\coctx{\Gamma}{\cclrd{c}} \vdash \lambda x\!:\!\tau_1.e : \tau$ then there is some  
  $\tau_2$ such that $\tau = \tau_1 \xrightarrow{\cclrd{c}} \tau_2$ and
  ${\coctx{\Gamma, x\!:\!\tau_1}{\cclrd{c}} \vdash e : \tau_2}$.
\end{enumerate}  
\end{lemma}
\begin{proof}
Follows from the individual rules given in Figure~\ref{fig:flat-resolve-impl}.
\end{proof}

\begin{theorem}[Uniqueness of coeffect typing for liveness and dataflow]
\label{thm:flat-ldf-unique}
In the type system for implicit parameters defined in Figure~\ref{fig:flat-types} with the (abs)
rule replaced by (idabs), when $\coctx{\Gamma}{\cclrd{r}} \vdash e : \tau$ and 
$\coctx{\Gamma}{\cclrd{r'}} \vdash e : \tau'$ then $\tau = \tau'$ and $\cclrd{r} = \cclrd{r'}$.
\end{theorem}
\begin{proof}
Suppose that (A) $\coctx{\Gamma}{\cclrd{c}} \vdash e : \tau$ and 
(B) $\coctx{\Gamma}{\cclrd{c'}} \vdash e : \tau'$. We show by induction over the typing 
derivation of $\coctx{\Gamma}{\cclrd{c}} \vdash e : \tau$ that $\tau = \tau'$ and $\cclrd{c}=\cclrd{c'}$. 

\vspace{0.5em}\noindent\hangindent=0.6cm 
Case (\emph{abs}): $e = \lambda x\!:\!\tau_1.e_1$. $\tau = \tau_1 \xrightarrow{\cclrd{c}} \tau_2$ 
  for some $\tau_2$ and $\coctx{\Gamma, x\!:\!\tau_1}{\cclrd{c}} \vdash e : \tau_2}$. 
  By case (5) of Lemma~\ref{thm:flat-ldf-invert}, the final rule of the derivation (B) must
  have also been (\emph{abs}) and this derivation has a sub-derivation with a conclusion
  ${\coctx{\Gamma, x\!:\!\tau_1}{\cclrd{c'}} \vdash e : \tau_2'}$.
  By the induction hypothesis $\tau_2 = \tau_2'$ and $\cclrd{c}=\cclrd{c'}$ and therefore
  also so $\tau = \tau'$.

\vspace{0.5em}\noindent\hangindent=0.7cm 
Cases (\emph{var}), (\emph{const}) are direct consequence of Lemma~\ref{thm:flat-impl-invert}.
  
\vspace{0.5em}\noindent\hangindent=0.7cm 
Cases (\emph{var}), (\emph{const}), (\emph{app}), (\emph{let}), (\emph{param}) and (\emph{letpar})
  similarly to (\emph{abs}).
\end{proof}

\paragraph{Implementation.}
As with implicit parameters, the implementation (discussed in Chapter~\ref{ch:impl}) does not
require changing the typing (\emph{abs}) rule of the flat coeffect system. The typing specified
by (\emph{idabs}) can be easily obtained by generating additional constraints:
%
\begin{equation*}
\tyrule{abs}
  {\coctx{\Gamma, x\!:\!\tau_1}{\cclrd{t}} \vdash e : \tau_2 ~|~ C}
  {\coctx{\Gamma}{\cclrd{s}} \vdash \lambda x\!:\!\tau_1.e : \tau_1 \xrightarrow{\cclrd{t}} \tau_2 ~|~
    C\cup\{\cclrd{t}=\cclrd{r} \,\czip\, \cclrd{s}, \cclrd{r}=\cclrd{t}, \cclrd{s}=\cclrd{t} \}} 
\end{equation*}
%
Here, the two additional constraints restrict both $\cclrd{r}$ and $\cclrd{s}$ to be equal to the
coeffect of the body $\cclrd{t}$ and so the only possible resolution is the one specified by 
(\emph{idabs}).


% ==================================================================================================
%
%    #####
%    #     # #   # #    # #####   ##    ####  ##### #  ####
%    #        # #  ##   #   #    #  #  #    #   #   # #    #
%     #####    #   # #  #   #   #    # #        #   # #
%          #   #   #  # #   #   ###### #        #   # #
%    #     #   #   #   ##   #   #    # #    #   #   # #    #
%     #####    #   #    #   #   #    #  ####    #   #  ####
%
% ==================================================================================================

\section{Syntactic equational theory}
\label{sec:flat-syntax}

Each of the concrete coeffect calculi discussed in this chapter has a different notion of context,
much like various effectful languages have different notions of effects (such as exceptions or
mutable state). However, in all of the calculi, the context has a number of common properties that 
are captured by the \emph{flat coeffect algebra}. This means that there are equational properties
that hold for all of the coeffect systems. Further properties hold for systems where the context
satisfies additional properties. 

In this section, we look at such shared syntactic properties. This accompanies the previous section, 
which provided a \emph{semantic} justification for the axioms of coeffect algebra with a 
\emph{syntactic} justification. Operationally, this section can also be viewed as providing a 
pathway to an operational semantics for two of our systems (implicit parameters and liveness),
which can be based on syntactic substitution. As we discuss later, the notion of context for
data-flow is more complex.

% --------------------------------------------------------------------------------------------------

\subsection{Syntactic properties}
\label{sec:flat-syntax-props}

Before discussing the syntactic properties of general coeffect calculus formally, it should be 
clarified what is meant by providing ``pathway to operational semantics'' in this section. We do
that by contrasting syntactic properties of coeffect systems with more familiar effect systems.
Assuming $\subst{e_1}{x}{e_2}$ is a standard capture-avoiding syntactic substitution, the following
equations define four syntactic reductions on the terms:
%
\begin{equation*}
\begin{array}{rllcl}
(\lambda x.e_1)~e_2 &{\reduce_{\ident{cbn}}}& \subst{e_1}{x}{e_2}   &&(\textit{call-by-name})\\
(\lambda x.e_1)~v   &{\reduce_{\ident{cbv}}}& \subst{e_1}{x}{v}     &\quad&(\textit{call-by-value})\\
e &{\reduce_\eta}& \lambda x.e~x                                    &&(\textit{$\eta$-expansion})
\end{array}
\end{equation*}
%
The rules capture syntactic reductions that can be performed in a general calculus, without any 
knowledge of the specific notion of context. If the reductions preserve the type of the expression
(type preservation), then operational semantics can be defined as a repeated application of the 
rules, until a specified normal form (\ie~a value) is reached.

In the rest of the section, we briefly outline the interpretation of the three
rules and then we focus on call-by-value (Section~\ref{sec:flat-syntax-cbv}) and call-by-name 
(Section~\ref{sec:flat-syntax-cbn}) in more details.

The focus of this chapter is on the general coeffect system and so we do not discuss the operational
semantics of the specific notions of context. However, some work in that area has been done
by Brunel et al. \cite{coeffects-quantitative}. We discuss concrete semantics of implicit 
parameters and dataflow in Chapter~\ref{ch:transl}.

\paragraph{Call-by-name.} 
In call-by-name, the argument is syntactically substituted for all occurrences of a variable. It
can be used as the basis for operational semantics of purely functional languages. However, using
the rule in effectful languages breaks the \emph{type preservation} property. For example, consider
a language with effect system where functions are annotated with sets of effects such as $\{ \ident{write} \}$.
A function $\lambda x.y$ is effect-free:
%
\begin{equation*}
y\!:\!\tau_1 \vdash \lambda x.y : \tau_1 \xrightarrow{\emptyset} \tau_2 \;\&\; \emptyset
\end{equation*}
%
Substituting an expression $e$ with effects $\{ \ident{write} \}$ for $y$ changes the type of 
the function by adding latent effects (without changing the immediate effects):
%
\begin{equation*}
\vdash \lambda x.e : \tau_1 \xrightarrow{ \{ \ident{write} \} } \tau_2 \;\&\; \emptyset
\end{equation*}
%
Similarly to effect systems, substituting a context-dependent computation $e$ for a variable $y$ can
add latent coeffects to the function type. However, this is not the case for \emph{all} flat coeffect
calculi. For example, call-by-name reduction preserves types and coeffects for the implicit 
parameters system. This means that certain coeffect systems support call-by-name evaluation strategy
and could be embedded in purely functional language (such as Haskell).

\paragraph{Call-by-value.}

The call-by-value evaluation strategy is often used by effectful languages. Here, an argument is
first reduced to a \emph{value} before performing the substitution. In effectful languages, 
value is defined syntactically. For example, in the \emph{Effect} language \cite{monads-effects-marriage},
values are identifiers $x$ or functions $(\lambda x.e)$.

The notion of \emph{value} in coeffect systems differs from the usual syntactic understanding.
A function $(\lambda x.e)$ does not defer all context requirements of the body $e$ and may have
immediate context requirements. Thus we say that $e$ is a value if it is a value in the usual
sense \emph{and} has not immediate context requirements. We define this formally in 
Section~\ref{sec:flat-syntax-cbv}.

The call-by-value evaluation strategy preserves typing for a wide range of flat coeffect calculi, 
including all our three examples. However, it is rather weak -- in order to use it, the concrete 
semantics needs to provide a way for reducing context-dependent term $\coctx{\Gamma}{\cclrd{r}} \vdash e : \tau$ 
to a value, \ie~a term $\coctx{\Gamma}{\cunit} \vdash e' : \tau$ with no context requirements. 

\paragraph{Local soundness and completeness.}
Two desirable properties of calculi, coined by Pfenning and Davies \cite{logic-modal-reconstruction},
are \emph{local soundness} and \emph{local completeness}. They guarantee that the rules which 
introduce a function arrow (lambda abstraction) and eliminate it (application) are sufficiently 
strong, but not too strong.

The local soundness property is witnessed by (call-by-name) $\beta$-reduction, which we discussed
already. The local completeness is witnessed by the $\eta$-expansion rule. We discuss the flat
coeffect algebra conditions under which the reduction holds in Section~\ref{sec:flat-syntax-cbn}.

% --------------------------------------------------------------------------------------------------

\subsection{Call-by-value evaluation}
\label{sec:flat-syntax-cbv}

As discussed in the previous section, call-by-value reduction can be used for most flat coeffect
calculi, but it provides a very weak general model \ie~the hard work of reducing context-dependent
term to a \emph{value} has to be provided for each system. Syntactic values are defined
in the usual way:
%
\begin{equation*}
\begin{array}{llll}
v\in\mathit{SynVal}   & v &::=& x \sep c \sep (\lambda x.e)\\
n\in\mathit{NonVal}   & n &::=& e_1\;e_2 \sep \kvd{let}~x=e_1~\kvd{in}~e_2\\
e\in\mathit{Expr}     & e &::=& v \sep n 
\end{array}
\end{equation*}
%
The syntactic form \emph{SynVal} captures syntactic values, but a context-dependency-free value in 
coeffect calculus cannot be defined purely syntactically, because a function $(\lambda x.e)$ does
not automatically defer all context requirements.

\begin{definition} 
An expression $e$ is a \emph{value}, written as $\textit{val}(e)$ if it is a syntactic value,
\ie~$e \in \mathit{SynVal}$ and it has no context-dependencies, \ie~$\coctx{\Gamma}{ \cunit } \vdash e : \tau$.
\end{definition}

\noindent
The call-by-value substitution substitutes a value, with context requirements $\cunit$, for a 
variable, whose access is also annotated with $\cunit$. Thus, it does not affect the type and
context requirements of the term:

\begin{lemma}[Call-by-value substitution]
\label{thm:flat-subst-cbv}
In a flat coeffect calculus with a coeffect algebra $(\C, \cseq, \cpar, \czip, \cunit, \czero, \cleq)$,
given a value $\coctx{\Gamma}{\cunit} \vdash v : \sigma$ and an expression
$\coctx{\Gamma,  x\!:\!\sigma}{ \cclrd{r}  } \vdash e : \tau$, then substituting $v$ for $x$ does not
change the type and context requirements, that is $\coctx{\Gamma}{ \cclrd{r} } \vdash \subst{e}{x}{v} : \tau$.
\end{lemma}
\begin{proof}
By induction over the type derivation, using the fact that $x$ and $v$ are annotated
with $\cunit$ and that $\Gamma$ is treated as a set in the flat calculus.
\end{proof}

\noindent
The substitution lemma~\ref{thm:flat-subst-cbv} holds for all flat coeffect systems. However, 
proving that call-by-value reduction preserves typing requires an additional constraint on the
flat coeffect algebra, which relates the $\czip$ and $\cpar$ operations. This is captured
by the (\emph{approximation}) property:
%
\begin{equation*}
\cclrd{r}\,\czip\,\cclrd{t}~\;\cleq~\;\cclrd{r}\,\cpar\,\cclrd{t}
\hspace{10em}(\textit{approximation})
\end{equation*}
%
Intuitively, this specifies that the $\czip$ operation (splitting of context requirements) 
under-approximates the actual context capabilities while the $\cpar$ operation (combining of
context requirements) over-approximates the actual context requirements.

The property holds for the three systems we consider -- for implicit parameters, this is 
an equality; for liveness and data-flow (which both use a lattice), the greatest lower bound 
is smaller than the least upper bound.

Assuming $\reduce_{\ident{cbv}}$ is call-by-value reduction that reduces the term 
$(\lambda x.e)\,v$ to a term $\subst{e}{x}{v}$, the type preservation theorem is
stated as follows:

\begin{theorem}[Type preservation for call-by-value]
\label{thm:cbv-reduction}
In a flat coeffect system with the (\emph{approxi\-mation}) property, 
if $\coctx{\Gamma}{\cclrd{r}} \vdash e : \tau$ 
and $e \reduce_{\ident{cbv}} e'$ then $\coctx{\Gamma}{\cclrd{r}} \vdash e' : \tau$.
\end{theorem}
\begin{proof}
Consider the typing derivation for the term $(\lambda x.e)\,v$ before reduction:
\begin{equation*}
\inference
  { \inference
  { \inference
      {\coctx{\Gamma, x\!:\!\tau_1}{\cclrd{r}\;\czip\;\cclrd{t}} \vdash e : \tau_2}
      {\coctx{\Gamma}{\cclrd{r}} \vdash \lambda x.e : \tau_1 \xrightarrow{\cclrd{t}} \tau_2 } &
   \coctx{\Gamma}{\cunit} \vdash v : \tau_1 }
  {\coctx{\Gamma}{\cclrd{r} \;\cpar\; (\cunit \,\cseq\, \cclrd{t})} \vdash (\lambda x.e)\,v : \tau_2} }
  {\coctx{\Gamma}{\cclrd{r} \;\cpar\; \cclrd{t}} \vdash (\lambda x.e)\,v : \tau_2}
\end{equation*}
The final step simplifies the coeffect annotation using the fact that $\cunit$ is a unit of $\cseq$.
From Lemma~\ref{thm:flat-subst-cbv}, $\subst{e}{x}{v}$ has the same coeffect annotation as $e$.
As $\cclrd{r}\,\czip\,\cclrd{t}\;\cleq\;\cclrd{r}\,\cpar\,\cclrd{t}$, we can apply sub-coeffecting:
%
\begin{equation*}
\tyrule{sub}
 { \coctx{\Gamma}{\cclrd{r}\;\czip\;\cclrd{t}} \vdash \subst{e}{x}{v} : \tau_2 }
 { \coctx{\Gamma}{\cclrd{r}\;\cpar\;\cclrd{t}} \vdash \subst{e}{x}{v} : \tau_2 } 
\end{equation*}
%
Comparing the final conclusions of the above two typing derivations shows that 
the reduction preserves type and coeffect annotation.
\end{proof}

% --------------------------------------------------------------------------------------------------

\subsection{Call-by-name evaluation}
\label{sec:flat-syntax-cbn}

When reducing the expression $(\lambda x.e_1)~e_2$ using the call-by-name strategy, the 
sub-expression $e_2$ is substituted for all occurrences of the variable $v$ in an expression $e_1$. 
As discussed in Section~\ref{sec:flat-syntax-props}, the call-by-name strategy does not \emph{in 
general} preserve the type of a terms in coeffect calculi, but it does preserve the typing in two 
interesting cases. 

The typing is preserved for different reasons in two of our systems, so we briefly review the 
concrete examples. Then, we prove the substitution lemma for two special cases of flat coeffects
(Lemma~\ref{thm:cbn-substitution-top} and Lemma~\ref{thm:cbn-substitution-bot}) and finally, we
state the conditions under which typing preservation hold for flat coeffect calculi 
(Theorem~\ref{thm:cbn-flat}).

\paragraph{Data-flow.} 
The type preservation property does not hold for data-flow. This case is similar to the example
shown earlier with effectful computations. As a minimal example, consider the substitution of 
a context-dependent expression $\kvd{prev}~z$ for a variable $y$ in a function $\lambda x.y$:
%
\begin{equation*}
\begin{array}{rclcl}
 \coctx{y\!:\!\tau_1,z\!:\!\tau_1}{0} &\narrow{\vdash}& \lambda x.y : \tau_1 \xrightarrow{0} \tau_2 & & (\textnormal{before}) \\
 \coctx{z\!:\!\tau_1}{1} &\narrow{\vdash}& \lambda x.\kvd{prev}~z : \tau_1 \xrightarrow{1} \tau_2 & & (\textnormal{after}) \\
\end{array}
\end{equation*}
%
After the substitution, the coeffect of the body is $1$. The rule for lambda abstraction requires
that $1=\mathit{min}(r,s)$ and so the least solution is to set both $r,s$ to $1$. The substitution
this affects the coeffects attached both to the function type and the overall context. 

Semantically, the coeffect over-approximates the actual requirements -- at run-time, the code does not 
actually access a previous value of the argument $x$. This cannot be captured by a flat coeffect 
system, but it can be captured using the structural system discussed in Chapter~\ref{ch:structural}. 

\paragraph{Implicit parameters.} In data-flow, there is no typing for the resulting expression that
preserves the type of the function. However, this is not the case for all systems. Consider substituting
an implicit parameter access $\ident{?p}$ for a free variable $y$ under a lambda:
%
\begin{equation*}
\begin{array}{rclcl}
 \coctx{y\!:\!\tau_1}{\emptyset} &\narrow{\vdash}& \lambda x.y : \tau_1 \xrightarrow{\emptyset} \tau_2 & & (\textnormal{before}) \\
 \coctx{\emptyset}{ \{\ident{?p}\} } &\narrow{\vdash}& \lambda x.\ident{?p} : \tau_1 \xrightarrow{\emptyset} \tau_2 & & (\textnormal{after}) \\
\end{array}
\end{equation*}
%
The above shows one possible typing of the body -- one that does not change the coeffects of the 
function type and attaches all additional coeffects (implicit parameters) to the context. In case
of implicit parameters (and, more generally, systems with set-like annotations) this is always 
possible. 

\paragraph{Liveness.} In liveness, the type preservation also holds, but for a different reason. Consider
substituting an arbitrary expression $e$ of type $\tau_1$ with coeffects $\cclrd{r}$ for a variable $y$:
%
\begin{equation*}
\begin{array}{rclcl}
 \coctx{y\!:\!\tau_1}{\ident{L}} &\narrow{\vdash}& \lambda x.y : \tau_1 \xrightarrow{\ident{L}} \tau_2 & & (\textnormal{before}) \\
 \coctx{\emptyset}{\ident{L}} &\narrow{\vdash}& \lambda x.e : \tau_1 \xrightarrow{\ident{L}} \tau_2 & & (\textnormal{after}) \\
\end{array}
\end{equation*}
%
In the original expression, both the overall context and the function type are annotated with \ident{L}, 
because the body contains a variable access. An expression $e$ can always be treated as being annotated
with $\ident{L}$ (because $\ident{L}$ is the top element of the lattice) and so we can also treat $e$
as being annotated with coeffects $\ident{L}$. As a result, substitution does not change the coeffect.

\paragraph{Reduction theorem.}
The above examples (implicit parameters and liveness) demonstrate two particular kinds of coeffect 
algebra for which typing preservation holds. Proving the type preservation separately provides 
more insight into how the systems work. We consider the two cases separately, but find a more general
formulation for both of them.

\begin{definition}
We call a flat coeffect algebra \emph{top-pointed} if $\cunit$ is the greatest (top) coeffect scalar
($\forall r \in \C \;.\; r \,\cleq\, \cunit $) and \emph{bottom-pointed} if it is the smallest (bottom) 
element ($\forall r \in \C \;.\; r \,\cgeq\, \cunit $).
\end{definition}

\noindent
Liveness is an example of top-pointed coeffects as variables are annotated with 
$\ident{L}$ and $\ident{D} \leq \ident{L}$, while implicit parameters and data-flow are examples
of bottom-pointed coeffects. For top-pointed flat coeffects, the substitution lemma holds without 
additional requirements:

\begin{lemma}[Top-pointed substitution]
\label{thm:cbn-substitution-top}
In a top-pointed flat coeffect calculus with an algebra $(\C, \cseq, \cpar, \czip, \cunit, \czero, \cleq)$,
substituting an expression $e_s$ with arbitrary coeffects $\cclrd{s}$ for a variable $x$ in $e_r$ does
not change the coeffects of $e_r$:
%
\begin{equation*}
\begin{array}{l}
 \coctx{\Gamma}{\cclrd{s}} \vdash e_s : \tau_s \;\; \wedge \;\; 
   \coctx{\Gamma_1,  x : \tau_s, \Gamma_2}{ \cclrd{r}  } \vdash e_r : \tau_r\\
 \quad \Rightarrow \;\; \coctx{\Gamma_1,\Gamma,\Gamma_2}{ \cclrd{r} } \vdash \subst{e_r}{x}{e_s} : \tau_r
\end{array}
\end{equation*}
\end{lemma}

\begin{proof}
Using sub-coeffecting ($\cclrd{s} \,\cleq\, \cunit$) and a variation of Lemma~\ref{thm:flat-subst-cbv}.
\end{proof}
%
\noindent
As variables are annotated with the top element $\cunit$, we can substitute the term $e_s$ 
for any variable and use sub-coeffecting to get the original typing (because
$\cclrd{s} \,\cleq\, \cunit$). 

In a bottom pointed coeffect system, substituting $e$ for $x$ increases the context 
requirements. However, if the system satisfies the strong condition that $\czip = \cseq = \cpar$ 
then the context requirements arising from the substitution can be associated with the context
$\Gamma$, leaving the context requirements of a function value unchanged. As a result, substitution 
does not break soundness as in effect systems. The requirement $\czip = \cseq = \cpar$ holds for 
our implicit parameters example (all three operators are a set union) and for other set-like 
coeffects. It allows the following substitution lemma:

\begin{lemma}[Bottom-pointed substitution]
\label{thm:cbn-substitution-bot}
In a bottom-pointed flat coeffect calculus with an algebra $(\C, \cseq, \cpar, \czip, \cunit, \czero, \cleq)$ 
where $\czip = \cseq = \cpar$ is an idempotent and commutative operation`  ' and
$\cclrd{r}\,\cleq\,\cclrd{r'} \Rightarrow \forall\cclrd{s}.\cclrd{r}\,\cseq\,\cclrd{s}\;\cleq\;\cclrd{r'}\,\cseq\,\cclrd{s}$ then:
%
\begin{equation*}
\begin{array}{l}
\coctx{\Gamma}{\cclrd{s}} \vdash e_s : \tau_s \;\; \wedge \;\; 
\coctx{\Gamma_1,  x : \tau_s, \Gamma_2}{ \cclrd{r}  } \vdash e_r : \tau_r\\
\quad \Rightarrow \;\; \coctx{\Gamma_1,\Gamma,\Gamma_2}{ \cclrd{r} \,\cseq\, \cclrd{s} } \vdash \subst{e_r}{x}{e_s} : \tau_r
\end{array}
\end{equation*}

\end{lemma}
\begin{proof}
By induction over $\vdash$, using the idempotent, commutative monoid structure to keep $\cclrd{s}$ with
the free-variable context. See Appendix~\ref{sec:appendix-flat-cbn}.
\end{proof}

\noindent
The flat system discussed here is \emph{flexible enough} to let us always re-associate new context 
requirements (arising from the substitution) with the free-vari\-able context. In contrast, the 
structural system discussed in Chapter~\ref{ch:structural} is \emph{precise enough} to keep the 
coeffects associated with individual variables, thus preserving typing in a complementary way.

The two substitution lemmas discussed above show that the call-by-name evaluation strategy can be 
used for certain coeffect calculi, including liveness and implicit parameters. Assuming
$\reduce_{\ident{cbn}}$ is the standard call-by-name reduction, the following theorem holds:

\begin{theorem}[Type preservation for call-by-name]
\label{thm:cbn-flat}
In a coeffect system that satisfies the conditions for Lemma~\ref{thm:cbn-substitution-top} or
Lemma~\ref{thm:cbn-substitution-bot}, if $\coctx{\Gamma}{\cclrd{r}} \vdash e : \tau$ 
and $e \rightarrow_{\ident{cbn}} e'$ then it is also the case that $\coctx{\Gamma}{\cclrd{r}} \vdash e' : \tau$.
\end{theorem}
\begin{proof}

For top-pointed coeffect algebra (using Lemma~\ref{thm:cbn-substitution-top}), the proof is similar
to the one in Theorem~\ref{thm:cbv-reduction}, using the facts that $\cclrd{s} \,\cleq\, \cunit$
and $r \,\czip\, t \;=\; r \,\cpar\, t$.
For bottom-pointed coeffect algebra, consider the typing derivation for the term
$(\lambda x.e_r)\,e_s$ before reduction:
%
\begin{equation*}
\inference
  { \inference
      {\coctx{\Gamma, x:\tau_s}{\cclrd{r}} \vdash e_r : \tau_r}
      {\coctx{\Gamma}{\cclrd{r}} \vdash \lambda x.e_r : \tau_s \xrightarrow{\cclrd{r}} \tau_r } &
   \coctx{\Gamma}{\cclrd{s}} \vdash e_s : \tau_s }
  {\coctx{\Gamma}{\cclrd{r} \;\cpar\; (\cclrd{s} \,\cseq\, \cclrd{r})} \vdash (\lambda x.e_r)\,e_s : \tau_r }
\end{equation*}
%
The derivation uses the idempotence of $\czip$ in the first step, followed by the 
(\emph{app}) rule. The type of the term after substitution, using Lemma~\ref{thm:cbn-substitution-bot} is:
%
\begin{equation*}
\inference
  { \coctx{\Gamma, x:\tau_s}{\cclrd{r}} \vdash e_r : \tau_r & 
    \coctx{\Gamma}{\cclrd{s}} \vdash e_s : \tau_s }
  { \coctx{\Gamma, x:\tau_r}{\cclrd{r}\;\cseq\;\cclrd{s}} \vdash \subst{e_r}{x}{e_s} : \tau_s }
\end{equation*}
%
From the assumptions of Lemma~\ref{thm:cbn-substitution-bot}, we know that $\cseq\,=\,\cpar$
and the operation is idempotent, so trivially:
$\cclrd{r}\;\cseq\;\cclrd{s} = \cclrd{r} \;\cpar\; (\cclrd{s} \,\cseq\, \cclrd{r})$
\end{proof}

\paragraph{Expansion theorem.}
The $\eta$-expansion (local completeness) is similar to $\beta$-reduction (local soundness) in that
it holds for some flat coeffect systems, but not for all. Out of the examples we discuss, it 
holds for implicit parameters, but does not hold for liveness and data-flow.

Recall that $\eta$-expansion turns $e$ into $\lambda x.e~x$. In the case of liveness, the 
expression $e$ may require no variables (both immediate and latent coeffects are marked as
$\ident{D}$). However, the resulting expression $\lambda x.e~x$ accesses a variable, 
marking the context and function argument as live. In case of data-flow, the immediate coeffects 
are made larger by the lambda abstraction -- the context requirements of the function value are 
imposed on the declaration site of the new lambda abstraction. We remedy this limitation in 
the next chapter.

However, $\eta$-expansion preserves the type for implicit parameters and, more generally,
for any flat coeffect algebra where $\cpar \,=\, \czip$. Assuming $\rightarrow_\eta$ is the 
standard $\eta$-reduction:

\begin{theorem}[Type preservation of $\eta$-expansion]
In a bottom-pointed flat coeffect calculus with an algebra $(\C, \cseq, \cpar, \czip, \cunit, \czero, \cleq)$ 
where $\czip = \cpar$, if $\coctx{\Gamma}{\cclrd{r}} \vdash e : \tau_1 \xrightarrow{\cclrd{s}} \tau_2$ 
and $e \rightarrow_\eta e'$ then $\coctx{\Gamma}{\cclrd{r}} \vdash e' : \tau_1 \xrightarrow{\cclrd{s}} \tau_2 $.
\end{theorem}
\begin{proof}
The following derivation shows that $\lambda x.f~x$ has the same type as $f$:
\begin{equation*}
\inference
  { \inference
    { \coctx{\Gamma}{\cclrd{r}} \vdash f : \tau_1 \xrightarrow{\cclrd{s}} \tau_2 &
      \coctx{x\!:\!\tau_1}{\cunit} \vdash x\!:\!\tau_1 }
    { \coctx{\Gamma, x\!:\!\tau_1}{\cclrd{r} \;\cpar\; (\cunit \,\cseq\, \cclrd{s}) } \vdash f~x : \tau_2 } }
  { \inference
    { \inference
      { \coctx{\Gamma, x\!:\!\tau_1}{\cclrd{r} \;\cpar\; \cclrd{s} } \vdash f~x : \tau_2 }
      { \coctx{\Gamma, x\!:\!\tau_1}{\cclrd{r} \;\czip\; \cclrd{s} } \vdash f~x : \tau_2 } }
    { \coctx{\Gamma}{ \cclrd{r} } \vdash \lambda x.f~x : \tau_1 \xrightarrow{\cclrd{s}} \tau_2 } }
\end{equation*}
%
The derivation starts with the expression $e$ and derives the type for $\lambda x.e~x$. The
application yields context requirements $\cclrd{r}\,\cpar\,\cclrd{s}$. In order to recover the
original typing, this must be equal to $\cclrd{r}\,\czip\,\cclrd{s}$. Note that the derivation
is showing just one possible typing -- the expression $\lambda x.e~x$ has other types -- but
this is sufficient for showing type preservation.
\end{proof}

\noindent
In summary, flat coeffect calculi do not \emph{in general} permit call-by-name evaluation, but
there are several cases where call-by-name evaluation can be used. Among the examples we discuss,
these include liveness and implicit parameters. Moreover, for implicit parameters (and more
generally, any set-like flat coeffect algebra), the $\eta$-expansion holds as well, giving us
both local soundness and local completeness as coined by Pfenning and Davies \cite{logic-modal-reconstruction}.



% ==================================================================================================
%
%    #######
%    #       #    # ##### ###### #    #  ####  #  ####  #    #  ####
%    #        #  #    #   #      ##   # #      # #    # ##   # #
%    #####     ##     #   #####  # #  #  ####  # #    # # #  #  ####
%    #         ##     #   #      #  # #      # # #    # #  # #      #
%    #        #  #    #   #      #   ## #    # # #    # #   ## #    #
%    ####### #    #   #   ###### #    #  ####  #  ####  #    #  ####
%
% ==================================================================================================

\section{Syntactic properties and extensions}
\label{sec:flat-exts}

The flat coeffect algebra introduced in Section~\ref{sec:flat-calculus} requires a number of laws.
The laws are required for three reasons -- to be able to define the categorical structure
in Section~\ref{sec:flat-semantics}, to prove equational properties in Section~\ref{sec:flat-syntax}
and finally, to guarantee intuitive syntactic properties for constructs such as 
$\lambda$-abstraction and pairs in context-aware calculi.

In this section, we look at the last point. We consider sub-coeffecting and subtyping 
(Section~\ref{sec:flat-exts-sub}), discuss what syntactic equivalences are permitted
by the properties of $\czip$ (Section~\ref{sec:flat-exts-lambda}) and we extend the calculus with 
pairs and units and discuss their syntactic properties (Section~\ref{sec:flat-exts-tup}). 

% --------------------------------------------------------------------------------------------------

\begin{figure}[t]
\begin{equation*}
\tyrule{sub-trans}
  { \tau_1 <: \tau_2 & \tau_2 <: \tau_3 }
  { \tau_1 <: \tau_3  }
\end{equation*}
\begin{equation*}
\tyrule{sub-fun}
  { \tau_1' <: \tau_1 & \tau_2 <: \tau_2' & \cclrd{r'} \cgeq \cclrd{r} }
  { \tau_1 \xrightarrow{\cclrd{r}} \tau_2 <: \tau_1' \xrightarrow{\cclrd{r'}} \tau_2' }
\end{equation*}
\begin{equation*}
\tyrule{sub-refl}
  { }
  { \tau <: \tau }
\end{equation*}

\figcaption{Subtyping rules for flat coeffect calculus}
\label{fig:flat-types-sub}
\end{figure}

% --------------------------------------------------------------------------------------------------

\subsection{Subcoeffecting and subtyping}
\label{sec:flat-exts-sub}

The \emph{flat coeffect algebra} includes the $\cleq$ relation which captures the ordering of
coeffects and can be used to define sub-coeffecting. Syntactically, an expression with context
requirements $\cclrd{r}$ can be treated as an expression with greater context requirements:
%
\begin{equation*}
\tyrule{sub}
  {\coctx{\Gamma}{\cclrd{r'}} \vdash e : \tau }
  {\coctx{\Gamma}{\cclrd{r}} \vdash e : \tau }\quad\quad(\cclrd{r'} \cleq \cclrd{r})
\end{equation*}
%
Semantically, this means that we can \emph{drop} some of the provided context. For example, 
if an expression requires implicit parameters $\{ \ident{?p} \}$ it can be treated as requiring
$\{ \ident{?p}, \ident{?q} \}$. The semantic function will then be provided with a dictionary
containing both assignments and it can drop the value for the unused parameter \ident{?q}.

Sub-coeffecting only affects the immediate coeffects attached to the free-variable context.
In Figure~\ref{fig:flat-types-sub}, we add sub-typing on function types, making it possible to treat
a function with smaller context requirements as a function with greater context requirements:
%
\begin{equation*}
\tyrule{typ}
  { \coctx{\Gamma}{\cclrd{r}} \vdash e : \tau & \tau <: \tau' }
  { \coctx{\Gamma}{\cclrd{r}} \vdash e : \tau' }
\end{equation*}
%
The definition uses the standard reflexive and transitive $<:$ operator. As the (\emph{sub-fun})
shows, the function type is contra-variant in the input and co-variant in the output. The 
(\emph{typ}) rule allows using sub-typing on expressions in the coeffect calculus.

%---------------------------------------------------------------------------------------------------

\subsection{Typing of let binding}
\label{sec:flat-exts-let}

Recall the (\emph{let}) rule in Figure~\ref{fig:flat-types}. It annotates the expression 
$\kvd{let}~x=e_1~\kvd{in}~e_2$ with context requirements $\cclrd{s}\;\cpar\;(\cclrd{s}\,\cseq\,\cclrd{r})$.
This is a special case of typing an expression $(\lambda x.e_2)~e_1$, using the idempotence
of $\czip$ as follows:
%
\begin{equation*}
\tyrule{app}
  {\begin{array}{l}
   \vspace{-1.5em}
   \coctx{\Gamma}{\cclrd{r}} \vdash e_1 : \tau_1
   \end{array} &
   \tyruler{abs}
       { \coctx{\Gamma, x\!:\!\tau_1}{\cclrd{s}} \vdash e_2 : \tau_2 }
       { \coctx{\Gamma}{\cclrd{s}} \vdash \lambda x.e_2 : \tau_1 \xrightarrow{\cclrd{s}} \tau_2 } }
  { \coctx{\Gamma}{\cclrd{s}\;\cpar\;(\cclrd{s}\,\cseq\,\cclrd{r})} \vdash (\lambda x.e_2)~e_1 : \tau_2 }    
\end{equation*}
%
This design decision is similar to ML value restriction, but it works the other way round. Our
\emph{let} binding is more restrictive than the typing of abstraction-application, rather than
being more general. The choice is motivated by the fact that the typing obtained using the special 
rule for let-binding is more precise for all the examples consider 
in this chapter. Table~\ref{tab:flat-simplelet} shows how the coeffect annotations are simplified 
for our examples.

\begin{table}
\begin{center}
\begin{tabular}{ | l | c | c |}
\hline
& \textbf{\footnotesize Definition\hspace{1em}} & \textbf{\footnotesize Simplified\hspace{1em}} \\ \hline
\hspace{-1em}{\footnotesize Implicit parameters} & $\cclrd{s} \cup (\cclrd{s} \cup \cclrd{r})$ & $\cclrd{s} \cup \cclrd{r}$ \\ \hline
\hspace{-1em}{\footnotesize Liveness} & $\cclrd{s} \sqcap (\cclrd{s} \sqcup \cclrd{r})$ & $\cclrd{s}$ \\ \hline
\hspace{-1em}{\footnotesize Data-flow} & $\mathit{max}(\cclrd{s}, \cclrd{s} + \cclrd{r})$ & $\cclrd{s} + \cclrd{r}$ \\ \hline
\end{tabular}
\end{center}

\vspace{-0.5em}
\figcaption{Simplified annotation for let binding in sample flat calculi instances}
\label{tab:flat-simplelet}
\end{table}

The simplified annotations directly follow from the definitions of particular flat coeffect 
algebras. It is perhaps somewhat unexpected that the annotation can be simplified in different
ways for different examples. 

To see that the simplified annotations are more precise, assume that we used arbitrary 
splitting $\cclrd{s} = \cclrd{s_1}\,\czip\,\cclrd{s_2}$ rather than idempotence. The
``Definition'' column would use $\cclrd{s_1}$ and $\cclrd{s_2}$ for the first and second 
$\cclrd{s}$, respectively. The corresponding simplified annotation would have 
$\cclrd{s_1}\,\czip\,\cclrd{s_2}$ instead of $\cclrd{s}$. For all our systems, the 
simplified annotation (on the right) is more precise than the original (on the left):
%
\begin{equation*}
\begin{array}{rclll}
\cclrd{s_1} \cup (\cclrd{s_2} \cup \cclrd{r}) &\supseteq& (\cclrd{s_1} \cup \cclrd{s_2}) \cup \cclrd{r} 
  && \textnormal{(implicit parameters)}\\
\cclrd{s_1} \sqcap (\cclrd{s_2} \sqcup \cclrd{r}) &\sqsupseteq&  (\cclrd{s_1} \sqcap \cclrd{s_2}) 
  && \textnormal{(liveness)} \\
\mathit{max}(\cclrd{s_1}, \cclrd{s_2} + \cclrd{r}) &\geq& \mathit{min}(\cclrd{s_1}, \cclrd{s_2}) + \cclrd{r} 
  &\quad& \textnormal{(data-flow)} \\
\end{array}
\end{equation*}
%
In other words, the inequality states that using idempotence, we get a more precise typing.
Using the $\cgeq$ operator this property can be expressed using the abstract operators of the flat coeffect algebra as:
%
\begin{equation*}
\cclrd{s_1} \;\cpar\; (\cclrd{s_2} \,\cseq\, \cclrd{r}) \;\cgeq\; 
  (\cclrd{s_1} \,\czip\, \cclrd{s_2}) \;\cpar\; ((\cclrd{s_1} \,\czip\, \cclrd{s_2}) \,\cseq\, \cclrd{r})
\end{equation*}
%
This property cannot be proved from other properties of the flat coeffect algebra. To make
the flat coeffect system as general as possible, we do not \emph{in general} require it as
an additional axiom, although the above examples provide reasonable basis for requiring 
that the specialized annotation for let binding is the least possible annotation for the 
expression $(\lambda x.e_2)~e_1$.

% --------------------------------------------------------------------------------------------------

\subsection{Properties of lambda abstraction}
\label{sec:flat-exts-lambda}

In Section~\ref{sec:flat-calculus-lambda}, we discussed how to reconcile two typings for
lambda abstraction -- for implicit parameters, the lambda function needs to split context 
requirements using $\cclrd{r} \cup \cclrd{s}$, but for data-flow and liveness it suffices to 
duplicate the requirement $\cclrd{r}$ of the body. We introduced the $\czip$ operation as 
a way of providing the additional abstraction.

In this section, we first identify coeffect calculi for which the simpler (\emph{idabs}) rule
introduced in Section~\ref{ec:flat-unique-ldf} is sufficient. Then we look at syntactic 
transformations corresponding to other common properties of $\czip$.

\paragraph{Simplified abstraction.}
Recall that $(\C, \czip)$ is a band, that is, $\czip$  is idempotent and associative. The 
idempotence means that the context requirements of the body can be required from both the 
declaration site and the call site. Thus, the (\emph{idabs}) typing is valid. For reference,
we repeat the earlier definitions here:

\begin{equation*}
\tyrule{idabs}
  {\coctx{\Gamma, x\!:\!\tau_1}{\cclrd{r}} \vdash e : \tau_2}
  {\coctx{\Gamma}{\cclrd{r}} \vdash \lambda x.e : \tau_1 \xrightarrow{\cclrd{r}} \tau_2 }
\tyrule{abs}
  {\coctx{\Gamma, x\!:\!\tau_1}{\cclrd{r}\,\czip\,\cclrd{r}} \vdash e : \tau_2}
  {\coctx{\Gamma}{\cclrd{r}} \vdash \lambda x.e : \tau_1 \xrightarrow{\cclrd{r}} \tau_2 }
\end{equation*}
% 
To derive (\emph{idabs}), we use idempotence on the body annotation $\cclrd{r}\;=\;\cclrd{r}\;\czip\;\cclrd{r}$
and then use the standard (\emph{abs}) rule. So, (\emph{idabs}) follows from (\emph{abs}), 
but the other direction is not necessarily the case. The following condition identifies 
coeffect calculi where (\emph{abs}) can be derived from (\emph{idabs}).

\begin{definition}
A flat coeffect algebra $(\C, \cseq, \cpar, \czip, \cunit, \czero, \cleq)$ is \emph{strictly oriented} if for all
$\cclrd{s}, \cclrd{r} \in \C$ it is the case that $\cclrd{r} \,\czip\, \cclrd{s} \;\cleq\; \cclrd{r}$.
\end{definition}

\begin{remark}
\label{thm:flat-alt-abs}
For a flat coeffect calculus with a strictly oriented algebra, equipped with sub-coeffecting and
subtyping, the standard (abs) rule can be derived from the (idabs) rule.
\end{remark}
\begin{proof}
The following derives the conclusion of (\emph{abs}) using (\emph{idabs}), sub-coeffecting, 
sub-typing and the fact that the algebra is \emph{strictly oriented}:

\begin{equation*}
\tyrule{typ}
  {\hspace{-5em} \tyrule{sub}
     {\hspace{-4em} \tyrule{idabs}  
        {\coctx{\Gamma, x\!:\!\tau_1}{\cclrd{r} \,\czip\, \cclrd{s}} \vdash e : \tau_2}
        {\coctx{\Gamma}{\cclrd{r} \,\czip\, \cclrd{s}} \vdash \lambda x.e : \tau_1 \xrightarrow{\cclrd{r} \,\czip\, \cclrd{s}} \tau_2 } }
     {\coctx{\Gamma}{\cclrd{r}} \vdash \lambda x.e : \tau_1 \xrightarrow{\cclrd{r} \,\czip\, \cclrd{s}} \tau_2} \;
           \textnormal{\footnotesize{($\cclrd{r} \;\cleq\; \cclrd{r} \,\czip\, \cclrd{s}$)}} \hspace{-5em} }
  {\coctx{\Gamma}{\cclrd{r}} \vdash \lambda x.e : \tau_1 \xrightarrow{\cclrd{s}} \tau_2} \; 
           \textnormal{\footnotesize{($\cclrd{r} \;\cleq\; \cclrd{r} \,\czip\, \cclrd{s}$)}}
\end{equation*}
\end{proof}

\noindent
The practical consequence of the Remark~\ref{thm:flat-alt-abs} is that, for strictly
oriented coeffect calculi (such as our liveness and data-flow computations), we can use
the (\emph{idabs}) rule and get an equivalent type system. This alternative formulation
removes the non-determinism of type checking that arises from the splitting of context 
requirements in the original (\emph{abs}) rule. 

\paragraph{Symmetry.}
The $\czip$ operation is idempotent and associative. In all of the three examples considered in 
this chapter, the operation is also \emph{symmetric}. To make our definitions more general, we
do not require this to be the case for \emph{all} flat coeffect systems. However, systems with
symmetric $\czip$ have the following property.

\begin{remark}
For a flat coeffect calculus such that $\cclrd{r}\,\czip\,\cclrd{s} = \cclrd{s}\,\czip\,\cclrd{r}$,
assuming that $\cclrd{r'}, \cclrd{s'}, \cclrd{t'}$ is a permutation of $\cclrd{r},\cclrd{s},\cclrd{t}$:
%
\begin{equation*}
\inference
  {\coctx{\Gamma, x\!:\!\tau_1, y:\tau_2}{\cclrd{r} \,\czip\, \cclrd{s} \,\czip\, \cclrd{t}} \vdash e : \tau_3}
  {\coctx{\Gamma}{\cclrd{r'}} \vdash \lambda x.\lambda y.e : 
      \tau_1 \xrightarrow{\cclrd{s'}} (\tau_2 \xrightarrow{\cclrd{t'}} \tau_3) }
\end{equation*} 
\end{remark}

\noindent
Intuitively, this means that the context requirements of a function with multiple arguments can be 
split arbitrarily between the declaration site and (multiple) call sites. In other words, it does
not matter how the context requirements are satisfied.

% --------------------------------------------------------------------------------------------------

\begin{figure}[t]
\begin{equation*}
\tyrule{pair}
  {\coctx{\Gamma}{\cclrd{r}} \vdash e_1 : \tau_1 & \coctx{\Gamma}{\cclrd{s}} \vdash e_2 : \tau_2}
  {\coctx{\cclrd{r}\,\cpar\,\cclrd{s}}{\Gamma} \vdash (e_1, e_2) : \tau_1 \times \tau_2 }
\end{equation*}
\begin{equation*}
\tyrule{proj}
  {\coctx{\Gamma}{\cclrd{r}} \vdash e : \tau_1 \times \tau_2 }
  {\coctx{\Gamma}{\cclrd{r}} \vdash \pi_i~e : \tau_{i} }
\end{equation*}
\begin{equation*}
\tyrule{unit}
  {}
  {\coctx{\Gamma}{\czero} \vdash () : \ident{unit} }
\end{equation*}

\figcaption{Typing rules for pairs and units}
\label{fig:flat-ext-types}
\end{figure}

% --------------------------------------------------------------------------------------------------

\subsection{Language with pairs and unit}
\label{sec:flat-exts-tup}

To show the key aspects of flat coeffect systems, the calculus introduced in Section~\ref{sec:flat-calculus} 
consists only of variables, abstraction, application and let binding. Here, we extend it 
with pairs and the unit value to sketch how it can be turned into a more complete programming 
language and to motivate the laws required about $\cpar$. The syntax of the language is extended as follows:

\noindent
\begin{equation*}
\begin{array}{rcl}
e &::=& \ldots \sep () \sep e_1, e_2 \\
\tau &::=& \ldots \sep \ident{unit} \sep \tau \times \tau
\end{array}
\end{equation*}
%
The typing rules for pairs and the unit value are shown in Figure~\ref{fig:flat-ext-types}.
The unit value (\emph{unit}) is annotated with the $\czero$ coeffect (the same as other constants).
Pairs, created using the $(e_1, e_2)$ expression, are annotated with a coeffect that combines
the coeffects of the two sub-expressions using the \emph{pointwise} operator $\cpar$. The operator
models the case when the (same) available context is split and passed to two independent 
sub-expressions. Finally, the (\emph{proj}) rule is uninteresting, because $\pi_i$ can be 
viewed as a pure function.

\paragraph{Properties.}
Pairs and the unit value in a lambda calculus typically form a monoid. Assuming $\simeq$ is an
isomorphism that performs appropriate transformation on values, without affecting other 
properties (here, coeffects) of the expressions. The monoid laws then correspond to 
the requirement that $(e_1, (e_2, e_3)) \simeq ((e_1, e_2), e_3)$ (associativity) and the
requirement that $((), e) \simeq e \simeq (e, ())$ (unit).

Thanks to the properties of $\cpar$, the flat coeffect calculus obeys the monoid laws for pairs. 
In the following, we assume that \ident{assoc} is a pure function transforming a pair $(x_1, (x_2, x_3))$ 
to a pair $((x_1, x_2), x_3)$. We write $e \equiv e'$ when for all $\Gamma, \tau$ and $\cclrd{r}$,
it is the case that $\coctx{\Gamma}{\cclrd{r}} \vdash e : \tau$ if and only if
$\coctx{\Gamma}{\cclrd{r}} \vdash e' : \tau$.

\begin{remark}
\label{thm:flat-tup-eq}
For a flat coeffect calculus with pairs and units, the following holds:
%
\begin{equation*}
\begin{array}{rclcl}
 \ident{assoc}~(e_1, (e_2, e_3)) &\equiv& ((e_1, e_2), e_3) &\qquad\qquad&(\emph{associativity}) \\
 \pi_1~(e, ()) &\equiv& e &&(\emph{right unit})\\
 \pi_2~((), e) &\equiv& e &&(\emph{left unit})\\
\end{array} 
\end{equation*}
\end{remark}
\begin{proof}
Follows from the fact that $(\C, \cpar, \czero)$ is a monoid and \ident{assoc}, $\pi_1$ and
$\pi_2$ are pure functions (treated as constants in the language).
\end{proof}

\noindent
The Remark~\ref{thm:flat-tup-eq} motivates the requirement of the monoid structure 
$(\C, \cpar, \czero)$ of the flat coeffect algebra. We require only unit and associativity
laws. In our three examples, the $\cpar$ operator is also symmetric, which additionally
gives us the property that $(e_1, e_2) \simeq (e_2, e_1)$.



% ==================================================================================================
%
%     #####
%    #     #  ####  #    #  ####  #      #    #  ####  #  ####  #    #  ####
%    #       #    # ##   # #    # #      #    # #      # #    # ##   # #
%    #       #    # # #  # #      #      #    #  ####  # #    # # #  #  ####
%    #       #    # #  # # #      #      #    #      # # #    # #  # #      #
%    #     # #    # #   ## #    # #      #    # #    # # #    # #   ## #    #
%     #####   ####  #    #  ####  ######  ####   ####  #  ####  #    #  ####
%                                                                             
% ==================================================================================================
\section{Conclusions}

This chapter presented the \emph{flat coeffect calculus} -- a unified system for tracking 
\emph{whole-context} properties of computations, that is properties related to the 
execution environment or the entire context in which programs are executed.
This is the first of the two \emph{coeffect calculi} developed in this thesis.

The flat coeffect calculus is parameterized by a \emph{flat coeffect algebra} that captures
the structure of the information tracked by the type system. We instantiated the system to 
capture three specific systems, namely liveness, data-flow and implicit parameters. However,
the system is more general and an capture numerous other applications outlined in 
Section~\ref{sec:applications-flat}.

An inherent property of flat coeffect systems is the ambiguity of the typing for lambda 
abstraction. The body of a function requires certain context, but the context can be often
provided by either the declaration-site or the call-site. Resolving this ambiguity has to be
done differently for each concrete coeffect system, depending on its specific notion of context.
We discussed this for implicit parameters, dataflow and liveness in Section~\ref{sec}.

Finally, we introduced the equational theory for flat coeffect calculus. Although each 
concrete instance of flat coeffect calculus models different notion of context, there are
syntactic properties that hold for all flat coeffect systems satisfying certain additional
conditions. In particular, two \emph{typing preservation} theorems prove that the operational
semantics for two classes of flat coeffect calculi (including liveness and implicit parameters) 
can be based on the standard call-by-name reduction.

In the next section, we move from abstract treatment of the flat coeffect calculus to a more
concrete discussion. We explain its category-theoretical motivation, we use it to define 
translational semantics (akin to Haskell's do notation) and we prove that well-typed programs
in flat coeffect calcului for implicit parameters and dataflow do not get stuck.
