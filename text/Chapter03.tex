\chapter{Context-aware applications} 
\label{ch:applications} 

%---------------------------------------------------------------------------------------------------

TODO

%===================================================================================================

\section{Through applications}
\label{sec:path-apps}

The general theme of this thesis is improving programming languages to better support writing
\emph{context-dependent} (or \emph{context-aware}) computations. With current trends in the 
computing industry such as mobile and ubiquitous computing, this is becoming an important topic.
In software engineering and programming community, a number of authors have addressed this problem
from different perspectives. Hirschfeld et al. propose \emph{Context-Oriented Programming} (COP)
as a methodology \cite{app-cop-method}, and the subject has also been addressed in mobile
computations \cite{app-cop-mobile,app-cop-mobile2}. In programming languages, Costanza 
\cite{app-cop-contextl} develops a domain-specific LISP-like language ContextL and Bardram 
\cite{app-cop-javafwk} proposes a Java framework for COP.

We approach the problem from a different perspective, building on the tradition of 
statically-typed functional programming languages and their theories. However, even in this field,
there is a number of calculi or language features that can be viewed as context-dependent.

%---------------------------------------------------------------------------------------------------

\subsection{Motivation for flat coeffects}

In a number of systems, the execution environment provides some additional data, resources or 
information about the execution context, but are independent of the variables used by the 
program. We look at implicit parameters and rebindable resources (that both provide additional
identifiers that can be accessed similarly to variables, but follow different scoping rules),
distributed programming, cross-compilation and data-flow.

%---------------------------------------------------------------------------------------------------

\paragraph{Implicit parameters} In Haskell, implicit parameters \cite{app-implicit-parameters} are 
a special kind of variables that may behave as dynamically scoped. This means, if a function uses 
parameter $\ident{?p}$, then the caller of the function must define $\ident{?p}$ and set its value.
Implicit parameters can be used to parameterise a computation (involving a chain of function calls)
without passing parameters explicitly as additional arguments of all involved functions. A simple 
language with implicit parameters has an expression $\ident{?p}$ to read a parameter value and an 
expression\footnote{Haskell uses $\kvd{let}~\ident{?p} = e_1~\kvd{in}~e_2$, but we use a different keyword to 
avoid confusion.} $\kvd{letdyn}~?p = e_1~\kvd{in}~e_2$ that sets a parameter $\ident{?p}$ to the value of $e_1$ 
and evaluates $e_2$ in a context containing $\ident{?p}$

An interesting question arises when we use implicit parameters in a nested function. The following 
function does some pre-processing and then returns a function that builds a formatted string based 
on two implicit parameters $\ident{?width}$ and $\ident{?size}$:
%
\begin{equation*}
\begin{array}{l}
\kvd{let}~\ident{format} = \lambda \ident{str}~\rightarrow \\
\quad \kvd{let}~\ident{lines} = \ident{formatLines}~\ident{str}~\ident{?width}~\kvd{in}\\
\quad (\lambda \ident{rest}~\rightarrow~\ident{append}~
         \ident{lines}~\ident{rest}~\ident{?width}~\ident{?size})\\
\end{array}
\end{equation*}
%
The body of the outer function accesses the parameter $\ident{?width}$, so it certainly requires a context 
$\{ \ident{?width} : \ident{int} \}$. The nested function (returned as a result) uses the parameter 
$\ident{?width}$, but in addition also uses $\ident{?size}$. Where should the parameters of the nested 
function come from?

In a purely dynamically scoped system, they would have to be defined when the user invokes the nested function.
However, in Haskell, implicit parameters behave as a combination of lexical and dynamic scoping. This means
that the nested function can capture the value of $\ident{?width}$ and require just $\ident{?size}$
In Haskell, this corresponds to the following type:
%
\begin{equation*}
(\ident{?width} :: \ident{Int}) \Rightarrow \ident{String} \rightarrow 
  ((\ident{?size} :: \ident{Int}) \Rightarrow \ident{String} \rightarrow \ident{string})
\end{equation*}
%
As a result, the function can be called as follows:
%
\begin{equation*}
\begin{array}{l}
\kvd{let}~\ident{formatHello} = \\
\quad(~\kvd{letdyn}~\ident{?width}=5~\kvd{in}\\
\quad~~\ident{format}~\texttt{"Hello"})~~\kvd{in}\\
\kvd{letdyn}~\ident{?size} = 10~\kvd{in}~\ident{formatHello}~\texttt{"world"}
\end{array}
\end{equation*}
%
This way of assigning type to \ident{format} and calling it is not the only possible, though. 
We could also say that the outer function requires both of the implicit parameters and the result
is a (pure) function with no context requirements. This interaction between implicit parameters 
and lambda abstraction demonstrates one of the key aspects of coeffects and will be discussed 
later. Implicit parameters will also sever as one of our examples in Chapter~Y.

%---------------------------------------------------------------------------------------------------

\paragraph{Type classes}
Implicit parameters are closely related to \emph{type classes} \cite{app-type-classes}. In Haskell,
type classes provide a principled form of ad-hoc polymorphism (overloading). When a code uses 
an overloaded operation (e.g.~comparison or numeric operators) a constraint is placed on the 
context in which the operation is used. For example:
%
\begin{equation*}
\begin{array}{l}
\ident{twoTimes}~::~\ident{Num}~\alpha \Rightarrow \alpha \rightarrow \alpha \\
\ident{twoTimes}~x=x+x
\end{array}
\end{equation*}
%
The constraint $\ident{Num}~a$ on the function type arises from the use of the $+$ operator. 
From the implementation perspective, the type class constraint means that the function takes 
a hidden parameter -- a dictionary that provides the operation $+ :: \alpha \rightarrow \alpha \rightarrow \alpha$.
Thus, the type $\ident{Num}~\alpha \Rightarrow \alpha \rightarrow \alpha$ can be viewed as
$(\ident{Num}_\alpha \times \alpha) \rightarrow \alpha$. Implicit parameters work in exactly
the same way -- they are passed around as hidden parameters.

The implementation of type classes and implicit parameters shows two important points about 
context-dependent properties. First, they are associated with some \emph{scope}, such as the body
of a function. Second, they are associated with the input. To call a function that takes an 
implicit parameter or has a type-class constraint, the caller needs to pass a (hidden) parameter
together with the function inputs.

%---------------------------------------------------------------------------------------------------

\paragraph{Rebindable resources}
The need for parameters that do not strictly follow static scoping rules also arises in distributed
computing. This problem has been addressed, for example, by Bierman et al. and Sewell et al. 
\cite{app-distributed-rebinding,app-distributed-acute}. To quote the first work: \emph{``Dynamic 
binding is required in various guises, for example when a marshalled value is received from the 
network, containing identifiers that must be rebound to local resources.''}

This situation arises when marshalling and transferring function values. A function may depend 
on a local resource (e.g.~a database available only on the server) and also resources that are 
available on the target node (e.g.~current time). In the following example, the construct
$\kvd{access}~\ident{Res}$ represents access to a re-bindable resource named $\ident{Res}$:
%
\begin{equation*}
\begin{array}{l}
\kvd{let}~\ident{recentEvents} = \lambda () \rightarrow\\
\quad\kvd{let}~\ident{db} = \kvd{access}~\ident{News}~\kvd{in}\\
\quad\ident{query}~\ident{db}~\str{"SELECT * WHERE Date > \%1"}~(\kvd{access}~\ident{Clock})
\end{array}
\end{equation*}
%
When \ident{recentEvents} is created on a server and sent to a client, a remote reference to the 
database (available only on the server) must be captured. If the client device supports a clock, 
then \ident{Clock} can be locally \emph{rebound}, e.g., to accommodate time-zone changes. 
Otherwise, the date and time needs to be obtained from the server too.

The use of re-bindable resources creates a context requirement similar to the one arising from
the use of implicit parameters. For function values, such context-requirements can be satisfied
in different ways -- resources must be available either at the declaration site (i.e.~when a 
function is created) or at the call site (i.e.~when a function is called).

%---------------------------------------------------------------------------------------------------

\paragraph{Distributed computing and multi-targetting}

An increasing number of programming languages is capable of running across multiple different 
platforms or execution environments. Functional programming languages that can be compiled to
JavaScript (to target web and mobile clients) include, among others, F\#, Haskell and OCaml \cite{app-ocaml-js}.

Links \cite{app-distributed-links}, F\# libraries \cite{app-fsharp-webapps,app-fsharp-webtools},
ML5 and QWeSST \cite{app-distributed-ml5, app-distributed-qwesst} and Hop \cite{app-hop-lang} go 
further and allow a single source program to be compiled to multiple target runtimes. This posses 
additional challenges -- it is necessary to track where each part of computation runs and statically 
guarantee that it will be possible to compile code to the required target platform 
(safe \emph{multi-targetting}).

We demonstrate the problem by looking at input validation. In distributed applications 
that communicate over unsecured HTTP channel, user input needs to be validated interactively
on the client-side (to provide immediate response) and then again on the server-side (to 
guarantee safety). For example:

\begin{equation*}
\begin{array}{l}
\kvd{let}~\ident{validateInput} = \lambda \ident{name} \rightarrow\\
\quad\ident{name} \neq \str{""} ~~\&\&~~ \ident{forall~isLetter~name}
\\[0.5em]
\kvd{let}~\ident{displayProduct} = \lambda \ident{name} \rightarrow\\
\quad\kvd{if}~\ident{validateInput name}~\kvd{then}~\ident{displayProductPage~name}\\
\quad\kvd{else}~\ident{displayErrorPage}~() 
\end{array}
\end{equation*}
%
The function \ident{validateInput} can be compiled to both JavaScript (for client-side) and
native code (for server-side). However, \ident{displayProduct} uses other functionality
(generating web pages) that is only available on the server-side, so it can only be compiled to
native code.

In Links \cite{app-distributed-links}, functions can be annotated as client-side, server-side
and database-side. F\# WebTools \cite{app-fsharp-webtools} adds functions that support multiple
targets (mixed-side). However, these are single-purpose language features and they are not 
extensible. For example, in modern mobile development it is also important to track minimal 
supported version of runtime\footnote{Android Developer guide \cite{app-android-multitarget} 
demonstrates how difficult it is to solve the problem without language support.}. 

Requirements on the execution environment can be viewed as contextual properties, but could be
also presented as effects (use of some API required only in certain environment is a computational
effect). We discuss the difference in Section~X. Furthermore, the theoretical foundations of
distributed languages like ML5 \cite{app-distributed-ml5} suggest that a contextual treatment
is more appropriate. We return to ML5 when discussing semantics in Section~\ref{sec:path-sem-contextdep}.

% --------------------------------------------------------------------------------------------------

\paragraph{Safe locking}
In the previous examples, the context provides additional values or functions that may be accessed
at runtime. However, it may also track \emph{permissions} to perform some operation. This is done
in the type system for safe locking of Flanagan and Abadi \cite{app-safe-locking}.

The system prevents race conditions (by only allowing access to mutable state under a lock)
and avoids deadlocks (by imposing strict partial order on locks). The following 
program uses a mutable state under a lock:
%
\begin{equation*}
\begin{array}{l}
\kvd{newlock}~l:\rho~\kvd{in}\\
\kvd{let}~\ident{state}~=~\ident{ref}_\rho~10~\kvd{in}\\
\kvd{sync}~l~(!\ident{state})
\end{array}
\end{equation*}
%
The declaration \kvd{newlock} creates a lock $l$ protecting memory region $\rho$. We can than
allocate mutable variables in that memory region (second line). An access to mutable variable
is only allowed in scope that is protected by a lock. This is done using the \kvd{sync} keyword,
which locks a lock and evaluates an expression in a context that contains permission to access
memory region of the lock ($\rho$ in the above example).

The type system for safe locking associates the list of permission with the variable context.
It uses judgements of a form $\Gamma, m \vdash e : \alpha$ specifying that an expression has a
type in context $\Gamma$, given permissions (a list of locked regions) $m$. However, the treatment
of lambda abstraction differs from the one for implicit parameters or rebindable resources.
In the system for locking, code inside lambda function cannot use permissions from the scope
where the function is declared. This is a necessary requirement -- a lambda function created 
under a lock cannot access protected memory, because it will be executed later. We discuss how
this restriction fits into our general coeffect framework in Section~X.Y.

%
% => This has coeffect style judgments, but it has effect-style lambda
%

%---------------------------------------------------------------------------------------------------

\paragraph{Data-flow languages}

The examples discussed so far are all -- to some extent -- similar. They attach additional 
information (implicit parameters, dictionaries) or restrictions (on execution environment) to the
context where code evaluates. By \emph{context}, we mean, most importantly, the values of variables
and declarations that are in scope. The examples so far add more information to the context, but
do not operate on the variable values.

Data-flow languages provide a different example. Lucid \cite{app-lucid} is a declarative data-flow 
language designed by Wadge and Ashcroft. In Lucid, variables represent streams and programs
are written as transformations over streams. A function application $\mathit{square}(a)$ represents
a stream of squares calculated from the stream of values $a$.

The data-flow approach has been successfully used in domains such as development of real-time embedded 
application where many \emph{synchronous languages} \cite{app-synchronous-lang} build on the data-flow
paradigm. The following example is inspired by the Lustre \cite{app-synchronous-lustre} language
and implements program to count the number of edges on a Boolean stream:
%
\begin{equation*}
\begin{array}{l}
\kvd{let}~\ident{edge} = \ident{false}~\kvd{fby}~(\ident{input}~\&\&~\ident{not}~(\kvd{prev}~\ident{input}))
\\[0.5em]
\kvd{let}~\ident{edgeCount} = \\
\quad 0~\kvd{fby}~ (~\kvd{if}~\ident{edge}~\kvd{then}~\kvd{prev}~\ident{edgeCount}\\
\quad\quad\quad~~~\, \kvd{else}~\kvd{prev}~\ident{edgeCount} ~)
\end{array}
\end{equation*}
%
The construct $\kvd{prev}~x$ returns a stream consisting of previous values of the stream 
$x$. The second value of $\kvd{prev}~x$ is first value of $x$ (and the first
value is undefined). The construct $y~\kvd{fby}~x$ returns a stream whose first element is the 
first element of $y$ and the remaining elements are values of $x$. Note that in Lucid, the constants
such as \ident{false} and $0$ are constant streams. Formally, the construct are defined as follows
(writing $x_n$ for $n$-th element of a stream $x$):
%
\[ 
(\kvd{prev}~x)_n = \left\{ 
  \begin{array}{ll}
    nil     & \; \text{if $n=0$}\\
    x_{n-1} & \; \text{if $n>0$}
  \end{array} \right.
\quad
(y~\kvd{fby}~x)_n = \left\{ 
  \begin{array}{ll}
    y_0     & \; \text{if $n=0$}\\
    x_n     & \; \text{if $n>0$}
  \end{array} \right.
\]  
%
When reading data-flow programs, we do not need to think about variables in terms of streams --
we can see them as simple values. However, the operations \kvd{fby} and \kvd{prev} cannot operate
on plain values -- they require additional \emph{context} which provides past values of variables
(for \kvd{prev}) and information about the current location in the stream (for \kvd{fby}). 

In this case, the context is not simply an additional (hidden) parameter. It completely changes
how variables must be represented. We may want to capture various \emph{contextual properties}
of Lucid programs. For example, how many past elements need to be cached when we evaluate the 
stream.

To understand the nature of the context, we later look at the semantics of Lucid. This can be
captured using a number of mathematical structures. Wadge \cite{app-lucid-monads} originally 
proposed to use monads, while Uustalu and Vene later used comonads \cite{app-dataflow-essence}.

%---------------------------------------------------------------------------------------------------

\subsection{Motivation for structural coeffects}

We now turn our attention to system where additional contextual information are associated not
with the context as a whole (or program scope), but with individual variables. We start by looking
simple static analysis -- variable \emph{liveness}. Then we revisit data-flow computations and
look at applications in security and software updating.

%---------------------------------------------------------------------------------------------------

\paragraph{Liveness analysis}

\emph{Live variable analysis} (LVA) \cite{app-modern-compiler} is a standard technique in compiler theory. 
It detects whether a free variable of an expression may be used by a program later (it is
\emph{live}) or whether it is definitely not needed (it is \emph{dead}). As an optimization, 
compiler can remove bindings to dead variables as the result is never accessed. Wadler 
\cite{app-strictness-absecnce} describes the property of a variable that is dead as the 
\emph{absence} of a variable. 

In this thesis, we first use a restricted (and not practically useful) form of liveness analysis
to introduce the theory of indexed comonads (Section~X) and then use liveness analysis as one of the
motivations for structural coeffects. Consider the following two simple functions:
%
\begin{equation*}
\begin{array}{l}
\kvd{let}~\ident{constant42} = \lambda \ident{x} \rightarrow 42\\
\kvd{let}~\ident{constant} = \lambda \ident{value} \rightarrow \lambda \ident{x} \rightarrow \ident{value}
\end{array}
\end{equation*}
%
In liveness analysis, we annotate the context with a value specifying whether the variables in
scope are \emph{live} or \emph{dead}. If we associate just a single value with the entire 
context, then the liveness analysis is very limited -- it can say that the context of the 
expression $42$ in the first function is dead, because no variables are accessed. 

A useful liveness analysis needs to consider individual variables. For example, in the body of
the second function (\ident{value}), two variables are in scope. The variable \ident{value} is
accessed and thus is \emph{live}, but the variable \ident{x} is dead.

Static analyses can be classified as either \emph{forward} or \emph{backward} (depending on how they 
propagate information) and as either \emph{must} or \emph{may} (depending on what properties they
guarantee). Liveness is a \emph{backward} analysis -- this means that the requirements propagates
from variables to their declaration sites. The distinction between \emph{must} and \emph{may} is 
apparent when we look at an example with conditionals:
%
\begin{equation*}
\begin{array}{l}
\kvd{let}~\ident{defaultArg}~= \lambda \ident{cond} \rightarrow \lambda \ident{input} \rightarrow\\
\quad\kvd{if}~\ident{cond}~\kvd{then}~42~\kvd{else}~\ident{input}
\end{array}
\end{equation*}
%
The liveness analysis is a \emph{may} analysis meaning that it marks variable as live when it
\emph{may} be used and as dead if it is \emph{definitely} not used. This means that the variable
\ident{input} is \emph{live} in the example above. A \emph{must} analysis would mark the variable
only if it was used in both of the branches (this is sometimes called \emph{neededness}).

The distinction between \emph{may} and \emph{must} analyses demonstrates the importance of 
interaction between contextual properties and certain language constructs such as conditionals.

% --------------------------------------------------------------------------------------------------

\paragraph{Data-flow languages (revisited)}
When discussing data-flow languages in the previous section, we said that the context provides 
past values of variables. This can be viewed as a flat contextual property (the context needs
to keep all past values), but we can also view it as a structural property. Consider the following
example:
%
\begin{equation*}
\kvd{let}~\ident{offsetZip} = 0~\kvd{fby}~(\ident{left} + \kvd{prev}~\ident{right})
\end{equation*}
%
The value \ident{offsetZip} adds values of \ident{left} with previous values of \ident{right}.
To evaluate a current value of the stream, we need the current value of \ident{left} and one past
value of \ident{right}. 

As mentioned earlier, a static analysis for data-flow computations could calculate how many past 
values must be cached. This can be done as a \emph{flat} coeffect analysis that produces just a 
single number for each function. However, we can design a more precise \emph{structural} analysis
and track the number of required elements for individual variables.

% --------------------------------------------------------------------------------------------------

\paragraph{Tainting and provenance}
Tainting is a mechanism where variables coming from potentially untrusted sources are marked
(\emph{tainted}) and the use of such variables is disallowed in contexts where untrusted input
can cause security issues or other problems. Tainting can be done dynamically as a runtime mark
(e.g.~in the Perl language) or statically using a type system. Tainting can be viewed as a special
case of \emph{provenance tracking}, known from database systems \cite{app-provenance-db}, where
values are annotated with more detailed information about their source.

Statically typed systems that based on tainting have been use to prevent cross-site scripting
attacks \cite{app-tainting-xss} and a well known attack known as SQL injection
\cite{app-tainting-sql,app-tainting-wasp}. In the latter chase, we want to check that SQL commands 
cannot be directly constructed from, potentially dangerous, inputs provided by the user. Consider the 
type checking of the following expression in a context containing variables \ident{id} and \ident{msg}:
%
\begin{equation*}
\begin{array}{l}
\kvd{let}~\ident{name} = \ident{query}~(\str{"SELECT Name WHERE Id = "}~+~\ident{id})~\kvd{in}\\
\ident{msg}~+~\ident{name}
\end{array}
\end{equation*}
%
In this example, \ident{id} must not come directly from a user input, because \ident{query} requires 
untainted string. Otherwise, the attacker could specify values such as \str{"1; DROP TABLE Users"}. 
The variable \ident{msg} may or may not be tainted, because it is not used in protected context 
(i.e.~to construct an SQL query). 

In runtime checking, all (string) values need to be wrapped in an object that stores Boolean 
flag (for tainting) or more complex data (for provenance). In static checking, the information
need to be associated with the variables in the variable context. We use tainting as a motivating
example for \emph{structural} coeffects in Section~X.

% --------------------------------------------------------------------------------------------------

\paragraph{Security and core dependency calculus}

The checking of tainting is a special case of checking of the \emph{non-interference} property 
in \emph{secure information flow}. Here, the aim is to guarantee that sensitive information (such
as credit card number) cannot be leaked to contexts with low secrecy (e.g.~sent via an unsecured
network channel). Volpano et al. \cite{app-secure-flow} provide the first (provably) sound type 
system that guarantees non-inference and Sabelfeld et al. \cite{app-secure-information-flow} survey
more recent work. The checking of information flows has been also integrated (as a single-purpose
extension) in the FlowCaml \cite{app-security-flowcaml} language. Finally, Russo et al. and 
Swamy et al. \cite{monad-secure-flow,monads-lightweight-ml} show that the properties can be checked
using a monadic library.

Systems for secure information flow typically define a lattice of security classes $(\mathcal{S}, \leq)$
where $\mathcal{S}$ is a finite set of classes and an ordering. For example a set $\{\ident{L}, \ident{H}\}$ 
represents low and high secrecy, respectively with $\ident{L} \leq \ident{H}$ meaning that low security
values can be treated as high security (but not the other way round).

An important aspect of secure information flow is called \emph{implicit flows}. Consider the following
example which may assign a new value to $z$:
%
\begin{equation*}
\kvd{if}~x>0~\kvd{then}~z := y
\end{equation*}
%
If the value of $y$ is high-secure, then $z$ becomes high-secure after the assignment
(this is an \emph{explicit} flow). However, if $x$ is high-secure, then the value of
$z$ becomes high-secure, regardless of the security level of $y$, because the fact whether an 
assignment is performed or not performed leaks information in its own (this is an 
\emph{implicit} flow).

Abadi et al. realized that there is a number of analyses similar to secure information flow
and proposed to unify them using a single model called Dependency Core Calculus (DCC) \cite{app-dcc}.
It captures other cases where some information about expression relies on properties of variables
in the context where it executes.  The DCC captures, for example, \emph{binding time analysis}
\cite{app-binding-time-analysis}, which detects which parts of programs can be partially evaluated
(do not depend on user input) and \emph{program slicing} \cite{app-slicing-survey} that identifies
parts of programs that contribute to the output of an expression.
	
% --------------------------------------------------------------------------------------------------

\subsection{Beyond passive contexts}

In the systems discussed so far, the context provides additional data (resources, implicit 
parameters, historical values) or meta-data (security, provenance). However, it is impossible to
write a function that modifies the context. We use the term \emph{passive} context for such 
applications. 

However, there is a number of systems where the context may be changed -- not just be evaluating
certain code block in a different scope (e.g. by wrapping it in $\ident{prev}$ in data-flow), but
also by calling a function that, for example, acquires new capabilities. While this thesis focuses
on systems with passive context, we quickly look at the most important examples of the 
\emph{active} variant.

% --------------------------------------------------------------------------------------------------

\paragraph{Calculus of capabilities}
Crary et al. \cite{app-capabilities} introduced the Calculus of Capabilities to provide 
a sound system with region-based memory management for low-level code that can be easily 
compiled to assembly language. They build on the work of Tofte and Talpin \cite{app-region-memory}
who developed an \emph{effect system} (discussed in Section~\ref{sec:path-sem-effects}) that uses
lexically scoped \emph{memory regions} to provide an efficient and controlled memory management.

In the work of Tofte and Talpin, the context is \emph{passive}. They extend a simple functional language
with the \kvd{letrgn} construct that defines a new memory region, evaluates an expression (possibly)
using memory in that region and then deallocates the memory of the region:
%
\begin{equation*}
\begin{array}{l}
\kvd{let}~\ident{calculate} = \lambda \ident{input} \rightarrow\\
\quad\kvd{letrgn}~\rho~\kvd{in}\\
\quad\kvd{let}~\ident{x} = \kvd{ref}_\rho \ident{input}~\kvd{in}~!\ident{x}
\end{array}
\end{equation*}
%
The memory region $\rho$ is a part of the context, but only in the scope of the body of 
\kvd{letrgn}. It is only available to the last line which allocates a memory cell in the region
and reads it (before the region is deallocated). There is no way to allocate a region inside a 
function and pass it back to the caller.

Calculus of capabilities differs in two ways. First, it allows explicit allocation and deallocation
of memory regions (and so region lifetimes do not follow strict LIFO ordering). Second, it
uses continuation-passing style. We ignore the latter aspect and so the following example:
%
\begin{equation*}
\begin{array}{l}
\kvd{let}~\ident{calculate} = \lambda \ident{input} \rightarrow\\
\quad \kvd{letrgn}~\rho~\kvd{in}\\
\quad \kvd{let}~\ident{x} = \kvd{ref}_\rho \ident{input}~\kvd{in}~\ident{x}
\end{array}
\end{equation*}
%
The example is almost identical to the previous one, except that it does not return the value
of reference \ident{x}. Instead, it returns the reference, which is located in a newly allocated
region. Together with the value, the function returns a \emph{capability} to access the region
$\rho$.

This is where systems with active context differ. To type check such programs, we do not only need
to know what context is required to call \ident{calculate}. We also need to know what effects it
has on the context when it evaluates and the current context meeds to be appropriately adjusted
after a function call. We briefly consider this problem in Section~X. % future work

% --------------------------------------------------------------------------------------------------

\paragraph{Software updating}
Dynamic software updating (DSU) \cite{app-dsu-programs,app-dsu} is the ability to update programs at
runtime without stopping them. The Proteus system developed by Stoyle et al. \cite{app-dsu-mutatis} 
investigates what language support is needed to enable safe dynamic software updating in C-like 
languages. The system is based on the idea of capabilities.

The system distinguishes between \emph{concrete} uses and \emph{abstract} uses of a value. When
a value is used concretely, the program examines its representation (and so it is not safe to
change the representation during an update). An abstract use of a value does not need to examine
the representation and so updating the value does not break the program.

The Proteus system uses capabilities to restrict what types may be used concretely after any point
in the program. All other types, not listed in the capability, can be dynamically updated as this
will not change concrete representation of types accessed later in the evaluation.

Similarly to Capability Calculus, capabilities in DSU can be changed by a function call. For 
example, calling a function that may update certain types makes it impossible to use those types
concretely following the function call. This means that DSU uses the context \emph{actively}
and not just \emph{passively}.

% --------------------------------------------------------------------------------------------------

\paragraph{Thesis perspective}

As demonstrated in this section, there is a huge number of systems and applications that exhibit
a form of context-dependence. The range includes different static analyses (liveness, provenance), 
well-known programming language features (implicit parameters and type classes) as well as features
not widely available (e.g. for distributed programming).

It is impossible to cover all of these topics in a single coherent thesis and so we focus on 
two key aspects:

\begin{compactitem}
\item \textbf{Flat vs. structural.} We look at both flat coeffects (single value for entire context) and 
  structural coeffects (single value per variable). We use liveness, implicit parameters and 
  data-flow to introduce flat coeffects (Section~X) and liveness, refined data-flow and tainting
  to talk about structural coeffects (Section~Y).
  
\item \textbf{Analysis vs. restriction.} Some of the discussed examples can be viewed as static
  analyses that obtain some information about programs (i.e.~the number of required past values
  in data-flow). Other examples provide type system that rules out certain invalid programs 
  (e.g.~safe locking). We cover this topic when discussing \emph{partial coeffects} in Section~Z.
  
\item \textbf{May vs. must analysis.} When discussing liveness, we observed that we can obtain 
  two different analyses depending on how conditionals are treated. We discuss this topic in 
  Section~X. % TBD - could be a chapter
\end{compactitem}

Although we also looked at examples of \emph{active} contextual computations (where developers can
write functions that modify the context), we do not consider these applications, to keep the 
material presented in this thesis focused. We briefly discuss them as future work in Section~X.

%===================================================================================================

\section{Summary}

TODO

% \section{Missing}
% ~
% 
% indexed/layered/etc. monads, productors or whatever