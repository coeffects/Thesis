\chapter{Coeffect meta-language} 
\label{ch:meta} 

%---------------------------------------------------------------------------------------------------

Both flat coeffect calculus and structural coeffect calculus (presented in the past two chapters)
use indexed comonads to define the semantics of the langauge. In this section, we follow the 
meta-language style and embed indexed comonads into the language -- the type constructor
$\ctyp{r}{\alpha}$ becomes a first-class value and we add language constructs corresponding to
primitive operations of the indexed comonad.

%===================================================================================================

\section{Introduction}
\label{sec:metalanguage-intro}

\section{Type system}
\ExecuteMetaData[rules/rules.tex]{cml-types}

\section{Operational properties}

\section{Categorical semantics}

\section{Applications}

\subsection{Meta-programming}

\subsection{Mobile computations}

\section{Related work}
This chapter is closely related to Contextual Modal Type Theory (CMTT) of Nanevski et al. However
they develop their language using model logic as a basis, while we use categorical foundations 
as the basis - leading to a different system.

\section{Summary}