%!TEX root = ../main.tex

\chapter{Appendix A} 
\label{ch:appendix} 

This appendix gives full typing derivations for the examplespresented in Chapter~\ref{ch:applications},
which demonstrate simple coeffect calculi for implicit parameters, liveness and data-flow.

% --------------------------------------------------------------------------------------------------

\section{Coeffect typing for implicit parameters}
\label{sec:appendixa-implicit} 

The example on page~\pageref{pg:applications-flat-paramsex} considers the typing of a function that
adds the values of two implicit parameters: $\lambda x.\ident{?a} + \ident{?b}$. Note that addition
is a function (and so we write $(+)~\ident{?a}~\ident{?b}$) that requires no implicit parameters.
Thus our typing derivation starts with: 
%
\begin{equation*}
\Gamma_0 = (+):\ident{int}\xrightarrow{\cclrd{\emptyset}}\ident{int}\xrightarrow{\cclrd{\emptyset}}\ident{int}
\end{equation*}
%
The typing for the sub-expression $(+)~\ident{?a}$ is shown in Figure~\ref{fig:appendixa-impl} (a).
Note that the resulting function does not require any implicit parameters, so there is no non-determinism so far.
It is worth noting that this would change if we used $\eta$-expansion and wrote $(\lambda y.(+)~\ident{?a})$.
We refer to the above as $\Delta$. The rest of the expresion is shown in Figure~\ref{fig:appendixa-impl} (b).

Here, the last rule applies (\emph{app}) and so we non-deterministically split the set of required resources.
This means that we need $\cclrd{r_1}, \cclrd{r_2}$ such that $\cclrd{r_1}\,\cup\,\cclrd{r_2}=\{\ident{?a}:\ident{int}, \ident{?b}:\ident{int} \}$.
The Figure~\ref{fig:appendixa-impl} (c) summarizes the 9 options.

% --------------------------------------------------------------------------------------------------

\section{Coeffect typing for liveness}
\label{sec:appendixa-liveness} 

\newcommand{\vrl}{\cclrd{\ident{L}}}
\newcommand{\vrd}{\cclrd{\ident{D}}}

The Section~\ref{sec:applications-flat-live} discusses a coeffect system where the coeffect 
annotations capture whether a variable may be accessed (marked as live using $\vrl$) or 
whether it is definitely not used (marked as dead using $\vrd$). The following three examples
are considered:
%
\begin{equation*}
\begin{array}{lll}
(\lambda x . 42)~y &~\hspace{1em}~&(1)\\
\ident{twoTimes}~42          &&(2)\\
(\lambda x . x)~42 &&(3)\\
\end{array}
\end{equation*}
%
The typing derivation for the expression ($1$) is shown in Figure~\ref{fig:appendixa-liv} (a).
The most interesting aspect about the previous example is the use of the (\emph{app}) rule, which
marks the resulting context as dead, even though a variable is accessed in the second part of the
expression (this part never needs to be evaluated). 

Assuming $\Gamma_0 = \ident{twoTimes} : \ident{int} \xrightarrow{\vrl} \ident{int}$,
the typing derivation for ($2$) is shown in Figure~\ref{fig:appendixa-liv} (b).
Here, the variable context is marked as live. This is not because the argument of \ident{twoTimes} 
is marked as live, but because the function itself is a variable that (always) needs to be obtained
from the variable context. Finally, the derivation for ($3$) is shown in Figure~\ref{fig:appendixa-liv} (c).

% --------------------------------------------------------------------------------------------------

\section{Coeffect typing for data-flow}
\label{sec:appendixa-dataflow}

The Section~\ref{sec:applications-flat-dataflow} presents a coeffect type system that tracks the 
maximal number of past values required by a data-flow computation. The discussion includes the 
following example:
%
\begin{equation*}
\begin{array}{l}
(\,   \kvd{if}~~(\kvd{prev}~\ident{tick})=0\\[-0.25em]
\,\,\, \kvd{then}~(\lambda x \rightarrow \kvd{prev}~x)\\[-0.25em]
\,\,\, \kvd{else}~(\lambda x \rightarrow x) \,)\qquad(\kvd{prev}~\ident{counter})
\end{array}
\end{equation*}
%
In order to give typing for the above example, we first need to extend the language with 
conditionals. The typing rule for the $\kvd{if}$ construct is:
%
\begin{equation*}
\tyrule{if}
  { \coctx{\Gamma}{\cclrd{m}} \vdash e : \ident{bool} &
    \coctx{\Gamma}{\cclrd{n}} \vdash e_1 : \tau &
    \coctx{\Gamma}{\cclrd{n}} \vdash e_2 : \tau }
  {\coctx{\Gamma}{\textnormal{max}(\cclrd{m},\cclrd{n})} \vdash \kvd{if}~e~\kvd{then}~e_1~\kvd{else}~e_2 : \tau_1 }
\end{equation*}
%
Given a condition that requires $\cclrd{m}$ past and two (alternative) bodies that each require
$\cclrd{n}$ past values, the composed expression requires at most the maximum of the two, \ie~the
same context is passed to both the condition and the body using point-wise composition. As we will
see, the fact that both branches require the same number of past values means that we need to use
the sub-coeffecting rule.

Assuming $\Gamma_0 = \ident{tick}:\ident{int}, \ident{counter}:\ident{int}$, the 
Figure~\ref{fig:appendixa-dfl} shows the typing derivation for the example. In the first part,
we derive the types for the sub-expressions of the application and the conditional. Note that 
to obtain compatible function types, we use sub-coeffecting \emph{before} abstraction in the
typing of $(\lambda x.x)$ -- the type $\ident{int} \xrightarrow{\cclrd{1}} \ident{int}$ is  a
valid overapproximation. Finally, the typing of the application yields the requirement of two 
past values, calculated as $\textnormal{max}(\cclrd{1}, \cclrd{1}+\cclrd{1})$.

\newpage
% --------------------------------------------------------------------------------------------------

\begin{figure}
{\small a.) Typing for the sub-expression $(+)~\ident{?a}$}

\begin{equation*}
\begin{array}{l}
 \tyrule{app}
  { \hspace{-15em}\tyrule{var}{}
    { \coctx{\Gamma_0, x:\ident{int}}{\cclrd{ \emptyset }} \vdash
      (+) : \ident{int} \xrightarrow{\cclrd{\emptyset}} \ident{int} \xrightarrow{\cclrd{\emptyset}} \ident{int}} \\ \hspace{10em}
    \tyruler{param}{}
    { \coctx{\Gamma_0, x:\ident{int}}{\cclrd{ \{ \ident{?a}:\ident{int} \} }} \vdash
      \ident{?a} : \ident{int} } }
  { \coctx{\Gamma_0, x:\ident{int}}{\cclrd{ \{ \ident{?a}:\ident{int} \} }} \vdash
    (+)~\ident{?a} : \ident{int} \xrightarrow{\cclrd{\emptyset}} \ident{int} }
\end{array}
\end{equation*}

\vspace{1em}
{\small b.) Typing for the expression $(+)~\ident{?a}~\ident{?b}$}

\begin{equation*}
\tyrule{app}
{ \Delta\qquad\qquad &
  \coctx{\Gamma_0, x:\ident{int}}{\cclrd{ \{ \ident{?b}:\ident{int} \} }} \vdash
  \ident{?b} : \ident{int} }
{ \tyrule{abs}
  { \coctx{\Gamma_0, x:\ident{int}}{\cclrd{ \{\ident{?a}:\ident{int}, \ident{?b}:\ident{int} \} }} \vdash 
    (+)~\ident{?a}~\ident{?b} : \ident{int} }
  { \coctx{\Gamma_0}{\cclrd{ r_1 }} \vdash 
    \lambda x.(+)~\ident{?a}~\ident{?b} : 
    \ident{int} \xrightarrow{ \cclrd{r_2} } \ident{int} } }
\end{equation*}

\vspace{1em}
{\small c.) Possible splittings of the implicit parameters }

\begin{equation*}
\begin{array}{ll}
 \cclrd{r_1} = \{ \} & \cclrd{r_2} = \{ \ident{?a}:\ident{int}, \ident{?b}:\ident{int} \} \\
 \cclrd{r_1} = \{ \ident{?a}:\ident{int} \} & \cclrd{r_2} = \{ \ident{?a}:\ident{int}, \ident{?b}:\ident{int} \} \\
 \cclrd{r_1} = \{ \ident{?b}:\ident{int} \} & \cclrd{r_2} = \{ \ident{?a}:\ident{int}, \ident{?b}:\ident{int} \} \\
 \cclrd{r_1} = \{ \ident{?a}:\ident{int}, \ident{?b}:\ident{int} \} & \cclrd{r_2} = \{ \ident{?a}:\ident{int}, \ident{?b}:\ident{int} \} \\
 \cclrd{r_1} = \{ \ident{?a}:\ident{int}, \ident{?b}:\ident{int} \} & \cclrd{r_2} = \{ \}\\
 \cclrd{r_1} = \{ \ident{?a}:\ident{int}, \ident{?b}:\ident{int} \} & \cclrd{r_2} = \{ \ident{?a}:\ident{int} \}\\
 \cclrd{r_1} = \{ \ident{?a}:\ident{int}, \ident{?b}:\ident{int} \} & \cclrd{r_2} = \{ \ident{?b}:\ident{int} \}\\
 \cclrd{r_1} = \{ \ident{?a}:\ident{int} \} & \cclrd{r_2} = \{ \ident{?b}:\ident{int} \} \\
 \cclrd{r_1} = \{ \ident{?b}:\ident{int} \} & \cclrd{r_2} = \{ \ident{?a}:\ident{int} \} \\
\end{array}
\end{equation*}

\figcaption{Coeffect typing for implicit parameters}
\label{fig:appendixa-impl}
\end{figure}

% --------------------------------------------------------------------------------------------------

\begin{figure}[t]
{\small a.) Typing for the expression $(\lambda x . 42)~y$}

\begin{equation*}
\tyrule{app}
  { \hspace{-5em}
    \tyrule{abs}
      { \hspace{-5em}
        \tyrule{const}{}{ \coctx{y\!:\!\tau, x\!:\!\tau}{\vrd} \vdash 42 : \ident{int} } }
      { \coctx{y\!:\!\tau}{\vrd} \vdash \lambda x . 42 : \tau \xrightarrow{\vrd} \ident{int} }  &
    \tyrule{var}{}{\coctx{y\!:\!\tau}{\vrl} \vdash y : \tau } }
  { \inference 
    { \coctx{y\!:\!\tau}{\vrd \sqcup (\vrl \sqcap \vrd)} \vdash (\lambda x . 42)~y : \ident{int} }
    { \coctx{y\!:\!\tau}{\vrd} \vdash (\lambda x . 42)~y : \ident{int} } }
\end{equation*}

\vspace{1em}
{\small b.) Typing for the expression $\ident{twoTimes}~42$}

\begin{equation*}
\tyrule{app}
  { \hspace{-5em}
    \tyrule{var}{}{ \coctx{\Gamma_0}{\vrl} \vdash \ident{twoTimes} : \ident{int} \xrightarrow{\vrl} \ident{int}  }&
    \tyrule{var}{}{ \coctx{\Gamma_0}{\vrd} \vdash 42 : \ident{int} } }
  { \inference 
      { \coctx{\Gamma_0}{\vrl \sqcup (\vrd \sqcap \vrl)} \vdash \ident{twoTimes}~42 : \ident{int}}
      { \coctx{\Gamma_0}{\vrl} \vdash \ident{twoTimes}~42 : \ident{int}} }
\end{equation*}

\vspace{1em}
{\small c.) Typing for the expression $(\lambda x . x)~42$}

\begin{equation*}
\tyrule{app}
  { \hspace{-5em}
    \tyrule{abs}
      { \hspace{-5em}
        \tyrule{var}{}{ \coctx{x\!:\!\ident{int}}{\vrl} \vdash x : \ident{int} } }
      { \coctx{()}{\vrl} \vdash \lambda x . x : \ident{int} \xrightarrow{\vrl} \ident{int} }  &
    \tyrule{const}{}{\coctx{()}{\vrd} \vdash 42 : \ident{int} } }
  { \inference 
    { \coctx{()}{\vrl \sqcup (\vrl \sqcap \vrd)} \vdash (\lambda x . x)~42 : \ident{int} }
    { \coctx{()}{\vrl} \vdash (\lambda x . x)~42 : \ident{int} } }
\end{equation*}


\figcaption{Coeffect typing for liveness}
\label{fig:appendixa-liv}
\end{figure}

% --------------------------------------------------------------------------------------------------

\begin{figure}[t]
{\small a.) Typing for sub-expressions of the conditional and for the argument}

\begin{equation*}
\tyrule{prev}
 { \coctx{\Gamma_0}{\cclrd{0}} \vdash \ident{tick} : \ident{int} }
 { \inference
   { \coctx{\Gamma_0}{\cclrd{1}} \vdash \kvd{prev}~\ident{tick} : \ident{int} }
   { \coctx{\Gamma_0}{\cclrd{1}} \vdash (\kvd{prev}~\ident{tick}) = 0 : \ident{bool} } }
\end{equation*}
\vspace{1em}
\begin{equation*}
\tyrule{prev}
 { \tyrule{var}
    {} 
    { \coctx{\Gamma_0,x\!:\!\ident{int}}{\cclrd{0}} \vdash x : \ident{int} } }
 { \tyrule{abs}
   { \coctx{\Gamma_0,x\!:\!\ident{int}}{\cclrd{1}} \vdash \kvd{prev}~x : \ident{int} }
   { \coctx{\Gamma_0}{\cclrd{1}} \vdash \lambda x . \kvd{prev}~x : \ident{int} \xrightarrow{\cclrd{1}} \ident{int} } }
\end{equation*}
\vspace{1em}
\begin{equation*}
\tyrule{abs}
  { \tyrule{sub}
      { \tyrule{var}{}{ \coctx{\Gamma_0, x\!:\!\ident{int}}{\cclrd{0}} \vdash x : \ident{int} } }
      { \coctx{\Gamma_0, x\!:\!\ident{int}}{\cclrd{1}} \vdash x : \ident{int} } } 
  { \coctx{\Gamma_0}{\cclrd{0}} \vdash \lambda x . x : \ident{int} \xrightarrow{\cclrd{1}} \ident{int} }
\end{equation*}
\vspace{1em}
\begin{equation*}
\tyrule{prev}
  { \tyrule{var}{}{ \coctx{\Gamma_0}{\cclrd{0}} \vdash \ident{counter} : \ident{int} } }
  { \coctx{\Gamma_0}{\cclrd{1}} \vdash \kvd{prev}~\ident{counter} : \ident{int} }
\end{equation*}

\vspace{1em}
{\small b.) Typing for the composed expression}

\begin{equation*}
\tyrule{app}
  { \hspace{-5em}
    \tyrule{if}
      { (\ldots) } 
      { \coctx{\Gamma_0}{\cclrd{1}} \vdash (\kvd{if}\ldots\kvd{then}\ldots\kvd{else}\ldots) : \ident{int} \xrightarrow{\cclrd{1}} \ident{int} } & 
        \coctx{\Gamma_0}{\cclrd{1}} \vdash (\ldots) : \ident{int} }
  { \hspace{-1em}
    \inference
     { \coctx{\Gamma_0}{\textnormal{max}(\cclrd{1}, \cclrd{1}+\cclrd{1})} \vdash (\kvd{if}\ldots\kvd{then}\ldots\kvd{else}\ldots)~(\ldots) : \ident{int} }
     { \coctx{\Gamma_0}{\cclrd{2}} \vdash (\kvd{if}\ldots\kvd{then}\ldots\kvd{else}\ldots)~(\ldots) : \ident{int} } }
\end{equation*}

\figcaption{Coeffect typing for data-flow}
\label{fig:appendixa-dfl}
\end{figure}

% --------------------------------------------------------------------------------------------------