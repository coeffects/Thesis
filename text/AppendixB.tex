%!TEX root = ../main.tex

\chapter{Appendix A} 
\label{ch:appendix} 

This appendinx provides additional details for some of the proofs for equational theory
of flat coeffect calculus from Chapter~\ref{ch:flat} and structural coeffect calculus
from Chapter~\ref{ch:structural}.

\section{Substitution for flat coeffects}
\label{sec:appendix-flat-cbn}
In Section~\ref{sec:flat-syntax-cbn}, we stated that, for a bottom-pointed flat coeffect
algebra (\ie~$\forall r \in \C \;.\; r \,\cgeq\, \cunit $), the call-by-name substitution 
preserves type if all operators of the flat coeffect algebra coincide (Lemma~\ref{thm:cbn-substitution-bot}).
This section provides the corresponding proof.

\begin{lemma*}[Bottom-pointed substitution]
In a bottom-pointed flat coeffect calculus with an algebra $(\C, \cseq, \cpar, \czip, \cunit, \czero, \cleq)$ 
where $\czip = \cseq = \cpar$ and the operation is also idempotent and commutative and
$\cclrd{r}\,\cleq\,\cclrd{r'} \Rightarrow \forall\cclrd{s}.\cclrd{r}\,\cseq\,\cclrd{s}\;\cleq\;\cclrd{r'}\,\cseq\,\cclrd{s}$ then:
%
\begin{equation*}
\begin{array}{l}
 \coctx{\Gamma}{\cclrd{S}} \vdash e_s : \tau_s \;\; \wedge \;\; 
 \coctx{\Gamma_1,  x : \tau_s, \Gamma_2}{ \cclrd{r}  } \vdash e_r : \tau_r\\
\quad \Rightarrow \;\; \coctx{\Gamma_1,\Gamma,\Gamma_2}{ \cclrd{r} \,\cseq\, \cclrd{S} } \vdash \subst{e_r}{x}{e_s} : \tau_r
\end{array}
\end{equation*}

\end{lemma*}
\begin{proof}
Assume that $\coctx{\Gamma}{\cclrd{S}} \vdash e_s : \tau_s$ and we are substituting a term $e_s$ for 
a variable $x$. Note that we use upper-case $\cclrd{S}$ to distinguish the coeffect of the expression
that is being substituted into an expression. Using structural induction over $\vdash$:

\paragraph{(var)} Given the following derivation using (\emph{var}):
\[
\inference
  { }
  {\coctx{\Gamma_1, y:\tau, \Gamma_2}{\cunit} \vdash y : \tau }
\]
There are two cases depending on whether $y$ is the variable $x$ or not:
\begin{itemize}
\item[--] If $y=x$ then also $\tau = \tau_s$ and thus $\subst{y}{x}{e_s} = e_s$. Using the assumption,
implicit weakening and the fact that $\cunit$ is a unit of $\cseq$:
\[
\inference
 {\inference 
  {\coctx{\Gamma}{\cclrd{S}} \vdash \subst{y}{x}{e_s} : \tau_s }
  {\coctx{\Gamma_1, \Gamma, \Gamma_2}{\cclrd{S}} \vdash \subst{y}{x}{e_s} : \tau } }
 {\coctx{\Gamma_1, \Gamma, \Gamma_2}{\cunit\,\cseq\,\cclrd{S}} \vdash \subst{y}{x}{e_s} : \tau } 
\]
\item[--] If $y\neq x$ then $\subst{y}{x}{e_s} = y$. Using the fact that $\cunit$ is the bottom element
and sub-coeffecting:
\[
\inference
  {\coctx{\Gamma_1, y:\tau, \Gamma_2}{\cunit} \vdash y : \tau }
  {\coctx{\Gamma_1, y:\tau, \Gamma_2}{\cunit\,\cseq\,\cclrd{S}} \vdash y : \tau }
\]
\end{itemize}

\paragraph{(const)} Similar to the (\emph{var}) case when the variable is not substituted.

\paragraph{(sub)} Given the following typing derivation using (\emph{sub}):
\[
\inference
  {\coctx{\Gamma_1,x:\tau_s,\Gamma_2}{\cclrd{r'}} \vdash e : \tau }
  {\coctx{\Gamma_1,x:\tau_s,\Gamma_2}{\cclrd{r}} \vdash e : \tau }\quad\quad(\cclrd{r'} \cleq \cclrd{r})
\]
From the induction hypothesis, we have that
$\coctx{\Gamma_1,\Gamma,\Gamma_2}{\cclrd{r'}\,\cseq\,\cclrd{S}} \vdash \subst{e}{x}{e_s} : \tau$.
The condition on $\cleq$ means that $\cclrd{r'}\,\cseq\,\cclrd{S}\;\cleq\;\cclrd{r}\,\cseq\,\cclrd{S}$
and so we can apply the (\emph{sub}) rule to obtain
$\coctx{\Gamma_1,\Gamma,\Gamma_2}{\cclrd{r}\,\cseq\,\cclrd{S}} \vdash \subst{e}{x}{e_s} : \tau$.

\paragraph{(abs)} Given the following typing derivation using (\emph{abs}):
\[
\inference
  {\coctx{\Gamma_1,x:\tau_s,\Gamma_2, y:\tau_1}{\cclrd{r}\;\czip\;\cclrd{s}} \vdash e : \tau_2}
  {\coctx{\Gamma_1,x:\tau_s,\Gamma_2}{\cclrd{r}} \vdash \lambda y.e : \tau_1 \xrightarrow{\cclrd{s}} \tau_2 }
\]
Assume w.l.o.g. that $x\neq y$. From the induction hypothesis, we have that:
\[
\coctx{\Gamma_1,\Gamma,\Gamma_2, y:\tau_1}{ \cclrd{r} \,\cseq\, \cclrd{s} } \vdash \subst{e}{x}{e_s} : \tau_2
\]
Now using the fact that $\czip\,=\,\cseq$, associativity and commutativity and (\emph{abs}):
\[
\inference
 {\inference
  {\coctx{\Gamma_1,\Gamma,\Gamma_2, y:\tau_1}{ (\cclrd{r} \,\czip\, \cclrd{s})\,\cseq\,\cclrd{S} } \vdash \subst{e}{x}{e_s} : \tau_2}
  {\coctx{\Gamma_1,\Gamma,\Gamma_2, y:\tau_1}{ (\cclrd{r} \,\cseq\, \cclrd{S})\,\czip\,\cclrd{s} } \vdash \subst{e}{x}{e_s} : \tau_2}}
 {\inference
  {\coctx{\Gamma_1,\Gamma,\Gamma_2}{\cclrd{r} \,\cseq\, \cclrd{S}} \vdash \lambda y.(\subst{e}{x}{e_s}) : \tau_1 \xrightarrow{\cclrd{s}} \tau_2}
  {\coctx{\Gamma_1,\Gamma,\Gamma_2}{\cclrd{r} \,\cseq\, \cclrd{S}} \vdash \subst{(\lambda y.e)}{x}{e_s} : \tau_1 \xrightarrow{\cclrd{s}} \tau_2} }
\]

\paragraph{(app)} Given the following typing derivation using (\emph{app}):
\[
\inference
  {\coctx{\Gamma_1,x:\tau_s,\Gamma_2}{\cclrd{r}} \vdash e_1 : \tau_1 \xrightarrow{\cclrd{t}} \tau_2 &
   \coctx{\Gamma_1,x:\tau_s,\Gamma_2}{\cclrd{s}} \vdash e_2 : \tau_1 }
  {\coctx{\Gamma_1,x:\tau_s,\Gamma_2}{\cclrd{r} \;\cpar\; (\cclrd{s} \,\cseq\, \cclrd{t})} \vdash e_1~e_2 : \tau_2}
\]
From the induction hypothesis, we have that:
\[
\begin{array}{l}
 \coctx{\Gamma_1, \Gamma, \Gamma_2}{\cclrd{r}\,\cseq\,\cclrd{S}} \vdash \subst{e_1}{x}{e_s} : \tau_1 \xrightarrow{\cclrd{t}} \tau_2 \\
 \coctx{\Gamma_1, \Gamma, \Gamma_2}{\cclrd{s}\,\cseq\,\cclrd{S}} \vdash \subst{e_2}{x}{e_s} : \tau_1
\end{array}\quad(*)
\]
Now using (\emph{app}) rule and the fact that $\cpar\,=\,\cseq$, associativity, commutativity and idempotence
(note that all three properties are needed):
\[
\inference
 { (*) }
 {\inference
   { \coctx{\Gamma_1,\Gamma,\Gamma_2}{(\cclrd{r}\,\cseq\,\cclrd{S}) \;\cpar\; ((\cclrd{s}\,\cseq\,\cclrd{S}) \,\cseq\, \cclrd{t})} 
       \vdash \subst{e_1}{x}{e_s}~\subst{e_2}{x}{e_s} : \tau_2}
   { \coctx{\Gamma_1,\Gamma,\Gamma_2}{(\cclrd{r} \;\cpar\; (\cclrd{s} \,\cseq\, \cclrd{t}))\;\cseq\;\cclrd{S}} 
       \vdash \subst{(e_1~e_2)}{x}{e_s} : \tau_2} }
\]

\paragraph{(let)} Given the following typing derivation using (\emph{let}):
\[
\inference
  { \coctx{\Gamma_1,x:\tau_s,\Gamma_2}{\cclrd{r}} \vdash e_1 : \tau_1 &
    \coctx{\Gamma_1,x:\tau_s,\Gamma_2, y:\tau_1}{\cclrd{s}} \vdash e_2 : \tau_2}
  {\coctx{\Gamma_1,x:\tau_s,\Gamma_2}{\cclrd{s} \;\cpar\; (\cclrd{s} \,\cseq\, \cclrd{r})} \vdash \kvd{let}~y=e_1~\kvd{in}~e_2 : \tau_2 }
\]
From the induction hypothesis, we have that:
\[
\begin{array}{l}
 \coctx{\Gamma_1, \Gamma, \Gamma_2}{\cclrd{r}\,\cseq\,\cclrd{S}} \vdash \subst{e_1}{x}{e_s} : \tau_1 \\
 \coctx{\Gamma_1, \Gamma, \Gamma_2, y:\tau_1}{\cclrd{s}\,\cseq\,\cclrd{S}} \vdash \subst{e_2}{x}{e_s} : \tau_2
\end{array}\quad(\dagger)
\]
Now using (\emph{let}) rule and similarly to the (\emph{app}) case:
\[
\inference
 { (\dagger) }
 { \inference
   { \coctx{\Gamma_1, \Gamma, \Gamma_2}{(\cclrd{s}\,\cseq\,\cclrd{S}) \;\cpar\; ((\cclrd{s}\,\cseq\,\cclrd{S}) \,\cseq\, (\cclrd{r}\,\cseq\,\cclrd{S}))} 
         \vdash \kvd{let}~y=\subst{e_1}{x}{e_s}~\kvd{in}~\subst{e_2}{x}{e_s} : \tau_2 } 
   { \coctx{\Gamma_1, \Gamma, \Gamma_2}{(\cclrd{s} \;\cpar\; (\cclrd{s} \,\cseq\, \cclrd{r})) \,\cseq\, \cclrd{S}} 
         \vdash \subst{(\kvd{let}~y=e_1~\kvd{in}~e_2)}{x}{e_s} : \tau_2 } }
\]
\end{proof}

%
% (e2 + (e1 * e2)) + (e3 * f)
% ----(glet)------
% ----------(app)------------
%
% = e2 + e1*e2 + e3*f
%  
% ~~~>
%
% (e2 + (e3 * f)) + ((e2 + (e3 * f)) * e1)
% ----(app)------    ----(app)-----
% ---------------(glet)-------------------
%
% = e2 + e3*f + e2*f + e1*e3*f


% e3 + ((e2 + (e1 * e2)) * e3)
%       ------(glet)---
% ---------(glet)-------------
%
% = e3 + e2*e3 + e1*e2*e3
%  
% ~~~>
%
% (e2 + (e3 * e2)) + (e1 * (e2 + (e3 * e2)))
% -----(glet)-----         -----(glet)-----
% ------------------(glet)------------------
%
% = e2 + e3*e2 + e1*e2 + e1*e2*e3

% Data-flow: e3 + (e2 + e1) = (e2 + e3) + e1
% Liveness:  e3 = e3
% Resources: OK

%---------------------------------------------------------------------------------------------------

% \section{Internalized substitution}

% \subsection{First transformation}

% \begin{equation*}
% (\kvd{glet}~x=e_1~\kvd{in}~e_2)~e_3 \leadsto \kvd{glet}~x=e_1~\kvd{in}~(e_2~e_3)
% \end{equation*}

% \begin{equation*}
% \tyrule{app}
%   { \tyrule{glet}
%       { \coctx{\Gamma}{\cclrd{s}} \vdash e_1 : \tau_1 &
%         \coctx{\Gamma, x\!:\!\tau_1}{\cclrd{r}} \vdash e_2 : \tau_3 \xrightarrow{\cclrd{t}} \tau_2 }
%       { \coctx{\Gamma}{\cclrd{r}\;\cpar\;({\cclrd{s}\,\cseq\,\cclrd{r})}} \vdash \kvd{glet}~x=e_1~\kvd{in}~e_2 : \tau_3 \xrightarrow{\cclrd{t}} \tau_2 } &
%     \coctx{\Gamma}{\cclrd{u}} \vdash e_3 : \tau_3  }
%   { \coctx{\Gamma}{(\cclrd{r}\;\cpar\;(\cclrd{s}\,\cseq\,\cclrd{r})) \;\cpar\; (\cclrd{u}\,\cseq\,\cclrd{t}) } \vdash (\kvd{glet}~x=e_1~\kvd{in}~e_2)~e_3 : \tau_2 }
% \end{equation*}

% to

% \begin{equation*}
% \tyrule{glet}
%   { \coctx{\Gamma}{\cclrd{s}} \vdash e_1 : \tau_1 &
%     \tyrule{app}
%       { \coctx{\Gamma, x\!:\!\tau_1}{\cclrd{r}} \vdash e_2 : \tau_3 \xrightarrow{\cclrd{t}} \tau_2 & 
%         \coctx{\Gamma}{\cclrd{u}} \vdash e_3 : \tau_3 }
%       { \coctx{\Gamma}{\cclrd{r}\;\cpar\;(\cclrd{u}\,\cseq\,\cclrd{t})} \vdash e_2~e_3 : \tau_2 } }
%   { \coctx{\Gamma}{ (\cclrd{r}\;\cpar\;(\cclrd{u}\,\cseq\,\cclrd{t})) \;\cpar\; (\cclrd{s}\,\cseq\,(\cclrd{r}\;\cpar\;(\cclrd{u}\,\cseq\,\cclrd{t})))  }
%       \vdash \kvd{glet}~x=e_1~\kvd{in}~(e_2~e_3) : \tau_2 }
% \end{equation*}

% meaning
% \begin{equation*}
% \begin{array}{l}
% (\cclrd{r}\;\cpar\;(\cclrd{s}\,\cseq\,\cclrd{r})) \;\cpar\; (\cclrd{u}\,\cseq\,\cclrd{t}) =
% \end{array}
% \end{equation*}

% \subsection{Second transformation}


% Second transformation

% \begin{equation*}
% \tyrule{glet}
%   { \coctx{\Gamma}{\cclrd{s}} \vdash e_s : \tau_s &
%     \coctx{\Gamma, x\!:\!\tau_1}{\cclrd{r}} \vdash e_r : \tau_r &
%     \coctx{\Gamma, x\!:\!\tau_1}{\cclrd{t}} \vdash e_t : \tau_t }
%   { \coctx{\Gamma}{
%       \cclrd{t}
%       \;\cpar\;
%       (  (\cclrd{r}\;\cpar\;(\cclrd{s}\,\cseq\,\cclrd{r}))
%          \,\cseq\, 
%          \cclrd{t} )
%     } 
%     \vdash \kvd{glet}~x_r=(\kvd{glet}~x_s=e_s~\kvd{in}~e_r)~\kvd{in}~e_t : \tau_t }
% \end{equation*}

% or

% \begin{equation*}
% \tyrule{glet}
%   { \coctx{\Gamma}{\cclrd{s}} \vdash e_s : \tau_s &
%     \coctx{\Gamma, x\!:\!\tau_1}{\cclrd{r}} \vdash e_r : \tau_r &
%     \coctx{\Gamma, x\!:\!\tau_1}{\cclrd{t}} \vdash e_t : \tau_t }
%   { \coctx{\Gamma}{
%       (\cclrd{t}\;\cpar\;(\cclrd{r}\,\cseq\,\cclrd{t})) 
%       \;\cpar\;
%       (  \cclrd{s}
%          \,\cseq\, 
%          (\cclrd{t}\;\cpar\;(\cclrd{r}\,\cseq\,\cclrd{t})) )
%     } 
%     \vdash \kvd{glet}~x_s=e_s~\kvd{in}~(\kvd{glet}~x_r=e_r~\kvd{in}~e_t) : \tau_t }
% \end{equation*}


% \begin{equation*}
% \begin{array}{l}
%   \cclrd{t}
%   \;\cpar\;
%   (  (\cclrd{r}\;\cpar\;(\cclrd{s}\,\cseq\,\cclrd{r}))
%      \,\cseq\, 
%      \cclrd{t} ) = 
% \\
%   \cclrd{t}
%   \;\cpar\;
%   (\cclrd{r}\,\cseq\, \cclrd{t})
%   \;\cpar\;
%   (\cclrd{s}\,\cseq\,\cclrd{r}\,\cseq\, \cclrd{t}) = 
% \\
%   \cclrd{s}\,\cseq\,\cclrd{r}\,\cseq\, \cclrd{t} 
% \end{array}
% \end{equation*}

% \begin{equation*}
% \begin{array}{l}
%   (\cclrd{t}\;\cpar\;(\cclrd{r}\,\cseq\,\cclrd{t})) 
%   \;\cpar\;
%   (  \cclrd{s}
%      \,\cseq\, 
%      (\cclrd{t}\;\cpar\;(\cclrd{r}\,\cseq\,\cclrd{t})) ) = 
% \\     
%   \cclrd{t}\;\cpar\;(\cclrd{r}\,\cseq\,\cclrd{t})
%   \;\cpar\;
%   (\cclrd{s} \,\cseq\, \cclrd{t})
%   \;\cpar\;
%   (\cclrd{s} \,\cseq\, \cclrd{r}\,\cseq\,\cclrd{t})  = 
% \\     
%   \cclrd{s} \,\cseq\, \cclrd{r}\,\cseq\,\cclrd{t}
% \end{array}
% \end{equation*}

% require

% \begin{equation*}
% \cclrd{r}\;\cpar\;(\cclrd{r} \,\cseq\, \cclrd{s}) = \cclrd{r} \,\cseq\, \cclrd{s}
% \end{equation*}
