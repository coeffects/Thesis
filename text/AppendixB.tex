%!TEX root = ../main.tex

\chapter{Appendix B} 
\label{ch:appendix} 

This appendinx provides additional details for some of the proofs for equational theory
of flat coeffect calculus from Chapter~\ref{ch:flat} and structural coeffect calculus
from Chapter~\ref{ch:structural}.



% =================================================================================================
%
%      #####   ##            #    
%      #        #            #    
%      #        #     ###   ####  
%      ####     #        #   #    
%      #        #     ####   #    
%      #        #    #   #   #  # 
%      #       ###    ####    ##  
%                               
% =================================================================================================

\section{Substitution for flat coeffects}
\label{sec:appendix-flat-cbn}
In Section~\ref{sec:flat-syntax-cbn}, we stated that, for a bottom-pointed flat coeffect
algebra (\ie~$\forall r \in \C \;.\; r \,\cgeq\, \cunit $), the call-by-name substitution 
preserves type if all operators of the flat coeffect algebra coincide (Lemma~\ref{thm:cbn-substitution-bot}).
This section provides the corresponding proof.

\begin{lemma*}[Bottom-pointed substitution]
In a bottom-pointed flat coeffect calculus with an algebra $(\C, \cseq, \cpar, \czip, \cunit, \czero, \cleq)$ 
where $\czip = \cseq = \cpar$ and the operation is also idempotent and commutative and
$\cclrd{r}\,\cleq\,\cclrd{r'} \Rightarrow \forall\cclrd{s}.\cclrd{r}\,\cseq\,\cclrd{s}\;\cleq\;\cclrd{r'}\,\cseq\,\cclrd{s}$ then:
%
\begin{equation*}
\begin{array}{l}
 \coctx{\Gamma}{\cclrd{S}} \vdash e_s : \tau_s \;\; \wedge \;\; 
 \coctx{\Gamma_1,  x : \tau_s, \Gamma_2}{ \cclrd{r}  } \vdash e_r : \tau_r\\
\quad \Rightarrow \;\; \coctx{\Gamma_1,\Gamma,\Gamma_2}{ \cclrd{r} \,\cseq\, \cclrd{S} } \vdash \subst{e_r}{x}{e_s} : \tau_r
\end{array}
\end{equation*}

\end{lemma*}
\begin{proof}
Assume that $\coctx{\Gamma}{\cclrd{S}} \vdash e_s : \tau_s$ and we are substituting a term $e_s$ for 
a variable $x$. Note that we use upper-case $\cclrd{S}$ to distinguish the coeffect of the expression
that is being substituted into an expression. Using structural induction over $\vdash$:

\paragraph{(var)} Given the following derivation using (\emph{var}):
\[
\inference
  { }
  {\coctx{\Gamma_1, y:\tau, \Gamma_2}{\cunit} \vdash y : \tau }
\]
There are two cases depending on whether $y$ is the variable $x$ or not:
\begin{itemize}
\item If $y=x$ then also $\tau = \tau_s$ and thus $\subst{y}{x}{e_s} = e_s$. Using the assumption,
implicit weakening and the fact that $\cunit$ is a unit of $\cseq$:
\[
\inference
 {\inference 
  {\coctx{\Gamma}{\cclrd{S}} \vdash \subst{y}{x}{e_s} : \tau_s }
  {\coctx{\Gamma_1, \Gamma, \Gamma_2}{\cclrd{S}} \vdash \subst{y}{x}{e_s} : \tau } }
 {\coctx{\Gamma_1, \Gamma, \Gamma_2}{\cunit\,\cseq\,\cclrd{S}} \vdash \subst{y}{x}{e_s} : \tau } 
\]
\item If $y\neq x$ then $\subst{y}{x}{e_s} = y$. Using the fact that $\cunit$ is the bottom element
and sub-coeffecting:
\[
\inference
  {\coctx{\Gamma_1, y:\tau, \Gamma_2}{\cunit} \vdash y : \tau }
  {\coctx{\Gamma_1, y:\tau, \Gamma_2}{\cunit\,\cseq\,\cclrd{S}} \vdash y : \tau }
\]
\end{itemize}

\paragraph{(const)} Similar to the (\emph{var}) case when the variable is not substituted.

\paragraph{(sub)} Given the following typing derivation using (\emph{sub}):
\[
\inference
  {\coctx{\Gamma_1,x:\tau_s,\Gamma_2}{\cclrd{r'}} \vdash e : \tau }
  {\coctx{\Gamma_1,x:\tau_s,\Gamma_2}{\cclrd{r}} \vdash e : \tau }\quad\quad(\cclrd{r'} \cleq \cclrd{r})
\]
From the induction hypothesis, we have that
$\coctx{\Gamma_1,\Gamma,\Gamma_2}{\cclrd{r'}\,\cseq\,\cclrd{S}} \vdash \subst{e}{x}{e_s} : \tau$.
The condition on $\cleq$ means that $\cclrd{r'}\,\cseq\,\cclrd{S}\;\cleq\;\cclrd{r}\,\cseq\,\cclrd{S}$
and so we can apply the (\emph{sub}) rule to obtain
$\coctx{\Gamma_1,\Gamma,\Gamma_2}{\cclrd{r}\,\cseq\,\cclrd{S}} \vdash \subst{e}{x}{e_s} : \tau$.

\paragraph{(abs)} Given the following typing derivation using (\emph{abs}):
\[
\inference
  {\coctx{\Gamma_1,x:\tau_s,\Gamma_2, y:\tau_1}{\cclrd{r}\;\czip\;\cclrd{s}} \vdash e : \tau_2}
  {\coctx{\Gamma_1,x:\tau_s,\Gamma_2}{\cclrd{r}} \vdash \lambda y.e : \tau_1 \xrightarrow{\cclrd{s}} \tau_2 }
\]
Assume w.l.o.g. that $x\neq y$. From the induction hypothesis, we have that:
\[
\coctx{\Gamma_1,\Gamma,\Gamma_2, y:\tau_1}{ \cclrd{r} \,\cseq\, \cclrd{s} } \vdash \subst{e}{x}{e_s} : \tau_2
\]
Now using the fact that $\czip\,=\,\cseq$, associativity and commutativity and (\emph{abs}):
\[
\inference
 {\inference
  {\coctx{\Gamma_1,\Gamma,\Gamma_2, y:\tau_1}{ (\cclrd{r} \,\czip\, \cclrd{s})\,\cseq\,\cclrd{S} } \vdash \subst{e}{x}{e_s} : \tau_2}
  {\coctx{\Gamma_1,\Gamma,\Gamma_2, y:\tau_1}{ (\cclrd{r} \,\cseq\, \cclrd{S})\,\czip\,\cclrd{s} } \vdash \subst{e}{x}{e_s} : \tau_2}}
 {\inference
  {\coctx{\Gamma_1,\Gamma,\Gamma_2}{\cclrd{r} \,\cseq\, \cclrd{S}} \vdash \lambda y.(\subst{e}{x}{e_s}) : \tau_1 \xrightarrow{\cclrd{s}} \tau_2}
  {\coctx{\Gamma_1,\Gamma,\Gamma_2}{\cclrd{r} \,\cseq\, \cclrd{S}} \vdash \subst{(\lambda y.e)}{x}{e_s} : \tau_1 \xrightarrow{\cclrd{s}} \tau_2} }
\]

\paragraph{(app)} Given the following typing derivation using (\emph{app}):
\[
\inference
  {\coctx{\Gamma_1,x:\tau_s,\Gamma_2}{\cclrd{r}} \vdash e_1 : \tau_1 \xrightarrow{\cclrd{t}} \tau_2 &
   \coctx{\Gamma_1,x:\tau_s,\Gamma_2}{\cclrd{s}} \vdash e_2 : \tau_1 }
  {\coctx{\Gamma_1,x:\tau_s,\Gamma_2}{\cclrd{r} \;\cpar\; (\cclrd{s} \,\cseq\, \cclrd{t})} \vdash e_1~e_2 : \tau_2}
\]
From the induction hypothesis, we have that:
\[
\begin{array}{l}
 \coctx{\Gamma_1, \Gamma, \Gamma_2}{\cclrd{r}\,\cseq\,\cclrd{S}} \vdash \subst{e_1}{x}{e_s} : \tau_1 \xrightarrow{\cclrd{t}} \tau_2 \\
 \coctx{\Gamma_1, \Gamma, \Gamma_2}{\cclrd{s}\,\cseq\,\cclrd{S}} \vdash \subst{e_2}{x}{e_s} : \tau_1
\end{array}\quad(*)
\]
Now using the (\emph{app}) rule and the fact that $\cpar\,=\,\cseq$, associativity, commutativity and idempotence
(note that all three properties are needed):
\[
\inference
 { (*) }
 {\inference
   { \coctx{\Gamma_1,\Gamma,\Gamma_2}{(\cclrd{r}\,\cseq\,\cclrd{S}) \;\cpar\; ((\cclrd{s}\,\cseq\,\cclrd{S}) \,\cseq\, \cclrd{t})} 
       \vdash \subst{e_1}{x}{e_s}~\subst{e_2}{x}{e_s} : \tau_2}
   { \coctx{\Gamma_1,\Gamma,\Gamma_2}{(\cclrd{r} \;\cpar\; (\cclrd{s} \,\cseq\, \cclrd{t}))\;\cseq\;\cclrd{S}} 
       \vdash \subst{(e_1~e_2)}{x}{e_s} : \tau_2} }
\]

\paragraph{(let)} Given the following typing derivation using (\emph{let}):
\[
\inference
  { \coctx{\Gamma_1,x:\tau_s,\Gamma_2}{\cclrd{r}} \vdash e_1 : \tau_1 &
    \coctx{\Gamma_1,x:\tau_s,\Gamma_2, y:\tau_1}{\cclrd{s}} \vdash e_2 : \tau_2}
  {\coctx{\Gamma_1,x:\tau_s,\Gamma_2}{\cclrd{s} \;\cpar\; (\cclrd{s} \,\cseq\, \cclrd{r})} \vdash \kvd{let}~y=e_1~\kvd{in}~e_2 : \tau_2 }
\]
From the induction hypothesis, we have that:
\[
\begin{array}{l}
 \coctx{\Gamma_1, \Gamma, \Gamma_2}{\cclrd{r}\,\cseq\,\cclrd{S}} \vdash \subst{e_1}{x}{e_s} : \tau_1 \\
 \coctx{\Gamma_1, \Gamma, \Gamma_2, y:\tau_1}{\cclrd{s}\,\cseq\,\cclrd{S}} \vdash \subst{e_2}{x}{e_s} : \tau_2
\end{array}\quad(\dagger)
\]
Now using the (\emph{let}) rule and similarly to the (\emph{app}) case:
\[
\hspace{-2em}
\inference
 { (\dagger) }
 { \inference
   { \coctx{\Gamma_1, \Gamma, \Gamma_2}{(\cclrd{s}\,\cseq\,\cclrd{S}) \;\cpar\; ((\cclrd{s}\,\cseq\,\cclrd{S}) \,\cseq\, (\cclrd{r}\,\cseq\,\cclrd{S}))} 
         \vdash \kvd{let}~y=\subst{e_1}{x}{e_s}~\kvd{in}~\subst{e_2}{x}{e_s} : \tau_2 } 
   { \coctx{\Gamma_1, \Gamma, \Gamma_2}{(\cclrd{s} \;\cpar\; (\cclrd{s} \,\cseq\, \cclrd{r})) \,\cseq\, \cclrd{S}} 
         \vdash \subst{(\kvd{let}~y=e_1~\kvd{in}~e_2)}{x}{e_s} : \tau_2 } }
\]
\end{proof}
\newpage



% =================================================================================================
%                                                                      
%     ###    #                           #                           ##   
%    #   #   #                           #                            #   
%    #      ####   # ##   #   #   ###   ####   #   #  # ##    ###     #   
%     ###    #     ##  #  #   #  #   #   #     #   #  ##  #      #    #   
%        #   #     #      #   #  #       #     #   #  #       ####    #   
%    #   #   #  #  #      #  ##  #   #   #  #  #  ##  #      #   #    #   
%     ###     ##   #       ## #   ###     ##    ## #  #       ####   ###  
%
% =================================================================================================


\section{Substitution for structural coeffects}
\label{sec:appendix-struct-cbn}

In order to prove that $\beta$-reduction and $\eta$-expansion preserve the type of an expression
in Section~\ref{sec:struct-syntactic-subst}, we required a multi-nary form of the substitution 
lemma (Lemma~\ref{thm:structural-substitution}). This section provides the corresponding proof.

\begin{lemma*}[Multi-nary substitution]
Given an expression with multiple holes filled by variables $x_{Ti}\!:\!\tau_{Ti}$ with coeffects $\cclrd{s_k}$:
%
\begin{equation*}
\coctx{\Gamma}{\aclrd{\textbf{r}}}~[\,\coctx{x_{T1}\!:\!\tau_{T1}}{\alift{\cclrd{s_1}}} \,|\, \ldots \,|\,
  \coctx{x_{Tk}\!:\!\tau_{Tk}}{\alift{\cclrd{s_k}}}\,] \vdash e_r : \tau_r
\end{equation*}
%
and a expressions $e_{Ti}$ with free-variable contexts $\Gamma_{Ti}$ annotated with $\aclrd{\mathbf{T_i}}$:
%
\begin{equation*}
\coctx{\Gamma_1}{\aclrd{\mathbf{T_1}}} \vdash e_{T1} : \tau_{T1}
\quad \ldots \quad
\coctx{\Gamma_k}{\aclrd{\mathbf{T_k}}} \vdash e_{Tk} : \tau_{Tk}
\end{equation*}
%
then substituting the expressions $e_{Ti}$ for variables $x_{Ti}$ results in an expression with a context
where the original holes are filled by contexts $\Gamma_{Ti}$ with coeffects $\cclrd{s_i} \aseq \aclrd{\mathbf{T_i}}$:
%
\begin{equation*}
\coctx{\Gamma}{\aclrd{\textbf{r}}}~[\,\coctx{\Gamma_{T1}}{\cclrd{s_1}\aseq\,\aclrd{\textbf{T}_1}} \,|\, \ldots \,|\, 
  \coctx{\Gamma_{Tk}}{\cclrd{s_k}\aseq\,\aclrd{\textbf{T}_k}}\,] \vdash \subst{\subst{e_r}{x_{T1}}{e_{T1}}\ldots}{x_{Tk}}{e_{Tk}} : \tau_r
\end{equation*}
\end{lemma*}
\begin{proof}
Assume that $\coctx{\Gamma}{\aclrd{\mathbf{T_i}}} \vdash e_{Ti} : \tau_{Ti}$ and we are substituting terms $e_{Ti}$ for 
variables $x_{Ti}$. Furthermore, we assume that are variables that are being substituted for actually appear in the original 
term (which means that $k$ is at most the number of variables). Note that we use upper-case $\aclrd{\textbf{T}}$ to distinguish 
the coeffect of the expression that is being substituted into an expression. Using structural induction over $\vdash$:


% -------------------------------------------------------------------------------------------------

\paragraph{Syntax-driven typing rules}
\paragraph{(var)} Given the following derivation using (\emph{var}):
\[
\inference
  { }
  {\coctx{y:\tau}{\alift{\cunit}} \vdash y : \tau }
\]
Here, the context contains exactly one variable and so $k=1$. There are two cases depending on whether $y$ is 
the (only substituted) variables $x_{T1}$ or not:
\begin{itemize}
\item If $y=x_{T1}$ then also $\tau = \tau_{T1}$ and thus $\subst{y}{x_{T1}}{e_{T_1}} = e_{T1}$. 
In this case, the context contains only a single hole $\coctx{\Gamma}{\aclrd{\mathbf{r}}} = \coctx{-}{-}$. Using the
assumption and the fact that $\cunit$ is a unit of $\cseq$:
\[
\inference
 { \inference 
   {\coctx{\Gamma_{T1}}{\aclrd{\mathbf{T_i}}} \vdash \subst{y}{x_{T1}}{e_{T1}} : \tau_r }
   {\coctx{\Gamma}{\aclrd{\mathbf{r}}}~[\,\coctx{\Gamma_{T1}}{\aclrd{\mathbf{T_i}}}\,] \vdash \subst{y}{x_{T1}}{e_{T1}} : \tau_r } }
 { \coctx{\Gamma}{\aclrd{\mathbf{r}}}~[\,\coctx{\Gamma_{T1}}{\cunit\,\aseq\,\aclrd{\mathbf{T_i}}}\,] \vdash \subst{y}{x_{T1}}{e_{T1}} : \tau_r }   
\]
\item If $y\neq x_{T1}$ then there is no substitution that could be performed ($y$ does not 
appear in the context) and so (trivially):
\[
\coctx{\Gamma}{\aclrd{\mathbf{r}}}~[\,] \vdash y : \tau_r
\]
\end{itemize}

\paragraph{(const)} Similar to the (\emph{var}) case when the variable is not substituted.

\paragraph{(app)} In the (\emph{app}) rule, the context $\Gamma$ is obtained as a tensor product $\Gamma_1, \Gamma_2$.
  Given $\Gamma$ of length $k$, we assume that $\Gamma_1$ has length $l$ and $\Gamma_2$ has length $k-l$. 
  Now, given the following typing derivation using (\emph{app}):
\[
\inference
  {\coctx{\Gamma_1}{\aclrd{\mathbf{s_1}}} 
   [\,\coctx{x_{T1}}{\alift{\cclrd{s_1}}} \,|\, \ldots \,|\, \coctx{x_{Tl}}{\alift{\cclrd{s_l}}}\,] 
      \vdash e_1 : \tau_1 \xrightarrow{\cclrd{t}} \tau_2 \\
   \coctx{\Gamma_2}{\aclrd{\mathbf{s_2}}} 
   [\,\coctx{x_{Tl+1}}{\alift{\cclrd{s_{l+1}}}} \,|\, \ldots \,|\, \coctx{x_{Tk}}{\alift{\cclrd{s_k}}}\,]
      \vdash e_2 : \tau_1 }
  { \begin{array}{l}
    \coctx{\Gamma_1, \Gamma_2}{\aclrd{\mathbf{s_1}\;\atimes\;(\cclrd{t} \,\aseq\, \aclrd{\mathbf{s_2}})}}~
     [\,\coctx{x_{T1}}{\alift{\cclrd{s_1}}} \,|\, \ldots \,|\, \coctx{x_{Tl}}{\alift{\cclrd{s_l}}}, \\[-0.35em] \hspace{9.3em}
       \coctx{x_{Tl+1}}{\alift{\cclrd{t} \,\cseq\, \cclrd{s_{l+1}}}} \,|\, \ldots \,|\, 
          \coctx{x_{Tk}}{\alift{\cclrd{t} \,\cseq\, \cclrd{s_k}}}\,] \vdash e_1~e_2 : \tau_2 
    \end{array} }
\]
Here, we use $\cclrd{t}\;\aseq\;\aclrd{\mathbf{s_2}}$ as a pointwise extension of $\czip$ that is additionally
applied to holes such that $\cclrd{t}\;\czip\; - = -$. That is, a hole in the context remains a hole and so
the second part of the context is filled with $\cclrd{t} \,\cseq\, \cclrd{s_i}$ for all $i \in \{ l+1 \ldots k \}$.
Next, from the induction hypothesis, we have that:
\[
\begin{array}{l}
 \coctx{\Gamma_1}{\aclrd{\mathbf{s_1}}} 
   [\,\coctx{\Gamma_{T1}}{\cclrd{s_1}\,\aseq\,\aclrd{\mathbf{T_1}}} \,|\, \ldots \,|\, \coctx{\Gamma_{Tl}}{\cclrd{s_l}\aseq\,\aclrd{\mathbf{T_l}}}\,] 
   \vdash e_1~[\,\ldots\,] : \tau_1 \xrightarrow{\cclrd{t}} \tau_2 \\
 \coctx{\Gamma_2}{\aclrd{\mathbf{s_{2}}}} 
   [\,\coctx{\Gamma_{Tl+1}}{\cclrd{s_{l+1}}\aseq\,\aclrd{\mathbf{T}_{l+1}}} \,|\, \ldots \,|\, \coctx{\Gamma_{Tk}}{\cclrd{s_k}\aseq\,\aclrd{\mathbf{T_k}}}\,]
   \vdash e_2~[\,\ldots\,] : \tau_1
\end{array}\quad(*)
\]
Now using the (\emph{app}) rule in the first step and associativity of $\cseq$ on the second part of the context in the second step:
\[
\hspace{-2em}
\inference 
 { \inference
   { (*) }
   { \begin{array}{l}
      \coctx{\Gamma_1, \Gamma_2}{\aclrd{\mathbf{s_1}\;\atimes\;(\cclrd{t} \,\aseq\, \aclrd{\mathbf{s_2}})}} ~~
       [\,\coctx{\Gamma_{T1}}{ \cclrd{s_1}\,\aseq\,\aclrd{\mathbf{T_1}} } \,|\, \ldots \,|\, 
          \coctx{\Gamma_{Tl}}{ \cclrd{s_l}\,\aseq\,\aclrd{\mathbf{T_l}} }, 
         \\[-0.35em] \qquad
         \coctx{\Gamma_{Tl+1}}{\cclrd{t} \,\aseq\, (\cclrd{s_{l+1}}\,\aseq\,\aclrd{\mathbf{T_{l+1}}}) } \,|\, \ldots \,|\,
         \coctx{\Gamma_{Tk}}{\cclrd{t} \,\aseq\, (\cclrd{s_k}\,\aseq\,\aclrd{\mathbf{T_k}})}\,] \vdash (e_1~e_2)~[\,\ldots\,] : \tau_2 
      \end{array} } }
 { \begin{array}{l}
    \coctx{\Gamma_1, \Gamma_2}{\aclrd{\mathbf{s_1}\;\atimes\;(\cclrd{t} \,\aseq\, \aclrd{\mathbf{s_2}})}} ~~
     [\,\coctx{\Gamma_{T1}}{ \cclrd{s_1}\,\aseq\,\aclrd{\mathbf{T_1}} } \,|\, \ldots \,|\, 
        \coctx{\Gamma_{Tl}}{ \cclrd{s_l}\,\aseq\,\aclrd{\mathbf{T_l}} }, 
       \\[-0.35em] \qquad
       \coctx{\Gamma_{Tl+1}}{(\cclrd{t} \,\cseq\, \cclrd{s_{l+1}})\,\aseq\,\aclrd{\mathbf{T_{l+1}}} } \,|\, \ldots \,|\,
       \coctx{\Gamma_{Tk}}{(\cclrd{t} \,\cseq\, \cclrd{s_k})\,\aseq\,\aclrd{\mathbf{T_k}}}\,] \vdash (e_1~e_2)~[\,\ldots\,] : \tau_2 
    \end{array} }
\]

\paragraph{(abs)} Without the loss of generality, we can assume that the bound variable is 
not one of the variables being substituted. Thus, the last variable (and the corresponding coeffect)
are not holes. The typing derivation using (\emph{abs}) then looks as follows:
\[
\inference
  {\coctx{\Gamma, x\!:\!\tau_1}{\aclrd{\textbf{r}}\,\atimes\,\alift{\cclrd{s}}} 
   ~[\,\coctx{x_{T1}\!:\!\tau_{T1}}{\alift{\cclrd{s_1}}} \,|\, \ldots \,|\, \coctx{x_{Tk}\!:\!\tau_{Tk}}{\alift{\cclrd{s_k}}}\,]
    \vdash e : \tau_2}
  {\coctx{\Gamma}{\aclrd{\textbf{r}}} 
   ~[\,\coctx{x_{T1}\!:\!\tau_{T1}}{\alift{\cclrd{s_1}}} \,|\, \ldots \,|\, \coctx{x_{Tk}\!:\!\tau_{Tk}}{\alift{\cclrd{s_k}}}\,]
    \vdash \lambda x.e : \tau_1 \xrightarrow{\cclrd{s}} \tau_2 }
\]
From the induction hypothesis, we have that:
\[
\coctx{\Gamma, x\!:\!\tau_1}{\aclrd{\textbf{r}}\,\atimes\,\alift{\cclrd{s}}}~
 [\,\coctx{\Gamma_{T1}}{\cclrd{s_1}\aseq\,\aclrd{\textbf{T}_1}} \,|\, \ldots \,|\, 
    \coctx{\Gamma_{Tk}}{\cclrd{s_k}\aseq\,\aclrd{\textbf{T}_k}}\,] \vdash e~[\,\ldots\,] : \tau_2
\]
Because the last position in the vector of variables is an actual variable rather than a hole, we
just need to apply the (\emph{abs}) rule:
\[
\inference 
 { \coctx{\Gamma, x\!:\!\tau_1}{\aclrd{\textbf{r}}\,\atimes\,\alift{\cclrd{s}}}~
   [\,\coctx{\Gamma_{T1}}{\cclrd{s_1}\aseq\,\aclrd{\textbf{T}_1}} \,|\, \ldots \,|\, 
      \coctx{\Gamma_{Tk}}{\cclrd{s_k}\aseq\,\aclrd{\textbf{T}_k}}\,] \vdash e~[\,\ldots\,] : \tau_2 }
 { \coctx{\Gamma}{\aclrd{\textbf{r}}}~
   [\,\coctx{\Gamma_{T1}}{\cclrd{s_1}\aseq\,\aclrd{\textbf{T}_1}} \,|\, \ldots \,|\, 
      \coctx{\Gamma_{Tk}}{\cclrd{s_k}\aseq\,\aclrd{\textbf{T}_k}}\,] \vdash \lambda x.e~[\,\ldots\,] : \tau_1 \xrightarrow{\cclrd{s}} \tau_2 }
\]

\paragraph{(let)} In the structural coeffect calculus let-binding can be viewed as a syntactic
sugar for abstraction/application and so the case follows from (\emph{abs}) and (\emph{app}).

\vspace{0.75em}
\noindent
Compared to the similar proof for the flat coeffect calculus, the proof for the structural system
requires fewer properties of the coeffect algebra. In particular, we only needed associativity of 
$\cseq$ in the (\emph{app}) rule the fact that $\cunit$ is a unit of $\cseq$ in (\emph{var}).


% -------------------------------------------------------------------------------------------------

\paragraph{Structural rules}

\paragraph{(contr)} In case of contraction, we can assume that the two variables to be contracted
(in the assumption) are not the variables that are being substituted. However, the resulting
variable  (in the conclusion) can be one of the variables being substituted for.

\begin{itemize}
\item Assuming that $x \neq x_{Ti}$ for all $i$, the original derivation is:
\[
\inference
  {{ \begin{array}{l}
     \coctx{\Gamma_1,y\!:\!\tau_1,z\!:\!\tau_1,\Gamma_2}{\aclrd{\textbf{r}} \atimes \alift{\cclrd{s},\cclrd{t}} \atimes \aclrd{\textbf{q}}} 
     \\[-0.35em] \hspace{4em}
     [\,\coctx{x_{T1}\!:\!\tau_{T1}}{\alift{\cclrd{s_1}}} \,|\, \ldots \,|\, \coctx{x_{Tk}\!:\!\tau_{Tk}}{\alift{\cclrd{s_k}}}\,]
    \vdash e : \tau
    \end{array} }}
  { \begin{array}{l}
    \coctx{\Gamma_1,x\!:\!\tau_1,\Gamma_2}{\aclrd{\textbf{r}} \atimes \alift{\cclrd{s} \,\cpar\, \cclrd{t}} \atimes \aclrd{\textbf{q}}} 
    \\[-0.35em] \quad
    [\,\coctx{x_{T1}\!:\!\tau_{T1}}{\alift{\cclrd{s_1}}} \,|\, \ldots \,|\, \coctx{x_{Tk}\!:\!\tau_{Tk}}{\alift{\cclrd{s_k}}}\,]
    \vdash \subst{e}{z,y}{x} : \tau
    \end{array} }
\]
Applying (\emph{contr}) to the induction hypothesis gives the required result:
\[
\inference
  {{ \begin{array}{l}
     \coctx{\Gamma_1,y\!:\!\tau_1,z\!:\!\tau_1,\Gamma_2}{\aclrd{\textbf{r}} \atimes \alift{\cclrd{s},\cclrd{t}} \atimes \aclrd{\textbf{q}}} 
     \\[-0.35em] \hspace{4em}
     [\,\coctx{\Gamma_{T1}}{\cclrd{s_1}\aseq\,\aclrd{\textbf{T}_1}} \,|\, \ldots \,|\, \coctx{\Gamma_{Tk}}{\cclrd{s_k}\aseq\,\aclrd{\textbf{T}_k}}\,]
    \vdash e~[\,\ldots\,] : \tau
    \end{array} }}
  { \begin{array}{l}
    \coctx{\Gamma_1,x\!:\!\tau_1,\Gamma_2}{\aclrd{\textbf{r}} \atimes \alift{\cclrd{s} \,\cpar\, \cclrd{t}} \atimes \aclrd{\textbf{q}}} 
    \\[-0.35em] \quad
    [\,\coctx{\Gamma_{T1}}{\cclrd{s_1}\aseq\,\aclrd{\textbf{T}_1}} \,|\, \ldots \,|\, \coctx{\Gamma_{Tk}}{\cclrd{s_k}\aseq\,\aclrd{\textbf{T}_k}}\,]
    \vdash \subst{e}{z,y}{x}[\,\ldots\,] : \tau
    \end{array} }
\]

\item In the other case, $x=x_{Ti}$ for some $i$. The original typing is:
\[
\inference
  {{ \begin{array}{l}
     \coctx{\Gamma_1,-,-,\Gamma_2}{\aclrd{\textbf{r}} \atimes \alift{-,-} \atimes \aclrd{\textbf{q}}}~
         [\,\coctx{x_{T1}\!:\!\tau_{T1}}{\alift{\cclrd{s_1}}} \,|\, \ldots \,|\,
     \\[-0.35em] \hspace{4em}
     \coctx{y\!:\!\tau_1}{\alift{\cclrd{s}}} \,|\,
     \coctx{z\!:\!\tau_1}{\alift{\cclrd{t}}}
     \,|\, \ldots \,|\, \coctx{x_{Tk}\!:\!\tau_{Tk}}{\alift{\cclrd{s_k}}}\,]
    \vdash e : \tau
    \end{array} }}
  { \begin{array}{l}
    \coctx{\Gamma_1,-,\Gamma_2}{\aclrd{\textbf{r}} \atimes \alift{-} \atimes \aclrd{\textbf{q}}}~
        [\,\coctx{x_{T1}\!:\!\tau_{T1}}{\alift{\cclrd{s_1}}} \,|\, \ldots \,|\,
     \\[-0.35em] \quad
     \coctx{x\!:\!\tau_1}{ \alift{\cclrd{s} \,\cpar\, \cclrd{t}} }
     \,|\, \ldots \,|\, \coctx{x_{Tk}\!:\!\tau_{Tk}}{\alift{\cclrd{s_k}}}\,]
    \vdash \subst{e}{z,y}{x} : \tau
    \end{array} }
\]
Applying (\emph{contr}) to the induction hypothesis gives the following result:
\[
\inference
  {{ \begin{array}{l}
     \coctx{\Gamma_1,-,-,\Gamma_2}{\aclrd{\textbf{r}} \atimes \alift{-,-} \atimes \aclrd{\textbf{q}}}~
         [\,\coctx{\Gamma_{T1}}{\cclrd{s_1}\aseq\,\aclrd{\textbf{T}_1}} \,|\, \ldots \,|\,
     \\[-0.35em] \quad
     \coctx{\Gamma_{Ti}}{\cclrd{s}\,\aseq\,\aclrd{\mathbf{T_k}} } \,|\,
     \coctx{\Gamma_{Ti}}{\cclrd{t}\,\aseq\,\aclrd{\mathbf{T_k}} }
     \,|\, \ldots \,|\, \coctx{\Gamma_{Tk}}{\cclrd{s_k}\aseq\,\aclrd{\textbf{T}_k}}\,]
    \vdash e~[\,\ldots\,] : \tau
    \end{array} }}
  {{ \begin{array}{l}
     \coctx{\Gamma_1,-,\Gamma_2}{\aclrd{\textbf{r}} \atimes \alift{-} \atimes \aclrd{\textbf{q}}}~
         [\,\coctx{\Gamma_{T1}}{\cclrd{s_1}\aseq\,\aclrd{\textbf{T}_1}} \,|\, \ldots \,|\,
     \\[-0.35em] \quad
     \coctx{\Gamma_{Ti}}{ (\cclrd{s}\,\aseq\,\aclrd{\mathbf{T_k}}) \aparstr (\cclrd{t}\,\aseq\,\aclrd{\mathbf{T_k}})}
     \,|\, \ldots \,|\, \coctx{\Gamma_{Tk}}{\cclrd{s_k}\aseq\,\aclrd{\textbf{T}_k}}\,]
    \vdash e~[\,\ldots\,] : \tau
    \end{array} }}
\]
Here the $\aparstr$ operation represents a pointwise extension of $\cpar$. Thus for $i^{\textit{th}}$ 
substituted coeffect, we have $(\cclrd{s}\,\cseq\,\cclrd{T_i})\;\cpar\;(\cclrd{t}\,\cseq\,\cclrd{T_i})$.
Using the distributivity law of structural coeffect algebra, we obtain the required structure:
$(\cclrd{s}\,\cpar\,\cclrd{t})\;\cseq\;\cclrd{T_i}$.

\paragraph{(sub)} As in the (\emph{contr}) case, in the (\emph{sub}) case we distinguish two situations.
If the sub-coeffecting is applied to variable that is \emph{not} being substituted for, then the case is
easy (sub-coeffecting does not interact with substitution), so we only consider the case when $x$ is one
of the variables being substituted for:
\[
\inference
  {{\begin{array}{l}
   \coctx{\Gamma_1,-,\Gamma_2}{\aclrd{\textbf{r}} \atimes \alift{-} \atimes \aclrd{\textbf{q}}} ~
   [\,\coctx{x_{T1}\!:\!\tau_{T1}}{\alift{\cclrd{s_1}}} \,|\, \ldots \,|\,
     \\[-0.35em] \qquad
     \coctx{x\!:\!\tau_1}{\alift{\cclrd{s'}}}
     \,|\, \ldots \,|\, \coctx{x_{Tk}\!:\!\tau_{Tk}}{\alift{\cclrd{s_k}}}\,]
   \vdash e : \tau
   \end{array}}}
  {{\begin{array}{l}
   \coctx{\Gamma_1,-,\Gamma_2}{\aclrd{\textbf{r}} \atimes \alift{-} \atimes \aclrd{\textbf{q}}} ~
   [\,\coctx{x_{T1}\!:\!\tau_{T1}}{\alift{\cclrd{s_1}}} \,|\, \ldots \,|\,
     \\[-0.35em] \qquad
     \coctx{x\!:\!\tau_1}{\alift{\cclrd{s}}}
     \,|\, \ldots \,|\, \coctx{x_{Tk}\!:\!\tau_{Tk}}{\alift{\cclrd{s_k}}}\,]
   \vdash e : \tau
   \end{array}}}
\quad(\cclrd{s'}\;\cleq\;\cclrd{s})   
\]
From the induction hypothesis, we have the following:
\[
  {{\begin{array}{l}
   \coctx{\Gamma_1,-,\Gamma_2}{\aclrd{\textbf{r}} \atimes \alift{-} \atimes \aclrd{\textbf{q}}} ~
   [\,\coctx{\Gamma_{T1}}{\cclrd{s_1}\aseq\,\aclrd{\textbf{T}_1}} \,|\, \ldots \,|\,
     \\[-0.35em] \qquad
     \coctx{\Gamma_{Ti}}{\cclrd{s'}\aseq\,\aclrd{\mathbf{T_k}} }
     \,|\, \ldots \,|\, \coctx{\Gamma_{Tk}}{\cclrd{s_k}\aseq\,\aclrd{\textbf{T}_k}}\,]
   \vdash e : \tau
   \end{array}}}
\]
To complete the case, we need to apply (\emph{sub}) repeatedly on each of the variables in 
$\Gamma_{Ti}$. For $i^{\textit{th}}$ variable $x_i$, the coeffect annotation is $\cclrd{s'}\cseq\,\cclrd{T_i}$.
Using the fact that combining coeffects with $\cseq$ preserves the ordering, we get that 
$(\cclrd{s'}\cseq\,\cclrd{T_i}) \cleq (\cclrd{s}\,\cseq\,\cclrd{T_i})$ and so the conditions
of (\emph{sub}) are satisfied.

\paragraph{(weak)} We again need to consider whether the removed variable is 
one of the variables that are being substituted for. If this is not the case,
the proof is easy, so we only look at the other case:
\[
\inference
  {\coctx{\Gamma}{ \aclrd{\textbf{r}} }~
   [\,\coctx{x_{T1}\!:\!\tau_{T1}}{\alift{\cclrd{s_1}}} \,|\, \ldots \,|\, \coctx{x_{Tk}\!:\!\tau_{Tk}}{\alift{\cclrd{s_k}}}\,]
   \vdash e : \tau}
  {\coctx{\Gamma, -}{\aclrd{\textbf{r}} \atimes \alift{ - }}~
   [\,\coctx{x_{T1}\!:\!\tau_{T1}}{\alift{\cclrd{s_1}}} \,|\, \ldots \,|\, \coctx{x_{Tk}\!:\!\tau_{Tk}}{\alift{\cclrd{s_k}}}\,|\,\coctx{x\!:\!\tau_1}{\czero}\,]
   \vdash e : \tau} 
\]
Now, we use the induction hypothesis, apply the (\emph{weak}) rule and use properties of the
structural coeffect algebra:
\[
\inference
  {{ \begin{array}{l}
    \coctx{\Gamma}{ \aclrd{\textbf{r}} }~
    [\,\coctx{\Gamma_{T1}}{\cclrd{s_1}\aseq\,\aclrd{\textbf{T}_1}} \,|\, \ldots \,|\, 
     \\[-0.35em] \qquad
     \coctx{\Gamma_{Tk-1}}{\cclrd{s_{k-1}}\aseq\,\aclrd{\textbf{T}_{k-1}}}\,]
   \vdash e~[\,\ldots\,] : \tau 
   \end{array} }}
{\tyrule{sub}
 {{\begin{array}{l}
   \coctx{\Gamma, -}{\aclrd{\textbf{r}} \atimes \alift{ - }}~
   [\,\coctx{\Gamma_{T1}}{\cclrd{s_1}\aseq\,\aclrd{\textbf{T}_1}} \,|\, \ldots \,|\, 
    \\[-0.35em] \qquad
    \coctx{\Gamma_{Tk-1}}{\cclrd{s_{k-1}}\aseq\,\aclrd{\textbf{T}_{k-1}}}
      \,|\,\coctx{x\!:\!\tau_1}{\czero}\,]
   \vdash e~[\,\ldots\,] : \tau
   \end{array} }} 
 {{\begin{array}{l}
   \coctx{\Gamma, -}{\aclrd{\textbf{r}} \atimes \alift{ - }}~
   [\,\coctx{\Gamma_{T1}}{\cclrd{s_1}\aseq\,\aclrd{\textbf{T}_1}} \,|\, \ldots \,|\, 
    \\[-0.35em] \qquad
    \coctx{\Gamma_{Tk-1}}{\cclrd{s_{k-1}}\aseq\,\aclrd{\textbf{T}_{k-1}}}
      \,|\,\coctx{x\!:\!\tau_1}{\czero \aseq\,\aclrd{\textbf{T}_k} }\,]
   \vdash e~[\,\ldots\,] : \tau
   \end{array} }}  }
\]
The derivation first applies the standard (\emph{weak}) rule and then uses sub-coeffecting
rule and the property $\czero\;\cleq\;(\czero\,\cseq\,\cclrd{r})$ to obtain conclusion of the required form.

\paragraph{(exch)} In the (\emph{exch}) case, the property follows directly from the 
induction hypothesis. The required conclusion is obtained by applying (\emph{exch}) 
repeatedly (as we now need to exchange not just two individual variables, but two 
contexts, possibly containing multiple variables).

\end{itemize}
\end{proof}


