\chapter{Conclusions}
\label{ch:conclusions}

Some of the most fundamental academic work is not the one solving hard research problems, but the
one that changes how we understand the world. Some philosophers argue that \emph{language} is the
key for understanding how we think, while in science the dominant thinking is determined by
\emph{paradigms} \cite{philosophy-kuhn} or \emph{research programmes} \cite{philosophy-lakatos}.
In a way, programming languages play a similar role for computer science and software development.

This thesis aims to change how developers and programming language designers think about
\emph{context} or \emph{execution environments} of programs. Such context or execution environment
has numerous forms -- mobile applications access network, GPS location or user's personal data;
journalists obtain information published on the web or through open government data initiatives;
in dataflow programming, we have access to past values of expressions.
This thesis aims to change our understanding of \emph{context} so that the above examples are
viewed uniformly through a single programming language abstraction, which we call \emph{coeffects},
rather than as disjoint cases.

In this chapter, we give a brief overview of the technical contributions of this thesis
(Section~\ref{sec:conc-summary}). The thesis looks at two kinds of context-dependence, identifies
common patterns and captures those using two \emph{coeffect calculi}. It gives formal semantics
using categorical foundations and uses it as a basis for prototype implementation. Finally,
Section~\ref{sec:conc-conclusions} concludes the thesis.


% ==================================================================================================
%
%    #######
%    #     # #    # ###### #####  #    # # ###### #    #
%    #     # #    # #      #    # #    # # #      #    #
%    #     # #    # #####  #    # #    # # #####  #    #
%    #     # #    # #      #####  #    # # #      # ## #
%    #     #  #  #  #      #   #   #  #  # #      ##  ##
%    #######   ##   ###### #    #   ##   # ###### #    #
%
% =================================================================================================

\section{Contributions}
\label{sec:conc-summary}

As observed in Chapter~\ref{ch:intro}, modern computer programs run in rich and diverse environments
or \emph{contexts}. The richness means that environments provides additional resources and
capabilities that may be accessed by the program. The diversity means that programs often need to
run in multiple different environments, such as mobile phones, servers, web browsers or even on the
GPU. In this thesis, we present the foundations for building programming languages that simplify
writing software for such rich and diverse environments.

% --------------------------------------------------------------------------------------------------

\paragraph{Notions of context.}

In $\lambda$-calculus, the term \emph{context} is usually refers to the free-variable context.
However, other programming language features are also related to context or program's
execution environment. In Chapter~\ref{ch:applications}, we revisit many of such features --
including resource tracking in distributed computing, cross-platform development, dataflow
programming and liveness analysis, but also Haskell's implicit parameters.

The main contribution of the chapter is that it presents the disjoint language features in a
unified way. We show type systems and semantics for many of the languages, illuminating
the fact that they are closely related.

Considering applications is one way of approaching the theory of coeffects introduced in this thesis.
Other pathways to coeffects are discussed in Chapter~\ref{ch:pathways}, which looks at theoretical
developments leading to coeffects, including the work on effect systems, comonadic semantics and
linear logics.

% --------------------------------------------------------------------------------------------------
~

\paragraph{Flat coeffect calculus.}

The applications discussed in Chapter~\ref{ch:applications} fall into two categories. In the first
category (Section~\ref{sec:applications-flat}), the additional contextual information are
\emph{whole-context} properties. They either capture properties of the execution environment or
affect the whole free-variable context.

In Chapter~\ref{ch:flat}, we develop a \emph{flat coeffect calculus} which gives us an unified way
of tracking \emph{whole-context} properties. The calculus is parameterized by a \emph{flat coeffect
algebra} that captures the algebraic properties of contextual information. Concrete
instances of flat coeffects include Haskell's implicit parameters, whole-context liveness and
whole-context dataflow.

We define a type system for the calculus (Section~\ref{sec:flat-calculus}),
discuss how to resolve the ambiguity inherent in context-aware programming for sample
languages we consider (Section~\ref{sec:flat-unique}) and study equational properties of the
calculus (Section~\ref{sec:flat-syntax}). In the flat coeffect calculus, $\beta$-reduction and
$\eta$-expansion do not generally preserve the type of an expression, but we identify two
conditions when this is the case.

% --------------------------------------------------------------------------------------------------

\paragraph{Structural coeffect calculus.}

~ The second category of context-aware systems discussed in Section~\ref{sec:applications-structural}
captures \emph{per-variable} contextual properties. The systems discussed here resemble substructural
logics, but rather than \emph{restricting} variable use, they \emph{track} how the variables are used.

We unify systems with \emph{per-variable} contextual properties in Chapter~\ref{ch:structural},
which describes our \emph{structural coeffect calculus} (Section~\ref{sec:struct-calculus}).
Similarly to the flat variant, the calculus is parameterized by a \emph{structural coeffect algebra}.
Concrete instances of the calculus track bounded variable use (\ie~how many times is a variable
accessed), dataflow properties (how many past values are needed) and liveness (\ie~can variable
be accessed).

The structural coeffect calculus has desirable equational properties (Section~\ref{sec:struct-syntax}).
In particular, type preservation for $\beta$-reduction and $\eta$-expansion holds for all instances of the
structural coeffect calculus. This follows from the fact that structural coeffect
associates contextual requirements with individual variables and preserves the connection by
including explicit structural rules (weakening, exchange and contraction).

% --------------------------------------------------------------------------------------------------

\paragraph{Semantics and safety.}

Both of the coeffect calculi are parameterized and can be instantiated to obtain a language with
a type system that tracks concrete notion of context dependence. The coeffect calculus can be seen
as a framework for building concrete domain-specific context-aware programming languages. In
addition to the type system, the framework also provides a way for defining the semantics of
concrete domain-specific context-aware languages, guides their implementation and simplifies
safety proofs.

This is done using a \emph{comonadically-inspired} semantics. We generalize the category-theoretical
dual of monad to \emph{indexed comonad} and use it to define the semantics of context-aware
programs for both the flat coeffect calculus (Chapter~\ref{ch:semantics}) and the structural coeffect
calculus (Section~\ref{sec:struct-semantics}). Due to the ambiguity inherent in contextual properties,
we define the semantics over a typing derivation, but we also give a domain-specific algorithm for
choosing a unique typing derivation for each of our sample languages (Section~\ref{sec:flat-unique}
and Section~\ref{sec:struct-unique}).

We use comonads in a syntactic way, following the example of Haskell's use of monads and treat
it as a \emph{translation} from source context-aware programming language to a simple target
functional language. The target language includes uninterpreted comonadically-inspired primitives
whose concrete reduction rules need to be provided for each concrete context-aware deomain-specific
language. We give concrete definitions and prove type safety for three sample context-aware
languages (Section~\ref{sec:semantics-proofs} and Section~\ref{sec:struct-transl}). Together with
well-typedness of our translation, these guarantee that ``well-typed context-aware programs do
not go wrong''.

% --------------------------------------------------------------------------------------------------

\paragraph{Interactive essay.}

A part of the thesis is an interactive web-based essay (\url{http://tomasp.net/coeffects}),
which implements three simple context-aware languages. In addition to more traditional prototype
language implementations, the essay provides a number of novel features. As usual, it lets users
write and evaluate sample programs, but it also lets them explore the details of the coeffect
theory -- this includes an interactive way of exploring typing derivations and program semantics.
The essay also provides a number of case studies that highlight interesting aspects of the theory.

The implementation (Chapter~\ref{ch:impl}) serves two purposes. First, it links together all parts
of the theory developed in this thesis (Section~\ref{sec:impl-case}). It uses the two coeffect calculi
to build a language framework that is then instantiated to implement three concrete context-aware
languages. The  framework provides the type system and translation (based on the
comonadically-inspired semantics) that is used for program execution. This shows the practical
benefits of using the theory of coeffects as a basis of real-world programming languages.

Second, the implementation uses novel kind of medium (Section~\ref{sec:impl-essay}) to make our work
\emph{accessible} and \emph{explorable}. Building a production-ready programming language is outside
the scope of the  thesis, so instead, our goal is to explain the theory and inspire authors of
real-world programming languages to include support for context-aware programming, ideally using
coeffects as a sound foundation. For this reason, we make the implementation accessible over web
without requiring installation of specialized software and we make it explorable to encourage
active reading.

% =================================================================================================
%
%     #####
%    #     # #    # #    # #    #   ##   #####  #   #
%    #       #    # ##  ## ##  ##  #  #  #    #  # #
%     #####  #    # # ## # # ## # #    # #    #   #
%          # #    # #    # #    # ###### #####    #
%    #     # #    # #    # #    # #    # #   #    #
%     #####   ####  #    # #    # #    # #    #   #
%
% =================================================================================================

\section{Summary}
\label{sec:conc-conclusions}

We believe that understanding what programs \emph{require} from the world is equally important as
how programs \emph{affect} the world.

The latter has been uniformly captured by effect systems and monads. Those provide not just
technical tools for defining semantics and designing type systems, but they also \emph{shape}
our thinking -- they let us view seemingly unrelated programming language features (exceptions,
state, I/O) as instances of the same concept and thus reduce the number of distinct language
features that developers need to understand.

This thesis aims to provide a similar unifying theory and tools for capturing context-dependence
in programming languages. We showed that programming language features (liveness, dataflow,
implicit parameters, etc.) that were previously treated separately can be captured by a common
framework developed in this thesis. The main technical contribution of this thesis is that it
provides the necessary tools for programming language designers -- including parameterized type
systems, categorical semantics based on indexed comonads and equational theory.

If there is a one thing that the reader should remember from this thesis, it is the fact that
there is a unified notion of \emph{context}, capturing many common scenarios in programming.
